\section{Introduction}
\label{sec:intro}

DEVise is a data exploration system that allows users to visualize and 
browse tabular data. The DEVise framework is intended for use in an 
environment where multiple users wish to analyze data in an interactive
and collaborative manner. In a typical application of DEVise, users would
want to examine data in several levels of detail, understand relationships
between various attributes and visualize the data in multiple (logical)
dimensions. DEVise has simple but extremely powerful mechanisms to create
sophisticated visualizations and allows interactive querying of the 
visualized data.

There are two aspects to visualizing data using DEVise. The first is 
creating a {\it visual presentation}. This is the process of determining
which attributes of data to visualize and how they would appear in 
relationship with one another. For instance, the user may want to plot
attribute Y v/s X on a bar chart with attribute Z indicated by the color
of the bar. The user may also want to plot attribute W v/s X and link
the two charts so that the X axes show the same ranges of values. DEVise
provides visual idioms such as {\it Visual Links}, {\it Record Links}
and {\it Cursors} to establish visual and logical relationships between
multiple charts.
After the presentation is created, the user can interactively browse the data
by zooming in on a region of the chart, scrolling and so on. Now, in a 
collaborative environment, users would want to share a visualization
and even modify it concurrently. Hence, it becomes essential to understand
an {\bf existing} visual presentation. There is a lot of meta-information
associated
with a visualization and it is important to be able to convey all of it in an
unambiguous and complete manner. This thesis addresses the problem of 
representing the meta-information about a DEVise presentation in an 
intuitive manner, {\bf visually}. We call this a {\it meta-visualization} i.e.
visualization of data about a  visual presentation. We have developed a tool
to aid a DEVise user in creating as well as deciphering visual presentations
of data. 

The remainder of the report is organized as follows. Section \ref{sec:devise}
gives a brief introduction to the visual idioms of DEVise with an example
and motivates the need for a meta-visualization tool.  Section \ref{sec:issues}
discusses the issues in creating an intuitive meta-presentation
framework. Section \ref{sec:impl} describes the tool we have developed
and how it can be used to understand a presentation. We discuss some ideas for
future work in Section \ref{sec:future} and conclude in Section \ref{sec:concl}. 


