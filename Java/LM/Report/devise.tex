\section{An overview of the DEVise framework}
\label{sec:devise}

We present a brief introduction to the visual idioms in DEVise. We focus
on the currently implemented version.

The unit of visualization in DEVise is called a {\it chart}. Visualization 
of a chart is based on {\it mapping} each data record of a table
to a visual symbol on the screen. The source table is an ascii
or binary file with a specified schema. We call this table a {\it TData}.  The
result of applying a mapping to a TData is {\it GData} which is a high-level
representation of what is to be painted on the screen. A GData record has
visual attributes such as {\it x, y, size, color, pattern orientation and symbol-type}. For instance, if data has to be represented as a bar chart, then 
the symbol type would be {\it bar} and x and y would specify the location and
height of the bar. The actual attributes that are mapped to x and y are
specified by the mapping. The range of GData attributes visible in a chart
is called its {\it visual filter}. A visual filter is a selection query on 
the GData, whose result is all records with values of attributes lying with
the ranges specified by the filter. In fact, every DEVise chart is 
the result of a query on the underlying data records. 

One of more charts may be organized into windows. DEVise supports automatic
layout modes such as {\it grid, stack and overlay(pile)}. Grids and stacks 
have no query-related semantics but aid in grouping related charts together. 
Overlays have the  semantics that they have a implicit visual link on x and y 
attribute. 

Cursors and links are mechanisms that can create an interdependency 
between the queries corresponding to different charts. We do not discuss
the formal semantics of these mechanisms here but instead present an 
informal description. For a more detailed formal treatment the reader is
referred to the \cite{ref:sigmod}.

A cursor allows the visual filter of one view to appear as a highlight in
another view. A cursor is useful to zoom in on a small portion (source) of a
larger view(destination). The zoomed in portion appears as the highlight
in the destination. A visual link between two views ensures that their visual 
filters show identical ranges of the linking attributes. Currently, DEVise 
supports visual links on x and y attributes of GData. If two views are linked
by a visual link on x, when a user zooms or scrolls in one of them, the other
updates itself to show the same range of x. A record link from view V1 to view
V2 on attribute A of TData means that the projection of A from records
displayed in V2 is a subset of the projection of A in V2. 
(The subset relationship comes from the fact that some of the records may lie 
outside the visual filter of V2). Here, V1 is said to be the master and V2 
a slave of the record link. Among other things, a record link can be used to
view different attributes of the same data records in two views. 

To understand these concepts better, we use the visualization shown in
Fig. \ref{fig:vis}. We use this example throughout the report.

%%%!!! Fig here and explanation 
\begin{figure}[tbp]
\centerline{\psfig{figure=vis.ps, width=6in}}
\caption{\label{fig:vis}{A DEVise Visualization}}
\end{figure}

This visualization shows information about documents from different sources
(A, B, C, D and E) and categories (A, B, C, D). The documents are from 
years 1988 to 1997 and are ranked. The "Rank vs Title" view shows all 
documents in yellow. All documents organized by year and category are
shown by "Category v/s Year" view in DEViseWn0 in red. All documents organized
by category and source are shown by "Category v/s Source" view in red. (Note
that multiple documents from a given category and year (or source) are plotted 
with randomization so that they appear as a cloud instead of a single point).
The window DEViseWn0 actually contains three views overlaid so that 
they have an implicit visual link on x and y attributes. The green rectangles
highlighting some of the records in DEViseWn0 and DEViseWn4 are because
of overlays. Fig. \ref{fig:visunpiled} shows the same visualization with all
overlays "unpiled". 

%%%!!! Fig here and explanation 
\begin{figure}[tbp]
\centerline{\psfig{figure=visunpiled.ps, width=6in}}
\caption{\label{fig:visunpiled}{Visualization in Fig. \ref{fig:vis} with no overlays}}
\end{figure}

There is a cursor between the "Category v/s Year" views in DEViseWn0 and 
CatYear windows. All records of a year and category are shown in the
"CatYear" window in red (source of the cursor). The same documents are
highlighted by the yellow  the "Category v/s Year" view (destination of the
cursor). The yellow cursor in DEViseWn0 can be moved around and the 
records in the CatYear window change appropriately. These documents are shown
in navy blue in the "Rank v/s Title" view.  They are also highlighted in white
in the "Category v/s Source" view. Thus, we have a visualization which can
be navigated and interactively queried in multiple dimensions.

Another cursor in "Rank v/s Title" view selects one out of all documents. 
The title and other information of this document is shown in the "TITLE" and
"TEXTINFO" views.  This document is shown as the single tall line in the
"Rank v/s Title(Master)" view. There is a record link from the 
"Rank v/s Title(Master)" view(master) to views in "TITLE" and "TEXTINFO" window
and a number of other views(slaves). Only those records seen in the master
view can appear in the slave views. All other documents from the same year and
category are highlighted by the green rectangles in the "Category v/s Year" 
view and "Category v/s Source" views. 

The "Rank v/s Year" and "Rank v/s Source" gives information about the 
quality (defined by rank) of the documents from different sources and years.
These views are linked by a visual link on the Y axis which means that
the two views show the same range of ranks. 

This presentation is an extremely powerful browser for the document 
information. Hopefully the reader must have realized that a sophisticated
visual presentation has a lot of underlying semantics that is not apparent in
the presentation itself. If the creator of this presentation wishes to
collaborate with a colleague, the latter has to understand how links and
cursors behave so as to navigate and enhance the presentation. This is the
motivation for the work presented in this thesis.  
















