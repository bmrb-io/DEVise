\section{Issues}
\label{sec:issues}

In order to create a visual presentation, a user must determine the following:
\begin{itemize}
\item the visual organization of charts and windows so that they are laid 
out on the screen in a visually useful manner.
\item the attributes to visualize and the form of display such as bar graph,
scatter plot etc. In order words, the mapping of the TData to GData has
to be chosen. Note that the mapping can be changed at any time, once
the chart is created.
\item the navigational behavior of the charts i.e. what should happen when 
a zoom/scroll operation is done in any one of them. This "browser" aspect of
the presentation involves linking two or more views appropriately. In some 
cases helper views may be required to obtain the desired behavior. The user may
wish to hide these views in the final presentation but link/cursor information
must be explicitly shown in any meta-visualization.
\end{itemize}

A very important distinction between a actual visual presentation of data
and a visual representation of the meta-information is that in the latter
data need not be drawn. Hence it can be much faster to change the look and feel
of the presentation without evaluating the query to draw the data repeatedly.
However, the meta-presentation must contain all information necessary to get 
the required visualization.

\subsection{Views as Tables}

Recall that the key idea in DEVise is to map each record of a table to
a symbol. The symbol can be arbitrarily complex, such as an image or 
a video (though we do not support it yet) or a text document. Going a 
step further, a symbol can be a DEVise view itself. Thus an arrangement
of two views in a window can be thought of as mapping a table
of two records to GData. Each record would contain all information
to create one view, such as x, y, background and foreground color etc.
In main memory the record can  contain a pointer to a view object and
on saving to a file, this can be replaced by a object id or name. The
properties of a View symbol would be its TData, Mapping and Visual Filter
and some auxiliary information such as axes labels and title. Generalizing
this to a hierarchy, we can place views in an arbitrary layout on screen.

\subsection{Separation of visual representation from the back end}

The visual representation of the visual primitives of DEVise presents
some very tough user-interface issues. There may be multiple ways
to represent the same meta-information depending on the complexity of
the visualization or the sophistication of the user. So it is important
to separate the back end data structures which deal with the semantics of the 
idiom from its visual representation. This makes the representation 
extensible and easily modifiable. For instance, a overlay in DEVise 
lays out multiple views at exactly the same location. This makes it
hard to distinguish the links that connect a view in the overlay to others.
However, in a meta-representation, the overlay can be represented as a
cascade of views each offset by a small amount from the preceding one. Thus,
some layout information (actual coordinates of views) is lost, gaining
a more intuitive meta-representation of information.
