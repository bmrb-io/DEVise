\section{Description of the meta-visualization tool}
\label{sec:impl}

In this section, we describe the look and feel of the tool we have
developed to create and understand DEVise visualizations. The
tool has been implemented in Java. It is currently a standalone tool
which means that a DEVise presentation created through it cannot be actually 
visualized. This will be possible once it is unified into the DEVise 
client-server architecture.

The top level entity in our tool is a Session. A session
has its own name-space and so multiple sessions can contain views with
the same names. A session can be made persistent by saving to a file.
The unit of visualization is a View. A view is a hierarchical structure
which can contain other views. The tool provides automatic layout
modes to place a view within its parent view or to overlay all children
of a view. 

\begin{figure}[tbp]
\centerline{\psfig{figure=vislayout.ps, height=8in}}
\caption{\label{fig:vislayout}{A Meta-Visualization}}
\end{figure}

Fig. \ref{fig:vislayout} shows a meta-representation of the layout 
of the visualization presented in Fig. \ref{fig:vis}. The correspondence
between a view and its representation is quite intuitive here. Notice
that overlays have been represented as stacks of rectangles. This
information is not evident in the original visualization without some
effort.

A visual link between two views is shown by a line segment between
their centers. 
A record link is represented as a directed line segment from
the master to the slave. A cursor is very intuitively represented with
a rectangle to indicate the destination. All existing links and cursors
in a visualization are shown on a side panel. Selecting a link/cursor on
its side panel highlights the corresponding segment(s) in the display window. 

Fig. \ref{fig:vislinks}, Fig. \ref{fig:reclinks} and Fig. \ref{fig:viscur}
shows visual links, record links and cursors for the document visualization
described earlier. All links can also be shown simultaneously.

\begin{figure}[tbp]
\centerline{\psfig{figure=vislinks.ps, height=8in}}
\caption{\label{fig:vislinks}{A meta-representation of visual links}}
\end{figure}

\begin{figure}[tbp]
\centerline{\psfig{figure=reclinks.ps, height=8in}}
\caption{\label{fig:reclinks}{A meta-representation of record links}}
\end{figure}

\begin{figure}[tbp]
\centerline{\psfig{figure=viscur.ps, height=8in}}
\caption{\label{fig:viscur}{A meta-representation of cursors}}
\end{figure}

In the presence of links between overlaid views, the representation is
slightly different. Fig. \ref{fig:pilelinks} shows an example (not from
the visualization in Fig. \ref{fig:vis}). The entire meta-visualization
for links and cursors for Fig. \ref{fig:vis} is shown in
Fig. \ref{fig:alllinks}.

\begin{figure}[tbp]
\centerline{\psfig{figure=vispilelink.ps, height=8in}}
\caption{\label{fig:pilelinks}{Representation of links between overlaid views}}
\end{figure}

\begin{figure}[tbp]
\centerline{\psfig{figure=alllinks.ps, height=8in}}
\caption{\label{fig:alllinks}{ The meta-visualization for Fig. \ref{fig:vis}}}
\end{figure}

Mappings and TData are represented as strings. In future a more intuitive
representation may be used. Similar to links, mappings and TData
tables are shown in  side panels and selecting one of them highlights
all views that are associated with it. Note that mappings and tdata can
exist without being associated with views. A view must necessarily
have a mapping/tdata only when the data has to be actually visualized.

\subsection{Layout Editor}

We have designed an editor to create a custom layout of views. Views
are drawn as rectangles using rubberbands. Operations such as resizing,
moving and alignment are supported. Multiple views can be grouped together to
move/resize them as a single entity. 

A special kind of grouping called a {\it rivet } has been developed to
restrict the movement/resizing of views in certain ways. A rivet joins
edges parallel to each other such that that they cannot move relative to
each other in a direction perpendicular to themselves.
Thus, the distance between the edges is fixed. The riveted edges can however
move parallel to each other.
Fig. \ref{fig:visrivet} shows riveted edges with a white cloud around the
edges. Moving the horizontal rivets changes the relative heights of views
keeping the total height constant.  The vertical rivets can change the relative
width of the views keeping the respective edges aligned. 

\begin{figure}[tbp]
\centerline{\epsfbox{visrivet.ps}}
\caption{\label{fig:visrivet}{A riveted layout}}
\end{figure}


