
%  ========================================================================
%  DEVise Data Visualization Software
%  (c) Copyright 1992-1996
%  By the DEVise Development Group
%  Madison, Wisconsin
%  All Rights Reserved.
%  ========================================================================
%
%  Under no circumstances is this software to be copied, distributed,
%  or altered in any way without prior permission from the DEVise
%  Development Group.

%%%%%%%%%%%%%%%%%%%%%%%%%%%%%%%%%%%%%%%%%%%%%%%%%%%%%%%%%%%%

%  $Id$

%  $Log$

%%%%%%%%%%%%%%%%%%%%%%%%%%%%%%%%%%%%%%%%%%%%%%%%%%%%%%%%%%%%

% \documentstyle[doublespace,11pt]{article}
\documentstyle[11pt]{article}
\renewcommand{\textwidth}{5.9in}
\oddsidemargin 0.30in
\evensidemargin 0.30in
\newenvironment{codespace}{\baselineskip 6pt }{}

\input{epsf}
% \pagestyle{empty}

\title{\bf{DEVise in 3D}}
\author{Yu Lun Chen}
\date{May 22, 1996}


\begin{document}
\maketitle

\section{Abstract}

``A Data Exploration via Visualization Environment'' ({\tt DEVise}) is a
system that supports a paradigm called {\em visual queries}.  A
{\em visual query} is an abstract description of a collection of 
data that are to be mapped to the screen.  This query
is like a template that can be applied to any data set with the
appropriate schema.

The main focuses in the earlier work for this project was on the 
exploration and visualization of the data set in 2D space with 2D 
shapes.  The purpose of this project is to extend the DEVise 
system, so it will support data set in 3D space with 3D 
shapes (e.g. block, sphere, and/or 3D vector).

% ------------------------------------------------------------
\section{Introduction}

This document is divided into four parts. The first part 
is a brief description of {\tt DEVise}.  The second part describes the
implementation of the 3D space and the 3D shapes.  The third part
describes the integration of the 2D and 3D interfaces.  And the 
last part describes some thoughts on how the implementation can 
be enhanced.

% ------------------------------------------------------------
\section{An overview of {\tt DEVise}}

Visualization in {\tt DEVise} is divided into several stages.  
These stages are:

\begin{enumerate}
	\item Using a user defined ``schema'' to import a data file 
		into {\tt DEVise} as a textual data (TData) set.
	\item Using a ``mapping'' to map the textual data into graphical
		data (GData).
	\item Selecting a view and assigning a visual filter the 
		displaying GData.
	\item Arranging views in windows.
	\item Creating visual links among visual links and axes.
\end{enumerate}

Followings are the descriptions for some of the key terms that
will be used throughout the discussions.

\subsection{Schema}

The schema file describes the layout of the ASCII input file.  It
maps the actual data set to the internal TData set.  It includes the
name, comment, separator, data type, data attributes, etc.

\subsection{TData}

The texture data (TData) is the internal representation of the 
actual data in {\tt DEVise}.  Because the uniform representation
there is a standard way of treating different types of input files.

\subsection{Mapping}

The mapping has four functions.  First, it defines the interface for
the converting textual data (TData) attributes into graphical data
attributes (GData).  Second, mapping serves as a handle to the
TData/GData pair.  Third, mapping creates the GData when the mapping
is created.  Fourth, it keeps track of the changes in GData
attributes at run time.  This allows compiler to generate the dynamic
attributes for GData.

\subsection{GData}

A graphical data (GData) is a graphical representation of TData.  
Each record of TData is mapped to a record in GData.  The attributes 
of GData are: $value_{1a}$, $value_{1b}$, $value_{1c}$, {\em color}, 
{\em size}, {\em pattern}, {\em orientation}, {\em shape}, 
$shape_{width}$, $shape_{height}$, and $shape_{depth}$.

\subsection{View}

A view is used to display GData, which fall within the range of the
visual filter and in a window.  It is implemented as a leaf window 
in the hierarchy of windows.  A view also keep tracks of the number
of dimension for the mapping.

Whenever a view's {\em visual filter}\footnote{A visual filter defines a
query over the GData attributes.} or a {\em camera}\footnote{A camera
defines the eye coordinates in 3D space.} is changed, its window will
be updated.  In addition, view is responsible for deciding how
overlapping GData is displayed.

\subsection{Window}

``View'' draws GData on a window.  Window is the hierarchically
structured output devices visible within a display.  It is
responsible for displaying data and arranging views.

\subsection{Visual Link}

A visual link implements the links among axes.  It links the visual
filters with multiple views.  Whenever a view's visual filter or
camera is changed, the visual link is notified and synchronized
all linked views.

% ------------------------------------------------------------
\section{Implementation of 3D stuffs}

In the previous version {\tt DEVise}, GData only supports {\it 
$value_{1a}$, $value_{1b}$, color, size, pattern, orientation, and 
shape}.  The purpose of this project is to extend the GData 
support to another dimension--``$value_{1c}$''.  To accommodate for the
extra parameters, new visual tools and functionalities are added to
the system.  The first portion of this section
is the derivation of the viewing system which have the appropriate
transformations from the three dimensional
(3D) world (or user) space to the two dimension (2D) viewing (or 
screen or camera or eye) space.  The second portion of this 
section presents a simple 
``volume clipping'' algorithm.  The final portion of this section 
deals with the creation of 3D shapes.

\subsection{3D space}

A viewing system is a set of 
transformations which will map points in world coordinates space onto
the viewing surface.  Such a system allows a user to specify a
projective transform and it enables the view plane to be positioned
anywhere in world coordinates space.  Fig.~\ref{fig:view_sys}
illustrates the viewing system.

\begin {figure}[htbp]
 \centerline{
% \epsfxsize=300pt
 \epsffile{3d-view_sys.eps}
 }
 \caption{The relationship between world coordinate system and the
 	view coordinate system.}
 \label{fig:view_sys}
\end   {figure}

The viewing system orients the axes so that, when the viewer's line
of sight is directed toward the origin of the $X_w$, $Y_w$, $Z_w$
axis system, the $Z_c$ axis points in the direction of the line of
sight, the $X_c$ axis points to the right, and the $Y_c$ axis points
upward.  The transformation, $T_r$, for mapping $(X_w, Y_w, Z_w)
\rightarrow (X_c, Y_c, Z_c)$ is composed of two steps:

\begin{enumerate}
	\item The camera coordinates $(X_c, Y_c, Z_c)$ associated with 
		a point having standard coordinates $(X_w, Y_w, Z_w)$.
	\item The screen coordinates $(X_s, Y_s)$ associated with a 
		specific set of camera (or eye) coordinates $(X_c, Y_c,
		Z_c)$.
\end{enumerate}

\subsubsection{Viewing transformation}

In the first step, the viewing transformation $T_{view}$ transforms
points in the world coordinates system into the view coordinate
system:

  \[ [x_v \  y_v \  z_v \  1] = [x_w \  y_w \  z_w \  1] \cdot T_{view}   \]

The $T_{view}$ required to take the objects from a world coordinate
system into a view coordinate system are derivated from the following
steps.

\begin{enumerate}
	\item Translate the focus (or the aim) of the world coordinate
		system to $(-f_x, -f_y, -f_z)$.

		\[ A = \left[  \begin{array}{cccc}
					1 & 0 & 0 & 0 \\
					0 & 1 & 0 & 0 \\
					0 & 0 & 1 & 0 \\
					-f_x & -f_y & -f_y & 1 \\
					\end{array}
			  \right] \]

	\item Translate the world coordinate system to 
		$(\rho\sin\phi\sin\theta, \rho\cos\theta, \rho\sin\phi\cos\theta)$,
		the position of the camera (or view point).  The axes of the
		view coordinate system and axes of the world coordinate
		system are parallel.  $A$ is the matrix for this
		transformation.

		\[ B = \left[  \begin{array}{cccc}
					1 & 0 & 0 & 0 \\
					0 & 1 & 0 & 0 \\
					0 & 0 & 1 & 0 \\
					-\rho\sin\phi\sin\theta & -\rho\cos\phi & 
					-\rho\sin\phi\cos\theta & 1
					\end{array}
			  \right] \]

	\item The next step is to rotate the coordinate system
		about $Y_c$ axis counter-clockwise by $\theta$, so the 
		negative $Z_c$ axis intersect the $Y_w$ axis.  The
		transformation matrix which accomplishes this is:

		\[ C = \left[  \begin{array}{cccc}
					\cos\theta & 0 & \sin\theta & 0 \\
					0 & 1 & 0 & 0 \\
					-\sin\theta & 0 & \cos\theta & 0 \\
					0 & 0 & 0 & 1 \\
					\end{array}
			  \right] \]

	\item  The next step is to rotate the coordinate system through
	$90^{\circ} -\phi$ counterclockwise about the $X_c$ axis.  This makes
	the $Z_c$ axis pass through the origin of the world coordinate
	system.  The transformation matrix which accomplishes this is:

		\[ \begin{array}{ll}
			D & = \left[  \begin{array}{cccc}
				1 & 0 & 0 & 0 \\
				0 & \cos(90 - \phi) & \sin(90^{\circ}-\phi) & 0 \\
				0 & -\sin(90^{\circ}-\phi) & \cos(90^{\circ}-\phi) & 0 \\
				0 & 0 & 0 & 1 \\
					\end{array}
			  \right] \\ \\
			& = \left[  \begin{array}{cccc}
				1 & 0 & 0 & 0 \\
				0 & \sin\phi & \cos\phi & 0 \\
				0 & -\cos\phi & \sin\phi & 0 \\
				0 & 0 & 0 & 1 \\
					\end{array}
			  \right] 
		  \end{array}
		\]


	\item The final step is to convert to a left-handed system as
		described above.

		\[ E = \left[  \begin{array}{cccc}
					1 & 0 & 0 & 0 \\
					0 & 1 & 0 & 0 \\
					0 & 0 & -1 & 0 \\
					0 & 0 & 0 & 1 \\
					\end{array}
			  \right] \]
\end{enumerate}

After multiplying the four matrix together, they give the net
transformation matrix which is the viewing transformation, $T_{view}$.

  \[ \begin{array}{ll}
	T_{view} & = A \cdot B \cdot C \cdot D \cdot E \\ \\
		& = \left[ \begin{array}{cccc}
				t_{11} & t_{12} & t_{13} & 0 \\ 
				t_{21} & t_{22} & t_{23} & 0 \\
				t_{31} & t_{32} & t_{33} & 0 \\
				t_{41} & t_{42} & t_{43} & 1
				\end{array}
		\right] 
	\end{array}
  \]

where: 

\[ \begin{array}{l}

	t_{11} = \cos\theta \\
	t_{12} = -\cos\phi\sin\theta \\
	t_{13} = -\sin\phi\sin\theta \\

  	t_{21} = 0 \\
  	t_{22} = \sin\phi \\
  	t_{23} = -\cos\phi \\

  	t_{31} = \sin\theta \\
  	t_{32} = \cos\phi\cos\theta \\
  	t_{33} = \cos\theta\sin\phi \\

  	t_{41} = -f_y\cos\theta + f_z\sin\theta \\
  	t_{42} = -((f_y + \rho\cos)\sin\phi) + \cos\phi(f_z \cos\theta +
			\rho\sin\phi + f_x \sin\theta)  \\
  	t_{43} = \cos\phi(f_y + \rho\cos\phi) + \sin\phi(f_z \cos\theta +
			\rho\sin\phi + f_x \sin\theta)
   \end{array}			
\]

If the world coordinates $(X_w, Y_w, Z_w)$ of a point are known, the
camera coordinates $(X_c, Y_c, Z_c)$ may be obtained by the following
equation.

   \[ [X_c \  Y_c \  Z_c \  1] = [X_w \  Y_w \  Z_w \  1] \cdot T_{view} \]

\subsubsection{Perspective projection}

The second step for mapping a point onto the display screen is
``perspective projection''.  ``Perspective projection'' is a finite
resolution approximation on the infinitely precise projection.   There
are two required elements for performing this projection: view point
and the plane of projection.  Fig.~\ref{fig:persp_view_3d} illustrates
the idea, and Fig.~\ref{fig:persp_view_2d} shows the similar triangles
for deriving the equations.

\begin {figure}[htbp]
  \centerline{
%  \epsfxsize=300pt
  \epsffile{3d-persp_view_3d.eps}
  }
  \caption{Relationship between camera coordinate system and display screen.}
  \label{fig:persp_view_3d}
\end   {figure}

\begin {figure}[htbp]
  \centerline{
%  \epsfxsize=300pt
  \epsffile{3d-persp_view_2d.eps}
  }
  \caption{Similar triangles.}
  \label{fig:persp_view_2d}
\end   {figure}


The equations for mapping $(X_c, Y_c, Z_c) \rightarrow (X_s, Y_s)$ are
as follow:

  \[ X_s = \frac{dvs \cdot X_c}{Z_c} \]

  \[ Y_s = \frac{dvs \cdot Y_c}{Z_c} \]

where ``dvs'' is the distance from viewing system.

\subsection{Volume clipping}

``Volume clipping'' is the set of all points in the ``world''
coordinate system which can appear on screen under certain projection
method.  Perspective projection is used in this project.  Because the
natural of the perspective projection, there are two degenerate cases.

\begin{enumerate}
	\item When a point is very close to the screen, the denominator
		becomes very small.  This results in wrong or distorted
		pictures on screen.
	\item When a point is behind the viewing screen or outside of 
		the clipping cone, it should not be displayed.
\end{enumerate}

This project uses a simple but effective volume clipping method.  The
idea is to implement volume clipping with special kind of perspective
view volume which uses $90^{\circ}$ between the clipping axes.  
The reason is that it's 
easier to computing the intersecting points between the clipping cone
and the view coordinate axes.  The intersecting points
are the clipping points on the axes.  In this case, the clipping
points are: x = z, y = z, x = -z, and y = -z.  

The shortcoming of this clipping method is clipping points do
not define an arbitrary directed plane.  However, the speed of this
method will out weight the constraints.

\subsection{3D shape}

3D shapes, besides 3D arrow, are composed of triangular or
parallelepiped facets.  The simplest example is a cube which has six 
parallelepiped facets.  The data structure uses the following syntax.
\\

\[
\begin{tt}
\begin{array}{lll}
	x_0 & y_0 & z_0 \\
	x_1 & y_1 & z_1 \\
	x_2 & y_2 & z_2 \\
	x_3 & y_3 & z_3 \\
	\ldots & & \\
	\ldots & & \\
	x_n & y_n & z_n \\
	line_{vert_0} & line_{vert_1} & \\
	line_{vert_3} & line_{vert_1}  &\\
	\ldots & & \\
	\ldots & & \\
	line_{vert_i} & line_{vert_j} & \\
\end{array}
\end{tt}
\]

In order to keep track of the face normal for each face, the order of
the edges on a face is in counterclockwise direction.  The reason is
to keep track of the face normal.  The direction of face normal is not
used in this project, but it is important for the hidden line removal
function which might be implemented in the future.

% ------------------------------------------------------------
\section{Integration of 2D and 3D interfaces}

In order to ease the transition from the 2D user interface to the 3D
user interface, same keystrokes are mapped to similar functions.  The
following table summaries the keystrokes and the corresponding
functions. \\

\begin{center}
\begin{tabular}{|c|c|} \hline
  Keystroke(s) & Function \\ \hline \hline
  6 & Move toward +X \\ \hline
  4 & Move toward -X \\ \hline

  8 & Move toward +Y \\ \hline
  2 & Move toward -Y \\ \hline

  7, 1 & Zoom in \\ \hline
  9, 3 & Zoom out \\ \hline

  i, I & Move toward +Z \\ \hline
  o, O & Move toward -Z \\ \hline

  f, F & Un-Fix/fix the focus \\ \hline
  r, R & On/Off to spherical coordinate \\ \hline
\end{tabular}
\end{center}

Another Tcl/tk tool is incorporated into the main {\tt DEVise} 
window in order
to show the current modes and the current position.  Under the {\em
View} option, ``3D query'' has been added as a direct control to the camera,
focus, and the toggle switches for focus and spherical coordinate
system.

% ------------------------------------------------------------
\section{Conclusions and recommendations}

There is one more thing that is needed  in order to make {\tt DEVise} 
a true 3D package.  The missing link is the hidden line removal
function.  The hidden line function will improve the visibility of
different objects.   In the current system, it's very easy to get confuse
about the orientation of adjacent objects.  One suggestion will be to
implement the simple Painter's algorithm~\cite{FDFH92} which is 
probably sufficient for the general usage.

Besides hidden line removal function, one can always add more 
shapes to the shape library.  It would be nice if there are
corresponding 3D shapes for each existing 2D shapes.  Then a user 
can switch back and forth between the dimensions without changing too
much information.

\begin{thebibliography}{99}

 \bibitem{FDFH92} J. D. Foley, A. van Dam, S. K. Feiner, J. F. 
  Hughes, {\em Computer Graphics}, Addison-Wesley Publishing 
  Company, Second eidtion, 1992.
\end{thebibliography}

\end{document}

