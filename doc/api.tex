%  ========================================================================
%  DEVise Data Visualization Software
%  (c) Copyright 1992-1996
%  By the DEVise Development Group
%  Madison, Wisconsin
%  All Rights Reserved.
%  ========================================================================
%
%  Under no circumstances is this software to be copied, distributed,
%  or altered in any way without prior permission from the DEVise
%  Development Group.

%%%%%%%%%%%%%%%%%%%%%%%%%%%%%%%%%%%%%%%%%%%%%%%%%%%%%%%%%%%%
%  $Id$
%  $Log$
%
%%%%%%%%%%%%%%%%%%%%%%%%%%%%%%%%%%%%%%%%%%%%%%%%%%%%%%%%%%%%

\documentstyle[fullpage,verbatim]{article}

\renewcommand{\topfraction}{1.0}
\renewcommand{\bottomfraction}{1.0}
\renewcommand{\textfraction}{0.0}
\advance\intextsep by 5pt

\def\filename#1{{\tt #1}}
\def\code#1{{\tt #1}}
\def\menu#1{{\tt #1}}
\def\term#1{#1}
\def\variable#1{{\tt #1}}

\def\scaleepspic[#1]#2#3{
\begin{figure}[htb]
\centering\leavevmode\epsfxsize=#1\epsfbox{#2}
\caption{#3}
\end{figure}
}

\def\fullepspic#1#2{
\begin{figure}[htb]
\centering\leavevmode\epsfxsize=\textwidth\epsfbox{#1}
\caption{#2}
\end{figure}
}

\begin{document}
\title{DEVise Application Programming Interface}
\author{The DEVise Development Group \\
\code{devise@cs.wisc.edu}
}
\date{\today}
\maketitle

\section{Introduction}

The DEVise application programming interface (API) allows a user
interface, application program or other entity to request
visualization services from the DEVise visualization engine. The
requestor ({\em client program}) and the visualization engine ({\em
server}) communicate using the DEVise command language which
implements the API.

This document is intended for client program developers who want to
customize the appearance of DEVise visualization sessions. The API
allows the programmer to specify such visualization characteristics as
placement of visualization windows, layout of views within windows,
and enabling and disabling of statistics displays. Also, the client
program can request information about objects such as Source Data,
Graph Data, and Schemas to build up dialogs for interacting with the
user.

\section{DEVise Command Language}

The commands supported in the DEVise API are divided into eight
categories: Generic, Source Data, Schema, Graph Data, View, Window,
Link and Cursor, and Miscellaneous. Commands in the Generic category
allow the client program to create and remove objects of the other
categories, or to query the parameters used when the objects were
created. See next section for a more detailed description
of the object creation command.

The category-specific commands control the behavior of the
objects. The Miscellaneous category includes commands for querying and
altering the state of the query processor and buffer manager, among
others. The Obsolete category lists some obsolete commands that are
being phased out and should not be used.

A DEVise command consist of a verb (command name) followed by zero or
more input parameters. The parameters may be of type string, integer,
float, or date, but their values are always represented as ASCII
strings. The result of the command is returned as another
string. Error conditions are indicated at the client-server protocol
level, and the returned string is the error message. This document
describes the return values resulting from a successful command
execution. Commands that do not have any other return values return
the value {\tt done}.

The commands are described in the following format.

\bigskip

\noindent
\begin{tabular}{l|p{5in}}
\hline
Command name & Name of the command \\
Description  & What the command does \\
Fingerprints & Command prototype \\
Input parameters & Parameters of the command \\
Return value & Return value or error message \\
\hline
\end{tabular}

\section{Object Creation in DEVise\protect\label{objcreate}}

DEVise stores information about all visualization objects in a common
data structure that has a uniform interface. DEVise uses a three-level
object hierarchy. At the top are {\em categories}: tdata, mapping,
view, window, link, or cursor. Next are {\em classes}. At the bottom
of the hierarchy are {\em instances} whose names share a global name
space.

The following table shows the meaning of ``class'' for each category.
For tdata and mapping, the class names are assigned by the client
program or the user. A view can either be sorted along the X axis
(SortedX) or not (Scatter). Only one window class (TileLayout) is
currently defined (older versions of DEVise defined WinVertical and
WinHorizontal but these classes are subsumed by TileLayout). A link
must belong to the Visual\_Link class and a cursor to the Cursor
class.

\bigskip

\noindent
\begin{tabular}{l|l|p{4in}}
Category & What class means           & Parameters\\
\hline
tdata    & Schema name                & TData name, TData type, parameter \\
mapping  & Client-program assigned    & Tdata name, mapping name, empty,
                                        X attribute, Y, Z, color, size,
                                        pattern, orientation, shape,
                                        shape attribute 0, 1, 2 \\
view     & ``SortedX'' or ``Scatter'' & View name, X low, X high, Y low,
                                        Y high, background color \\
window   & ``TileLayout''             & Window name, X position, Y position,
                                        width, height (all expressed as a
                                        fraction of screen size) \\
link     & ``Visual\_Link''           & Link name, link flag \\
cursor   & ``Cursor''                 & Cursor name, cursor type \\
\hline
\end{tabular}

\bigskip

The command used to create an object of any aforementioned category
and class is simply {\tt create}. The number of parameters varies
according to object category. The types of the parameters are shown in
the table.

The TData type parameter is either UNIXFILE, WWW, or BASICSTAT. These
are the only types of data streams understood by the Devise server at
this point.  The parameter is either a filename (for UNIXFILE), a URL
(for WWW), or the name of a view (BASICSTAT).

The link flag is a binary value with the following values OR'ed: X
(1), Y (2), size (4), pattern (8), color (16), orientation (32), shape
(64). The bits that are one specify the attributes that are used to
link views.

The cursor type is a binary value with X (1) and Y (2) OR'ed. The bits
that are one specify whether the cursor operates in the X-direction,
the Y-direction, or both.

\section{Generic Data Commands}
\noindent
\begin{tabular}{l|p{5in}}
\hline
Command name &{\tt getparam }\\ 
Description &
Return the parameters for a class of a category
 	\\
Fingerprints & strings(string,string)\\
Input Parameters&category name, class name\\
Return Value&class parameters\\
\hline
\end{tabular}
\bigskip

\noindent
\begin{tabular}{l|p{5in}}
\hline
Command name &{\tt create }\\ 
Description &
 Create a new instance of a class 
 	\\
Fingerprints & string(string,string,strings)\\
Input Parameters& Category name , class name , parameter values \\
Return Value& Name of the class created or empty \\
\hline
\end{tabular}
\bigskip

\noindent
\begin{tabular}{l|p{5in}}
\hline
Command name &{\tt destroy }\\ 
Description &
 Destroys the given instance 
 	\\
Fingerprints & string(string)\\
Input Parameters& Instance name \\
Return Value&{\tt done}\\
\hline
\end{tabular}
\bigskip

\noindent
\begin{tabular}{l|p{5in}}
\hline
Command name &{\tt setDefault }\\ 
Description &
 Sets the default parameter values for a class 
 	\\
Fingerprints & string(strings)\\
Input Parameters& List of parameter values \\
Return Value&{\tt done}\\
\hline
\end{tabular}
\bigskip

\noindent
\begin{tabular}{l|p{5in}}
\hline
Command name &{\tt getCreateParam }\\ 
Description &
 Get the parameters of an instance 
 	\\
Fingerprints & strings(string,string,string)\\
Input Parameters& Category name , class name , instance name \\
Return Value& List of parameter values or empty \\
\hline
\end{tabular}
\bigskip

\noindent
\begin{tabular}{l|p{5in}}
\hline
Command name &{\tt changeParam }\\ 
Description &
 Change the parameters for the given instance 
 	\\
Fingerprints & string(string,strings)\\
Input Parameters& Instance name , list of parameter values \\
Return Value&{\tt done}\\
\hline
\end{tabular}
\bigskip

\noindent
\begin{tabular}{l|p{5in}}
\hline
Command name &{\tt changeableParam }\\ 
Description &
 Check whether parameters of instance are changeable 
 	\\
Fingerprints & integer(string)\\
Input Parameters& Instance name \\
Return Value& 0/1 \\
\hline
\end{tabular}
\bigskip

\noindent
\begin{tabular}{l|p{5in}}
\hline
Command name &{\tt getInstParam }\\ 
Description &
 Get the current parameters of the instance 
 	\\
Fingerprints & strings(string)\\
Input Parameters& Instance name \\
Return Value& List of parameter values or empty \\
\hline
\end{tabular}
\bigskip

\noindent
\begin{tabular}{l|p{5in}}
\hline
Command name &{\tt get }\\ 
Description &
 Get the classes corresponding to the given category 
 	\\
Fingerprints & strings(string)\\
Input Parameters& Category name \\
Return Value& List of class names or empty \\
\hline
\end{tabular}
\bigskip

\noindent
\begin{tabular}{l|p{5in}}
\hline
Command name &{\tt get }\\ 
Description &
 Get all the instances belonging to the class 
 	\\
Fingerprints & strings(string,string)\\
Input Parameters& Category name , class name , \\
Return Value& List of instance names or empty \\
\hline
\end{tabular}
\bigskip


\section{Source Data Commands}
\noindent
\begin{tabular}{l|p{5in}}
\hline
Command name &{\tt createTData }\\ 
Description &
Create a TData object from a data source
 	\\
Fingerprints & string(string)\\
Input Parameters&dataSourceName\\
Return Value&{\tt done}\\
\hline
\end{tabular}
\bigskip

\noindent
\begin{tabular}{l|p{5in}}
\hline
Command name &{\tt checkTDataForRecLink }\\ 
Description &
Check the last mapping of the specified view to see whether its TData
 
source agrees with that of all the slave views that belongs to the 
 
same link
 	\\
Fingerprints & boolean(string,string,boolean)\\
Input Parameters&linkName, view name, isMaster\\
Return Value&flag\\
\hline
\end{tabular}
\bigskip

\noindent
\begin{tabular}{l|p{5in}}
\hline
Command name &{\tt dteCheckSQLViewEntry }\\ 
Description &
Check the validity of a SQL query string
 	\\
Fingerprints & string(string,string)\\
Input Parameters&As Clause, query string\\
Return Value&""/Error message\\
\hline
\end{tabular}
\bigskip

\noindent
\begin{tabular}{l|p{5in}}
\hline
Command name &{\tt dteCreateIndex }\\ 
Description &
Create an index 
 	\\
Fingerprints & string(string,string,string,string,string)\\
Input Parameters&tableName,indexName,keyAttributes, dataAttributes,isStandAlone="Yes/No"\\
Return Value&query execution results\\
\hline
\end{tabular}
\bigskip

\noindent
\begin{tabular}{l|p{5in}}
\hline
Command name &{\tt dteDeleteCatalogEntry }\\ 
Description &
Delete a catalog entry indexed by a tableName
 	\\
Fingerprints & string(string)\\
Input Parameters&tableName\\
Return Value&""\\
\hline
\end{tabular}
\bigskip

\noindent
\begin{tabular}{l|p{5in}}
\hline
Command name &{\tt dteDeleteIndex }\\ 
Description &
Delete an index defined on the table
 	\\
Fingerprints & string(string,string)\\
Input Parameters&tableName,indexName\\
Return Value&""\\
\hline
\end{tabular}
\bigskip

\noindent
\begin{tabular}{l|p{5in}}
\hline
Command name &{\tt dteImportFileType }\\ 
Description &
Return the type of a specific data source
 	\\
Fingerprints & string(string)\\
Input Parameters&data source name\\
Return Value&file type\\
\hline
\end{tabular}
\bigskip

\noindent
\begin{tabular}{l|p{5in}}
\hline
Command name &{\tt dteInsertCatalogEntry }\\ 
Description &
Insert an entry into the catalog 
 	\\
Fingerprints & string(string,string)\\
Input Parameters& catalog name, catalog entry value\\
Return Value&""\\
\hline
\end{tabular}
\bigskip

\noindent
\begin{tabular}{l|p{5in}}
\hline
Command name &{\tt dteListAllIndexes }\\ 
Description &
Return all index names for a given table
 	\\
Fingerprints & strings(string)\\
Input Parameters&tablename\\
Return Value&index names\\
\hline
\end{tabular}
\bigskip

\noindent
\begin{tabular}{l|p{5in}}
\hline
Command name &{\tt dteListAttributes }\\ 
Description &
Return all the attribute names for a given table
 	\\
Fingerprints & strings(string)\\
Input Parameters&tablename\\
Return Value&attribute names\\
\hline
\end{tabular}
\bigskip

\noindent
\begin{tabular}{l|p{5in}}
\hline
Command name &{\tt dteListCatalog }\\ 
Description &
Return all the catalog entries for a given catalog, each entry has
 
the form (name, type)
 	\\
Fingerprints & strings(string)\\
Input Parameters&catalog name\\
Return Value&entries\\
\hline
\end{tabular}
\bigskip

\noindent
\begin{tabular}{l|p{5in}}
\hline
Command name &{\tt dteListQueryAttributes }\\ 
Description &
Return all the query attributes for a query
 	\\
Fingerprints & strings(string)\\
Input Parameters&DTE query\\
Return Value&attribute names\\
\hline
\end{tabular}
\bigskip

\noindent
\begin{tabular}{l|p{5in}}
\hline
Command name &{\tt dteMaterializeCatalogEntry }\\ 
Description &
Materialize a given table
 	\\
Fingerprints & string(string)\\
Input Parameters&tableName\\
Return Value&""\\
\hline
\end{tabular}
\bigskip

\noindent
\begin{tabular}{l|p{5in}}
\hline
Command name &{\tt dteReadSQLFilter }\\ 
Description &
Read SQL filter from a given file, and return a SQL statement
 	\\
Fingerprints & string(string)\\
Input Parameters& SQL filter file name\\
Return Value&SQL statement\\
\hline
\end{tabular}
\bigskip

\noindent
\begin{tabular}{l|p{5in}}
\hline
Command name &{\tt dteShowAttrNames }\\ 
Description &
Return the attribute names for a given schema file and a data file
 	\\
Fingerprints & strings(string,string)\\
Input Parameters&schema file, data file\\
Return Value&attribute names\\
\hline
\end{tabular}
\bigskip

\noindent
\begin{tabular}{l|p{5in}}
\hline
Command name &{\tt dteShowCatalogEntry }\\ 
Description &
Return a catalog entry by tablename
 	\\
Fingerprints & string(string)\\
Input Parameters&table name\\
Return Value&entry value\\
\hline
\end{tabular}
\bigskip

\noindent
\begin{tabular}{l|p{5in}}
\hline
Command name &{\tt dteShowIndexDesc }\\ 
Description &
Return the descriptor for a given table's specified index
 	\\
Fingerprints & string(string,string)\\
Input Parameters&tablename,indexname\\
Return Value&descriptor\\
\hline
\end{tabular}
\bigskip

\noindent
\begin{tabular}{l|p{5in}}
\hline
Command name &{\tt getTDataName }\\ 
Description &
Lookup TData name from a given Tdata instance name
 	\\
Fingerprints & string(string)\\
Input Parameters&instance name\\
Return Value&tdata name\\
\hline
\end{tabular}
\bigskip

\noindent
\begin{tabular}{l|p{5in}}
\hline
Command name &{\tt invalidateTData }\\ 
Description &
Invalidate the tdata
 	\\
Fingerprints & string(string)\\
Input Parameters&tdata instance name\\
Return Value&{\tt done}\\
\hline
\end{tabular}
\bigskip

\noindent
\begin{tabular}{l|p{5in}}
\hline
Command name &{\tt saveTdata }\\ 
Description &
Save TData into a file
 	\\
Fingerprints & string(string,string)\\
Input Parameters&tdata instance name, filename\\
Return Value&{\tt done}\\
\hline
\end{tabular}
\bigskip

\noindent
\begin{tabular}{l|p{5in}}
\hline
Command name &{\tt dataSegment }\\ 
Description &
 Sets the data segment information for the next 'create tdata' command 
 	\\
Fingerprints & string(string,string)\\
Input Parameters& Tdata name , filename , file offset \\
Return Value&{\tt done}\\
\hline
\end{tabular}
\bigskip

\noindent
\begin{tabular}{l|p{5in}}
\hline
Command name &{\tt tdataFileName }\\ 
Description &
 Return the actual file name for the given tdata name 
 	\\
Fingerprints & string(string)\\
Input Parameters& Tdata name \\
Return Value& File name \\
\hline
\end{tabular}
\bigskip

\noindent
\begin{tabular}{l|p{5in}}
\hline
Command name &{\tt tcheckpoint }\\ 
Description &
 Checkpoints all tdata 
 	\\
Fingerprints & string(string)\\
Input Parameters& Tdata name \\
Return Value&{\tt done}\\
\hline
\end{tabular}
\bigskip

\noindent
\begin{tabular}{l|p{5in}}
\hline
Command name &{\tt getSchema }\\ 
Description &
 Gets the schema for the given tdata 
 	\\
Fingerprints & strings(string)\\
Input Parameters& Tdata name \\
Return Value&type,sorted,highval,lowval\\
\hline
\end{tabular}
\bigskip


\section{Schema Commands}
\noindent
\begin{tabular}{l|p{5in}}
\hline
Command name &{\tt parseSchema }\\ 
Description &
Move physical and logical schemas
 
into buffers and parse a schema from buffers
 	\\
Fingerprints & string(string,string,string)\\
Input Parameters&schema name, physical schema, logical schema\\
Return Value&schema name\\
\hline
\end{tabular}
\bigskip

\noindent
\begin{tabular}{l|p{5in}}
\hline
Command name &{\tt importFileType }\\ 
Description &
 Import the schema from the specified file 
 	\\
Fingerprints & string(string)\\
Input Parameters& File name \\
Return Value& Schema name or empty \\
\hline
\end{tabular}
\bigskip

\noindent
\begin{tabular}{l|p{5in}}
\hline
Command name &{\tt catFiles }\\ 
Description &
 Get all the schemas imported 
 	\\
Fingerprints & strings()\\
Input Parameters&\\
Return Value& List of schema names currently loaded \\
\hline
\end{tabular}
\bigskip

\noindent
\begin{tabular}{l|p{5in}}
\hline
Command name &{\tt getTopGroups }\\ 
Description &
 Returns the top level groups for the given schema 
 	\\
Fingerprints & strings(string)\\
Input Parameters& Schema name \\
Return Value& List of top group names or empty \\
\hline
\end{tabular}
\bigskip

\noindent
\begin{tabular}{l|p{5in}}
\hline
Command name &{\tt getItems }\\ 
Description &
 Return items in a given subgroup 
 	\\
Fingerprints & strings(string,string,string)\\
Input Parameters& Schema name , top group name , group name \\
Return Value& List of items or empty \\
\hline
\end{tabular}
\bigskip


\section{Graph Data Commands}
\noindent
\begin{tabular}{l|p{5in}}
\hline
Command name &{\tt checkGstat }\\ 
Description &
Return 1 if GDataStat is in memory, otherwise 0
 	\\
Fingerprints & boolean(string)\\
Input Parameters&view name\\
Return Value&{\tt done}\\
\hline
\end{tabular}
\bigskip

\noindent
\begin{tabular}{l|p{5in}}
\hline
Command name &{\tt isInterpreted }\\ 
Description &
Return true iff the tdata map is interpreted
 	\\
Fingerprints & boolean(string)\\
Input Parameters&tdata map instance name\\
Return Value&0/1\\
\hline
\end{tabular}
\bigskip

\noindent
\begin{tabular}{l|p{5in}}
\hline
Command name &{\tt mapG2TAttr }\\ 
Description &
Get the TData attribute information for a gdata attributes, X/Y/Z/Color
 	\\
Fingerprints & string(string,string)\\
Input Parameters&view name, gdata attribute\\
Return Value&TData attribute name\\
\hline
\end{tabular}
\bigskip

\noindent
\begin{tabular}{l|p{5in}}
\hline
Command name &{\tt mapT2GAttr }\\ 
Description &
Get the GData attribute information for a tdata attributes
 	\\
Fingerprints & string(string,string)\\
Input Parameters&view name, TData attribute name\\
Return Value&X/Y/Z/Color\\
\hline
\end{tabular}
\bigskip

\noindent
\begin{tabular}{l|p{5in}}
\hline
Command name &{\tt createMappingClass }\\ 
Description &
 Create interpreted mapping class 
 	\\
Fingerprints & string(string)\\
Input Parameters& Mapping class name \\
Return Value& Mapping class name \\
\hline
\end{tabular}
\bigskip

\noindent
\begin{tabular}{l|p{5in}}
\hline
Command name &{\tt getMappingTData }\\ 
Description &
 Get name of tdata used in mapping 
 	\\
Fingerprints & string(string)\\
Input Parameters& Mapping name \\
Return Value& Tdata name \\
\hline
\end{tabular}
\bigskip

\noindent
\begin{tabular}{l|p{5in}}
\hline
Command name &{\tt isInterpretedGData }\\ 
Description &
 Check if the given GData is interpreted or not 
 	\\
Fingerprints & integer(string)\\
Input Parameters& Mapping name \\
Return Value& 1 if true \\
\hline
\end{tabular}
\bigskip

\noindent
\begin{tabular}{l|p{5in}}
\hline
Command name &{\tt insertMapping }\\ 
Description &
 Inserts the mapping to a view 
 	\\
Fingerprints & string(string,string,string)\\
Input Parameters& View name , mapping name , legend , \\
Return Value&{\tt done}\\
\hline
\end{tabular}
\bigskip

\noindent
\begin{tabular}{l|p{5in}}
\hline
Command name &{\tt getMappingLegend }\\ 
Description &
 Returns the legend used for the mapping 
 	\\
Fingerprints & string(string,string)\\
Input Parameters& View name , mapping name \\
Return Value& Legend \\
\hline
\end{tabular}
\bigskip

\noindent
\begin{tabular}{l|p{5in}}
\hline
Command name &{\tt setMappingLegend }\\ 
Description &
 Sets the legend used for the mapping 
 	\\
Fingerprints & string(string,string,string)\\
Input Parameters& View name , mapping name , legend \\
Return Value&{\tt done}\\
\hline
\end{tabular}
\bigskip

\noindent
\begin{tabular}{l|p{5in}}
\hline
Command name &{\tt removeMapping }\\ 
Description &
 Removes a mapping from a view 
 	\\
Fingerprints & string(string,string)\\
Input Parameters& View name , mapping name \\
Return Value&{\tt done}\\
\hline
\end{tabular}
\bigskip

\noindent
\begin{tabular}{l|p{5in}}
\hline
Command name &{\tt getPixelWidth }\\ 
Description &
 Get the current pixel width for the given mapping 
 	\\
Fingerprints & integer(string)\\
Input Parameters& Mapping name \\
Return Value& Pixel width \\
\hline
\end{tabular}
\bigskip

\noindent
\begin{tabular}{l|p{5in}}
\hline
Command name &{\tt setPixelWidth }\\ 
Description &
 Sets the pixel width for the given view 
 	\\
Fingerprints & string(string,integer)\\
Input Parameters& Mapping name , width \\
Return Value&{\tt done}\\
\hline
\end{tabular}
\bigskip


\section{View Commands}
\noindent
\begin{tabular}{l|p{5in}}
\hline
Command name &{\tt getViewDisplayDataValues }\\ 
Description &
Return 1, if data value display flag is ture; Otherwise 0. 
 	\\
Fingerprints & boolean(string)\\
Input Parameters&view name\\
Return Value&1/0\\
\hline
\end{tabular}
\bigskip

\noindent
\begin{tabular}{l|p{5in}}
\hline
Command name &{\tt setViewDisplayDataValues }\\ 
Description &
Set display data value flag to be ON or OFF
 	\\
Fingerprints & string(string,boolean)\\
Input Parameters&view name, flag\\
Return Value&{\tt done}\\
\hline
\end{tabular}
\bigskip

\noindent
\begin{tabular}{l|p{5in}}
\hline
Command name &{\tt getAllStats }\\ 
Description &
Return STAT-MAX, STAT-MEAN, STAT-MIN, 
 
STAT-MIN, STAT-COUNT for a view
 	\\
Fingerprints & strings()\\
Input Parameters&view name\\
Return Value&statistics\\
\hline
\end{tabular}
\bigskip

\noindent
\begin{tabular}{l|p{5in}}
\hline
Command name &{\tt getAllViews }\\ 
Description &
Return all view names
 	\\
Fingerprints & strings()\\
Input Parameters&\\
Return Value&all view names\\
\hline
\end{tabular}
\bigskip

\noindent
\begin{tabular}{l|p{5in}}
\hline
Command name &{\tt getFont }\\ 
Description &
Return font information for a view in the form of 4 strings. 
 
All of them are integers. They are respectively font family,
 
point size, bold style, italic style. The origin parameter specifies
 
the title of the view, or X/Y axis of the view.
 	\\
Fingerprints & strings(string,string)\\
Input Parameters&view name, origin\\
Return Value&font descriptions\\
\hline
\end{tabular}
\bigskip

\noindent
\begin{tabular}{l|p{5in}}
\hline
Command name &{\tt getHistViewname }\\ 
Description &
Return the historgram view instance 
 	\\
Fingerprints & string(string)\\
Input Parameters&view name\\
Return Value&view instance\\
\hline
\end{tabular}
\bigskip

\noindent
\begin{tabular}{l|p{5in}}
\hline
Command name &{\tt getHistogram }\\ 
Description &
Return information for histogram of a given view in the form 
 
of 3 strings: min(float), max(float), number of buckets(integer)
 	\\
Fingerprints & strings(string)\\
Input Parameters&view name\\
Return Value&histogram information\\
\hline
\end{tabular}
\bigskip

\noindent
\begin{tabular}{l|p{5in}}
\hline
Command name &{\tt getSourceName }\\ 
Description &
Return data source name assoicated with a given view
 	\\
Fingerprints & string(string)\\
Input Parameters&view name\\
Return Value&data source name\\
\hline
\end{tabular}
\bigskip

\noindent
\begin{tabular}{l|p{5in}}
\hline
Command name &{\tt getStatBuffer }\\ 
Description &
General function for returning statistics for a view. The following 
 
formats are supported. "Stat:view name","Hist:view name", 
 
"GstatX:view name", "GstatY:viwe name". They are used to request 
 
statistics for view, histogram defined on the view, Gdata 
 
statistics along X/Y.
 	\\
Fingerprints & strings(string)\\
Input Parameters&statistics request\\
Return Value&statistics information\\
\hline
\end{tabular}
\bigskip

\noindent
\begin{tabular}{l|p{5in}}
\hline
Command name &{\tt getViewGDS }\\ 
Description &
Return GData's send parameters. 6 strings will be returned:
 
draw to screen flag(0/1), send to socket flag(0/1), 
 
port number(integer), filename(string), send text flag(0/1), 
 
seperator(string)
 	\\
Fingerprints & strings(string)\\
Input Parameters&view name\\
Return Value&GData's send parameters\\
\hline
\end{tabular}
\bigskip

\noindent
\begin{tabular}{l|p{5in}}
\hline
Command name &{\tt getViewPileMode }\\ 
Description &
Return true iff the specified view is in pile mode
 	\\
Fingerprints & boolean(string)\\
Input Parameters&view name\\
Return Value&pile flag\\
\hline
\end{tabular}
\bigskip

\noindent
\begin{tabular}{l|p{5in}}
\hline
Command name &{\tt getViewXYZoom }\\ 
Description &
Return true iff XYZoom flag of the view is set
 	\\
Fingerprints & boolean(string)\\
Input Parameters& view name\\
Return Value&XYZoom flag\\
\hline
\end{tabular}
\bigskip

\noindent
\begin{tabular}{l|p{5in}}
\hline
Command name &{\tt getWinGeometry }\\ 
Description &
Return the gemoetry for a specified window. The return values 
 
are: abosolute origin X(integer), Y(integer), relative origin of
 
the window X(integer), Y(integer), width(integer), height(integer)
 	\\
Fingerprints & strings(string)\\
Input Parameters&window name\\
Return Value&geometry information\\
\hline
\end{tabular}
\bigskip

\noindent
\begin{tabular}{l|p{5in}}
\hline
Command name &{\tt isXDateType }\\ 
Description &
Return true iff X type is date type
 	\\
Fingerprints & boolean(string)\\
Input Parameters&view name\\
Return Value&date type flag\\
\hline
\end{tabular}
\bigskip

\noindent
\begin{tabular}{l|p{5in}}
\hline
Command name &{\tt isYDateType }\\ 
Description &
Return true iff Y type is date type
 	\\
Fingerprints & boolean(string)\\
Input Parameters&view name\\
Return Value&date type\\
\hline
\end{tabular}
\bigskip

\noindent
\begin{tabular}{l|p{5in}}
\hline
Command name &{\tt setFont }\\ 
Description &
Set the font for the title/X/Y of a view
 	\\
Fingerprints & string(string,string,integer,integer,integer,integer)\\
Input Parameters&view name, destination, font family, point size,isBold,isItalic\\
Return Value&{\tt done}\\
\hline
\end{tabular}
\bigskip

\noindent
\begin{tabular}{l|p{5in}}
\hline
Command name &{\tt setHistViewname }\\ 
Description &
Set the view instance name for a histogram of a view
 	\\
Fingerprints & string(string,string)\\
Input Parameters&view name,view instance\\
Return Value&{\tt done}\\
\hline
\end{tabular}
\bigskip

\noindent
\begin{tabular}{l|p{5in}}
\hline
Command name &{\tt setHistogram }\\ 
Description &
Set the histogram parameters for a view
 	\\
Fingerprints & string(float,float,integer)\\
Input Parameters&view name,min,max,buckets\\
Return Value&{\tt done}\\
\hline
\end{tabular}
\bigskip

\noindent
\begin{tabular}{l|p{5in}}
\hline
Command name &{\tt setViewGDS }\\ 
Description &
Set the GData send parameters for a view
 	\\
Fingerprints & string(string,integer,integer,integer,string,integer,string)\\
Input Parameters&view name,draw to screen flag, send to socket flag, port number,gds file name, send text flag, gds separator \\
Return Value&{\tt done}\\
\hline
\end{tabular}
\bigskip

\noindent
\begin{tabular}{l|p{5in}}
\hline
Command name &{\tt setViewPileMode }\\ 
Description &
Set the piled mode for a view
 	\\
Fingerprints & string(string,boolean)\\
Input Parameters&view name, piled flag\\
Return Value&{\tt done}\\
\hline
\end{tabular}
\bigskip

\noindent
\begin{tabular}{l|p{5in}}
\hline
Command name &{\tt setViewXYZoom }\\ 
Description &
Set the XYZoom flag for a view
 	\\
Fingerprints & string(string,boolean)\\
Input Parameters&view name, XYZoom flag\\
Return Value&{\tt done}\\
\hline
\end{tabular}
\bigskip

\noindent
\begin{tabular}{l|p{5in}}
\hline
Command name &{\tt setWinGeometry }\\ 
Description &
Set the gemoetry for a specified window. The parameters to be set 
 
include: relative origin X(integer), Y(integer), 
 
width(integer), height(integer)
 	\\
Fingerprints & string(string,integer,integer,integer,integer)\\
Input Parameters&view name, X, Y, width, height  \\
Return Value&{\tt done}\\
\hline
\end{tabular}
\bigskip

\noindent
\begin{tabular}{l|p{5in}}
\hline
Command name &{\tt viewGetAlign }\\ 
Description &
Return the alignment value of a view
 	\\
Fingerprints & integer(string)\\
Input Parameters&view name\\
Return Value&alignment value\\
\hline
\end{tabular}
\bigskip

\noindent
\begin{tabular}{l|p{5in}}
\hline
Command name &{\tt viewGetHome }\\ 
Description &
Return a view's home information: view home mode(integer),
 
X margin(float), Y margin(float), xlow(float), ylow(float),
 
xHigh(float) and yHigh(float)
 	\\
Fingerprints & strings(string)\\
Input Parameters&view name\\
Return Value&view home information\\
\hline
\end{tabular}
\bigskip

\noindent
\begin{tabular}{l|p{5in}}
\hline
Command name &{\tt viewGetHorPan }\\ 
Description &
Return a view's panel information: mode(integer), relative Pan(float),
 
absolute Pan(float)
 	\\
Fingerprints & strings(string)\\
Input Parameters&view name\\
Return Value&view's pan information\\
\hline
\end{tabular}
\bigskip

\noindent
\begin{tabular}{l|p{5in}}
\hline
Command name &{\tt viewSetAlign }\\ 
Description &
Set a view's alignment value
 	\\
Fingerprints & string(string,string)\\
Input Parameters&view name, alignment value\\
Return Value&{\tt done}\\
\hline
\end{tabular}
\bigskip

\noindent
\begin{tabular}{l|p{5in}}
\hline
Command name &{\tt viewSetHome }\\ 
Description &
Set up the view home parameters
 	\\
Fingerprints & string(string,integer,float,float,float,float,float,float)\\
Input Parameters&view name,view home mode, X margin, Y margin, xLow, yLow,\\
Return Value&{\tt done}\\
\hline
\end{tabular}
\bigskip

\noindent
\begin{tabular}{l|p{5in}}
\hline
Command name &{\tt viewSetHorPan }\\ 
Description &
Set up the pan information for a viwe
 	\\
Fingerprints & string(string,integer,float,float)\\
Input Parameters&view name,mode,relative pan, absolute pan\\
Return Value&{\tt done}\\
\hline
\end{tabular}
\bigskip

\noindent
\begin{tabular}{l|p{5in}}
\hline
Command name &{\tt winGetPrint }\\ 
Description &
Get the print flags for a specified view window. 
 
Return two flags: exclue mode, pixmap mode
 	\\
Fingerprints & strings(string)\\
Input Parameters&win name\\
Return Value&flags\\
\hline
\end{tabular}
\bigskip

\noindent
\begin{tabular}{l|p{5in}}
\hline
Command name &{\tt winSetPrint }\\ 
Description &
Change the exclude and pixmap flags for a view win
 	\\
Fingerprints & string(string,integer,integer)\\
Input Parameters&exclude flag, pixmap flag\\
Return Value&{\tt done}\\
\hline
\end{tabular}
\bigskip

\noindent
\begin{tabular}{l|p{5in}}
\hline
Command name &{\tt setLabel }\\ 
Description &
 Sets the label parameters for the given view 
 	\\
Fingerprints & string(string,,integer,string)\\
Input Parameters& View name , occupyTop , extent , label name \\
Return Value&{\tt done}\\
\hline
\end{tabular}
\bigskip

\noindent
\begin{tabular}{l|p{5in}}
\hline
Command name &{\tt setFilter }\\ 
Description &
 Sets the current filter for the view 
 	\\
Fingerprints & string(string,integer,integer,integer,integer)\\
Input Parameters& View name , xLow , yLow , xHigh , yHigh \\
Return Value&{\tt done}\\
\hline
\end{tabular}
\bigskip

\noindent
\begin{tabular}{l|p{5in}}
\hline
Command name &{\tt getViewOverrideColor }\\ 
Description &
 Returns the override color used by a view 
 	\\
Fingerprints & strings(string)\\
Input Parameters& View name \\
Return Value& override color,if active\\
\hline
\end{tabular}
\bigskip

\noindent
\begin{tabular}{l|p{5in}}
\hline
Command name &{\tt setViewOverrideColor }\\ 
Description &
 Sets the override color used by a view 
 	\\
Fingerprints & string(string,boolean,string)\\
Input Parameters& View name , active , override color \\
Return Value&{\tt done}\\
\hline
\end{tabular}
\bigskip

\noindent
\begin{tabular}{l|p{5in}}
\hline
Command name &{\tt insertWindow }\\ 
Description &
 Insert the view in to the window 
 	\\
Fingerprints & string(string,string)\\
Input Parameters& View name , window name \\
Return Value&{\tt done}\\
\hline
\end{tabular}
\bigskip

\noindent
\begin{tabular}{l|p{5in}}
\hline
Command name &{\tt swapView }\\ 
Description &
 Swaps the position of the two views within the window 
 	\\
Fingerprints & string(string,string,string)\\
Input Parameters& Window name, view1 name , view2 name \\
Return Value&{\tt done}\\
\hline
\end{tabular}
\bigskip

\noindent
\begin{tabular}{l|p{5in}}
\hline
Command name &{\tt setAxisDisplay }\\ 
Description &
 Set the axis display on/off for the given view 
 	\\
Fingerprints & string(string,string,integer)\\
Input Parameters& View name , "X"/"Y" , 0/1 \\
Return Value&{\tt done}\\
\hline
\end{tabular}
\bigskip

\noindent
\begin{tabular}{l|p{5in}}
\hline
Command name &{\tt insertviewHistory }\\ 
Description &
 Insert history in to the view without changing the filter 
 	\\
Fingerprints & string(string,float,float,float,float,integer)\\
Input Parameters& View name , xLow , yLow , xHigh , yHigh , marked \\
Return Value&{\tt done}\\
\hline
\end{tabular}
\bigskip

\noindent
\begin{tabular}{l|p{5in}}
\hline
Command name &{\tt markViewFilter }\\ 
Description &
 Mark the n'th view as marked or unmarked 
 	\\
Fingerprints & string(string,integer,integer)\\
Input Parameters& View name , index , 0/1 \\
Return Value&{\tt done}\\
\hline
\end{tabular}
\bigskip

\noindent
\begin{tabular}{l|p{5in}}
\hline
Command name &{\tt highlightView }\\ 
Description &
 Highlight/unhighlight the given view 
 	\\
Fingerprints & string(string,integer)\\
Input Parameters& View name , 0/1 \\
Return Value&{\tt done}\\
\hline
\end{tabular}
\bigskip

\noindent
\begin{tabular}{l|p{5in}}
\hline
Command name &{\tt setAxis }\\ 
Description &
 Sets the axis label for the specified axis and view 
 	\\
Fingerprints & string(string,string,string)\\
Input Parameters& View name , axis label name , axis type X/Y \\
Return Value&{\tt done}\\
\hline
\end{tabular}
\bigskip

\noindent
\begin{tabular}{l|p{5in}}
\hline
Command name &{\tt getAxis }\\ 
Description &
 Gets the axis label for the given axis and view 
 	\\
Fingerprints & string(string,string)\\
Input Parameters& View name , axis X/Y \\
Return Value& Label name \\
\hline
\end{tabular}
\bigskip

\noindent
\begin{tabular}{l|p{5in}}
\hline
Command name &{\tt setAction }\\ 
Description &
 Sets the action for a view 
 	\\
Fingerprints & string(string,string)\\
Input Parameters& View name , action name \\
Return Value&{\tt done}\\
\hline
\end{tabular}
\bigskip

\noindent
\begin{tabular}{l|p{5in}}
\hline
Command name &{\tt getAxisDisplay }\\ 
Description &
 Check if the given axis is turned on or off 
 	\\
Fingerprints & string(string,string)\\
Input Parameters& View name , axis  X/Y \\
Return Value& 1/0 \\
\hline
\end{tabular}
\bigskip

\noindent
\begin{tabular}{l|p{5in}}
\hline
Command name &{\tt setViewStatistics }\\ 
Description &
 Set the statistical display status 
 	\\
Fingerprints & string(string)\\
Input Parameters& View name \\
Return Value&{\tt done}\\
\hline
\end{tabular}
\bigskip

\noindent
\begin{tabular}{l|p{5in}}
\hline
Command name &{\tt getViewStatistics }\\ 
Description &
 Get the status of the statistic display for the given view 
 	\\
Fingerprints & void(string)\\
Input Parameters& View name \\
Return Value&void\\
\hline
\end{tabular}
\bigskip

\noindent
\begin{tabular}{l|p{5in}}
\hline
Command name &{\tt getViewDimensions }\\ 
Description &
 Returns the number of dimensions of the given view 
 	\\
Fingerprints & integer(string)\\
Input Parameters& View name \\
Return Value& Dimensions \\
\hline
\end{tabular}
\bigskip

\noindent
\begin{tabular}{l|p{5in}}
\hline
Command name &{\tt setViewDimensions }\\ 
Description &
 Sets the dimensions for the view 
 	\\
Fingerprints & string(string,integer)\\
Input Parameters& View name , dimensions \\
Return Value&{\tt done}\\
\hline
\end{tabular}
\bigskip

\noindent
\begin{tabular}{l|p{5in}}
\hline
Command name &{\tt getViewSolid3D }\\ 
Description &
 Returns flag indicating wireframe or solid 3D objects 
 	\\
Fingerprints & integer(string)\\
Input Parameters& View name \\
Return Value& 1:solid or 0:wireframe \\
\hline
\end{tabular}
\bigskip

\noindent
\begin{tabular}{l|p{5in}}
\hline
Command name &{\tt setViewSolid3D }\\ 
Description &
 Sets 3D objects to wireframe or solid display mode 
 	\\
Fingerprints & string(string,integer)\\
Input Parameters& View name , 1:solid or 0:wireframe\\
Return Value&{\tt done}\\
\hline
\end{tabular}
\bigskip

\noindent
\begin{tabular}{l|p{5in}}
\hline
Command name &{\tt savePixmap }\\ 
Description &
 Saves the view's pixmap to the given file 
 	\\
Fingerprints & string(string,long)\\
Input Parameters& View name , file pointer FILE* \\
Return Value&{\tt done}\\
\hline
\end{tabular}
\bigskip

\noindent
\begin{tabular}{l|p{5in}}
\hline
Command name &{\tt loadPixmap }\\ 
Description &
 Load the view's pixmap 
 	\\
Fingerprints & string(string,long)\\
Input Parameters& View name , pixelmap file pointer FILE* \\
Return Value&{\tt done}\\
\hline
\end{tabular}
\bigskip

\noindent
\begin{tabular}{l|p{5in}}
\hline
Command name &{\tt replaceView }\\ 
Description &
 Replaces the first view with the second 
 	\\
Fingerprints & string(string,,string)\\
Input Parameters& View name , , view name , \\
Return Value&{\tt done}\\
\hline
\end{tabular}
\bigskip

\noindent
\begin{tabular}{l|p{5in}}
\hline
Command name &{\tt getCurVisualFilter }\\ 
Description &
 Get the current visual filter for the given view 
 	\\
Fingerprints & strings(string)\\
Input Parameters& View name \\
Return Value& xHigh, yLow and yHigh with only 2 digits after the decimal point\\
\hline
\end{tabular}
\bigskip

\noindent
\begin{tabular}{l|p{5in}}
\hline
Command name &{\tt getVisualFilters }\\ 
Description &
 Get all the visual filters for a view in string format 
 	\\
Fingerprints & strings(string)\\
Input Parameters& View name \\
Return Value&yLow ,High,yHigh,marked\\
\hline
\end{tabular}
\bigskip

\noindent
\begin{tabular}{l|p{5in}}
\hline
Command name &{\tt removeView }\\ 
Description &
 Remove a view from it's window 
 	\\
Fingerprints & string(string)\\
Input Parameters& View name \\
Return Value&{\tt done}\\
\hline
\end{tabular}
\bigskip

\noindent
\begin{tabular}{l|p{5in}}
\hline
Command name &{\tt getViewMappings }\\ 
Description &
 Get all the mappings connected to the given view 
 	\\
Fingerprints & strings(string)\\
Input Parameters& View name \\
Return Value& List of mapping names \\
\hline
\end{tabular}
\bigskip

\noindent
\begin{tabular}{l|p{5in}}
\hline
Command name &{\tt refreshView }\\ 
Description &
 Refreshes the given View 
 	\\
Fingerprints & string(string)\\
Input Parameters& View name \\
Return Value&{\tt done}\\
\hline
\end{tabular}
\bigskip

\noindent
\begin{tabular}{l|p{5in}}
\hline
Command name &{\tt getAction }\\ 
Description &
 Get the Action for the view 
 	\\
Fingerprints & string(string)\\
Input Parameters& View name \\
Return Value& Action name \\
\hline
\end{tabular}
\bigskip

\noindent
\begin{tabular}{l|p{5in}}
\hline
Command name &{\tt showkgraph }\\ 
Description &
 Create a new Kiviat graph or use an existing one to display statistics of specified views 
 	\\
Fingerprints & string(string,,string,strings)\\
Input Parameters& KGraph name , statistic , , views to be included in the KGraph \\
Return Value&{\tt done}\\
\hline
\end{tabular}
\bigskip

\noindent
\begin{tabular}{l|p{5in}}
\hline
Command name &{\tt invalidatePixmap }\\ 
Description &
 Invalidate the cached pixelmap for the view 
 	\\
Fingerprints & string(string)\\
Input Parameters& View name \\
Return Value&{\tt done}\\
\hline
\end{tabular}
\bigskip

\noindent
\begin{tabular}{l|p{5in}}
\hline
Command name &{\tt isMapped }\\ 
Description &
 Returns 1 if the view is mapped, 0 otherwise 
 	\\
Fingerprints & integer(string)\\
Input Parameters& View name \\
Return Value& 1/0 \\
\hline
\end{tabular}
\bigskip

\noindent
\begin{tabular}{l|p{5in}}
\hline
Command name &{\tt getLabel }\\ 
Description &
 Get the label parameters for the given view, return "x y z" where x = occupyTop (1/0, integer),                y = extent (integer), z = title (string)
 	\\
Fingerprints & strings(string)\\
Input Parameters& View name \\
Return Value& "x y z" \\
\hline
\end{tabular}
\bigskip

\noindent
\begin{tabular}{l|p{5in}}
\hline
Command name &{\tt getViewWin }\\ 
Description &
 Get the window name for this view 
 	\\
Fingerprints & string(string)\\
Input Parameters& View name \\
Return Value& Window name \\
\hline
\end{tabular}
\bigskip

\noindent
\begin{tabular}{l|p{5in}}
\hline
Command name &{\tt clearViewHistory }\\ 
Description &
 Clear the visual filter history of the specified view 
 	\\
Fingerprints & string(string)\\
Input Parameters& View name \\
Return Value&{\tt done}\\
\hline
\end{tabular}
\bigskip

\noindent
\begin{tabular}{l|p{5in}}
\hline
Command name &{\tt raiseView }\\ 
Description &
 Raises the view to top of window stacking order 
 	\\
Fingerprints & string(string)\\
Input Parameters& View name \\
Return Value&{\tt done}\\
\hline
\end{tabular}
\bigskip

\noindent
\begin{tabular}{l|p{5in}}
\hline
Command name &{\tt lowerView }\\ 
Description &
 Lowers the view to bottom of window stacking order 
 	\\
Fingerprints & string(string)\\
Input Parameters& View name \\
Return Value&{\tt done}\\
\hline
\end{tabular}
\bigskip


\section{Window Commands}
\noindent
\begin{tabular}{l|p{5in}}
\hline
Command name &{\tt getWindowImageAndSize }\\ 
Description &
Export a GIF picture of a window to the specified port, the data file
 
is prefixed with the size of the GIF file
 	\\
Fingerprints & string(integer,string,string)\\
Input Parameters&port,imageType,widowName\\
Return Value&{\tt done}\\
\hline
\end{tabular}
\bigskip

\noindent
\begin{tabular}{l|p{5in}}
\hline
Command name &{\tt getWinCount }\\ 
Description &
Return the total number of devise windows
 	\\
Fingerprints & integer()\\
Input Parameters&\\
Return Value&total number of windows\\
\hline
\end{tabular}
\bigskip

\noindent
\begin{tabular}{l|p{5in}}
\hline
Command name &{\tt getWinViews }\\ 
Description &
 Get all the views in the given window 
 	\\
Fingerprints & strings(string)\\
Input Parameters& Window name \\
Return Value& List of view names \\
\hline
\end{tabular}
\bigskip

\noindent
\begin{tabular}{l|p{5in}}
\hline
Command name &{\tt saveWindowImage }\\ 
Description &
 Saves the window image to a file 
 	\\
Fingerprints & string(string,,string,string)\\
Input Parameters& Window name , format , , file name \\
Return Value&{\tt done}\\
\hline
\end{tabular}
\bigskip

\noindent
\begin{tabular}{l|p{5in}}
\hline
Command name &{\tt getWindowLayout }\\ 
Description &
 Returns the window's height and width, "x y z" where x = height (integer),  y = width (integer), z = 1/0 (integer) indicating stacked mode on/off 
 	\\
Fingerprints & strings(string)\\
Input Parameters& Window name \\
Return Value& "x y z"\\
\hline
\end{tabular}
\bigskip


\section{Link and Cursor Commands}
\noindent
\begin{tabular}{l|p{5in}}
\hline
Command name &{\tt getLinkType }\\ 
Description &
Return 1, if the link is Positive, otherwise 0. 
 	\\
Fingerprints & boolean(string)\\
Input Parameters&linkname\\
Return Value&1/0 or Error message\\
\hline
\end{tabular}
\bigskip

\noindent
\begin{tabular}{l|p{5in}}
\hline
Command name &{\tt getCursorGrid }\\ 
Description &
Return 3 strings, the first string is really a 0/1 indicating whether
 
grid is used, the last two strings are floats giving gridX and gridY.
 	\\
Fingerprints & strings(string)\\
Input Parameters&cursor name\\
Return Value&grid parameters\\
\hline
\end{tabular}
\bigskip

\noindent
\begin{tabular}{l|p{5in}}
\hline
Command name &{\tt setCursorGrid }\\ 
Description &
Set cursor grid
 	\\
Fingerprints & string(string,integer,float,float)\\
Input Parameters&grid cursor name,use grid flag,cursorGridX,cursorGridY\\
Return Value&{\tt done}\\
\hline
\end{tabular}
\bigskip

\noindent
\begin{tabular}{l|p{5in}}
\hline
Command name &{\tt setLinkType }\\ 
Description &
Set the record link type to be postive(0) or negative(1)
 	\\
Fingerprints & string(string,boolean)\\
Input Parameters&record link instance name,type\\
Return Value&void/Error message\\
\hline
\end{tabular}
\bigskip

\noindent
\begin{tabular}{l|p{5in}}
\hline
Command name &{\tt insertLink }\\ 
Description &
 Connects a view to the link 
 	\\
Fingerprints & string(string,string)\\
Input Parameters& Link name , view name \\
Return Value&{\tt done}\\
\hline
\end{tabular}
\bigskip

\noindent
\begin{tabular}{l|p{5in}}
\hline
Command name &{\tt setLinkFlag }\\ 
Description &
 Sets the type of the link (x/y/color/etc) 
 	\\
Fingerprints & string(string,integer)\\
Input Parameters& Link name , flag value \\
Return Value&{\tt done}\\
\hline
\end{tabular}
\bigskip

\noindent
\begin{tabular}{l|p{5in}}
\hline
Command name &{\tt getLinkFlag }\\ 
Description &
 Return the type of the link 
 	\\
Fingerprints & integer(string)\\
Input Parameters& Link name \\
Return Value& Type of the link \\
\hline
\end{tabular}
\bigskip

\noindent
\begin{tabular}{l|p{5in}}
\hline
Command name &{\tt setLinkMaster }\\ 
Description &
 Sets a view to be the master of a link 
 	\\
Fingerprints & string(string,string)\\
Input Parameters& Link name , view name \\
Return Value&{\tt done}\\
\hline
\end{tabular}
\bigskip

\noindent
\begin{tabular}{l|p{5in}}
\hline
Command name &{\tt resetLinkMaster }\\ 
Description &
 Sets a link to be without a master 
 	\\
Fingerprints & string(string)\\
Input Parameters& Link name \\
Return Value&{\tt done}\\
\hline
\end{tabular}
\bigskip

\noindent
\begin{tabular}{l|p{5in}}
\hline
Command name &{\tt getLinkMaster }\\ 
Description &
 Return name of link master 
 	\\
Fingerprints & string(string)\\
Input Parameters& Link name \\
Return Value& Name of master view \\
\hline
\end{tabular}
\bigskip

\noindent
\begin{tabular}{l|p{5in}}
\hline
Command name &{\tt getLinkViews }\\ 
Description &
 Get all the views in the given link 
 	\\
Fingerprints & strings(string)\\
Input Parameters& Link name \\
Return Value& List of view names \\
\hline
\end{tabular}
\bigskip

\noindent
\begin{tabular}{l|p{5in}}
\hline
Command name &{\tt viewInLink }\\ 
Description &
 Check if the view is in the given link 
 	\\
Fingerprints & integer(string,string)\\
Input Parameters& Link name , view name \\
Return Value& 0/1 \\
\hline
\end{tabular}
\bigskip

\noindent
\begin{tabular}{l|p{5in}}
\hline
Command name &{\tt unlinkView }\\ 
Description &
 Unlinks the view from the given link 
 	\\
Fingerprints & string(string,string)\\
Input Parameters& Link name , view name \\
Return Value&{\tt done}\\
\hline
\end{tabular}
\bigskip

\noindent
\begin{tabular}{l|p{5in}}
\hline
Command name &{\tt setCursorSrc }\\ 
Description &
 Sets the source view for the given cursor 
 	\\
Fingerprints & string(string,string)\\
Input Parameters& Cursor name , view name \\
Return Value&{\tt done}\\
\hline
\end{tabular}
\bigskip

\noindent
\begin{tabular}{l|p{5in}}
\hline
Command name &{\tt setCursorDst }\\ 
Description &
 Sets the destination view for the given cursor 
 	\\
Fingerprints & string(string,string)\\
Input Parameters& Cursor name , view name \\
Return Value&{\tt done}\\
\hline
\end{tabular}
\bigskip

\noindent
\begin{tabular}{l|p{5in}}
\hline
Command name &{\tt getCursorViews }\\ 
Description &
 Gets the source and destination views of a cursor 
 	\\
Fingerprints & strings(string)\\
Input Parameters& Cursor name \\
Return Value& source dest\\
\hline
\end{tabular}
\bigskip


\section{Miscellaneous Commands}
\noindent
\begin{tabular}{l|p{5in}}
\hline
Command name &{\tt getDisplayImageAndSize }\\ 
Description &
It generates a GIF file from
 
devise's display and outputs to the specified port. The
 
data file is prefixed with its size.
 	\\
Fingerprints & string(integer,string)\\
Input Parameters&port, imageType\\
Return Value&{\tt done}\\
\hline
\end{tabular}
\bigskip

\noindent
\begin{tabular}{l|p{5in}}
\hline
Command name &{\tt serverExit }\\ 
Description &
If connected clients is one, this command forces the server to quits
 	\\
Fingerprints & string()\\
Input Parameters&\\
Return Value&{\tt done}\\
\hline
\end{tabular}
\bigskip

\noindent
\begin{tabular}{l|p{5in}}
\hline
Command name &{\tt setScreenSize }\\ 
Description &
Set the width and height of the screen
 	\\
Fingerprints & string(integer,integer)\\
Input Parameters&width, height\\
Return Value&{\tt done}\\
\hline
\end{tabular}
\bigskip

\noindent
\begin{tabular}{l|p{5in}}
\hline
Command name &{\tt abortQuery }\\ 
Description &
Abort query on a view object
 	\\
Fingerprints & string(string)\\
Input Parameters&view name\\
Return Value&{\tt done}\\
\hline
\end{tabular}
\bigskip

\noindent
\begin{tabular}{l|p{5in}}
\hline
Command name &{\tt dumpLinkCursor }\\ 
Description &
Dump the link info into a file
 	\\
Fingerprints & string(string)\\
Input Parameters&filename\\
Return Value&{\tt done}\\
\hline
\end{tabular}
\bigskip

\noindent
\begin{tabular}{l|p{5in}}
\hline
Command name &{\tt copyright }\\ 
Description &
Return the copyright string
 	\\
Fingerprints & string()\\
Input Parameters&\\
Return Value&copyright string/none\\
\hline
\end{tabular}
\bigskip

\noindent
\begin{tabular}{l|p{5in}}
\hline
Command name &{\tt version }\\ 
Description &
Return the version string
 	\\
Fingerprints & string()\\
Input Parameters&\\
Return Value&version string/none\\
\hline
\end{tabular}
\bigskip

\noindent
\begin{tabular}{l|p{5in}}
\hline
Command name &{\tt compDate }\\ 
Description &
Return the compilation date 
 	\\
Fingerprints & string()\\
Input Parameters&\\
Return Value&compilation date/none\\
\hline
\end{tabular}
\bigskip

\noindent
\begin{tabular}{l|p{5in}}
\hline
Command name &{\tt getFileHeader }\\ 
Description &
Get the standard devise file header for a specific file type. The file
 
header contains information like file type, version, compilation date
 
etc.
 	\\
Fingerprints & string(string)\\
Input Parameters&file type\\
Return Value&file header\\
\hline
\end{tabular}
\bigskip

\noindent
\begin{tabular}{l|p{5in}}
\hline
Command name &{\tt getStringCount }\\ 
Description &
Return the number strings in string storage
 	\\
Fingerprints & double()\\
Input Parameters&\\
Return Value&string count\\
\hline
\end{tabular}
\bigskip

\noindent
\begin{tabular}{l|p{5in}}
\hline
Command name &{\tt insertViewHistory }\\ 
Description &
Insert a visual filter query into history
 	\\
Fingerprints & string(string,float,float,float,float,boolean)\\
Input Parameters&view name, xLow,yLow,xHi,yHi, filter marked flag\\
Return Value&{\tt done}\\
\hline
\end{tabular}
\bigskip

\noindent
\begin{tabular}{l|p{5in}}
\hline
Command name &{\tt loadStringSpace }\\ 
Description &
Initializing string table from the given filename, it will refresh
 
all views
 	\\
Fingerprints & string(string)\\
Input Parameters&file name\\
Return Value&{\tt done}\\
\hline
\end{tabular}
\bigskip

\noindent
\begin{tabular}{l|p{5in}}
\hline
Command name &{\tt openSession }\\ 
Description &
Open a session file
 	\\
Fingerprints & string(string)\\
Input Parameters&file name\\
Return Value&{\tt done}\\
\hline
\end{tabular}
\bigskip

\noindent
\begin{tabular}{l|p{5in}}
\hline
Command name &{\tt saveDisplayImageAndMap }\\ 
Description &
Export the display and the map file associated with the display. 
 
EPS, GIF and POSTSCRIPT are currently supported formats.
 	\\
Fingerprints & string(string,string,string,string,string)\\
Input Parameters&format,gif filename, map filename, URL, default URL\\
Return Value&{\tt done}\\
\hline
\end{tabular}
\bigskip

\noindent
\begin{tabular}{l|p{5in}}
\hline
Command name &{\tt saveDisplayView }\\ 
Description &
Export the view in EPS, GIF or POSTSCRIPT format
 	\\
Fingerprints & string(string,string)\\
Input Parameters&format,filename \\
Return Value&{\tt done}\\
\hline
\end{tabular}
\bigskip

\noindent
\begin{tabular}{l|p{5in}}
\hline
Command name &{\tt saveSession }\\ 
Description &
Save a session file as a template/export version, with/without data
 	\\
Fingerprints & string(string,integer,integer,integer)\\
Input Parameters&filename, asTemplate flag, asExport flag, withData flag\\
Return Value&{\tt done}\\
\hline
\end{tabular}
\bigskip

\noindent
\begin{tabular}{l|p{5in}}
\hline
Command name &{\tt saveStringSpace }\\ 
Description &
Save all strings in the string space into a file
 	\\
Fingerprints & string(string)\\
Input Parameters&filename\\
Return Value&{\tt done}\\
\hline
\end{tabular}
\bigskip

\noindent
\begin{tabular}{l|p{5in}}
\hline
Command name &{\tt startLayoutManager }\\ 
Description &
Fork a child process to run layout manager
 	\\
Fingerprints & string()\\
Input Parameters&\\
Return Value&void\\
\hline
\end{tabular}
\bigskip

\noindent
\begin{tabular}{l|p{5in}}
\hline
Command name &{\tt waitForQueries }\\ 
Description &
Activate the dispatcher to do SingleStepCurrent
 	\\
Fingerprints & string()\\
Input Parameters&\\
Return Value&{\tt done}\\
\hline
\end{tabular}
\bigskip

\noindent
\begin{tabular}{l|p{5in}}
\hline
Command name &{\tt writeDesc }\\ 
Description &
Write a session description file
 	\\
Fingerprints & string(string)\\
Input Parameters&Description file name\\
Return Value&{\tt done}\\
\hline
\end{tabular}
\bigskip

\noindent
\begin{tabular}{l|p{5in}}
\hline
Command name &{\tt date }\\ 
Description &
 Returns the current date 
 	\\
Fingerprints & string()\\
Input Parameters&\\
Return Value& mm/dd/yy \\
\hline
\end{tabular}
\bigskip

\noindent
\begin{tabular}{l|p{5in}}
\hline
Command name &{\tt clearAll }\\ 
Description &
 Clears all session-related data structures from memory 
 	\\
Fingerprints & string()\\
Input Parameters&\\
Return Value&{\tt done}\\
\hline
\end{tabular}
\bigskip

\noindent
\begin{tabular}{l|p{5in}}
\hline
Command name &{\tt setBatchMode }\\ 
Description &
 Sets server to batch or normal mode 
 	\\
Fingerprints & string(integer)\\
Input Parameters& 0/1 indicating normal or batch mode \\
Return Value&{\tt done}\\
\hline
\end{tabular}
\bigskip

\noindent
\begin{tabular}{l|p{5in}}
\hline
Command name &{\tt sync }\\ 
Description &
 Requests a SyncNotify message from the server when it has processed all commands up to this point 
 	\\
Fingerprints & string()\\
Input Parameters&\\
Return Value&{\tt done}\\
\hline
\end{tabular}
\bigskip

\noindent
\begin{tabular}{l|p{5in}}
\hline
Command name &{\tt parseDateFloat }\\ 
Description &
 Converts a date in string format to Unix time format 
 	\\
Fingerprints & double(string)\\
Input Parameters& Date \\
Return Value& Unix time \\
\hline
\end{tabular}
\bigskip

\noindent
\begin{tabular}{l|p{5in}}
\hline
Command name &{\tt changeMode }\\ 
Description &
 Change modes to Layout or Display 
 	\\
Fingerprints & string(integer)\\
Input Parameters& 0/1 , indicating display mode , or layout mode \\
Return Value&{\tt done}\\
\hline
\end{tabular}
\bigskip

\noindent
\begin{tabular}{l|p{5in}}
\hline
Command name &{\tt exists }\\ 
Description &
 Check to see if the given instance exists 
 	\\
Fingerprints & integer(string)\\
Input Parameters& Instance name \\
Return Value& 0/1 \\
\hline
\end{tabular}
\bigskip

\noindent
\begin{tabular}{l|p{5in}}
\hline
Command name &{\tt addReplicaServer }\\ 
Description &
 Adds a replica server where client commands are echoed 
 	\\
Fingerprints & string(string,integer)\\
Input Parameters& Host name , port number \\
Return Value&{\tt done}\\
\hline
\end{tabular}
\bigskip

\noindent
\begin{tabular}{l|p{5in}}
\hline
Command name &{\tt removeReplicaServer }\\ 
Description &
 Removes a replica server 
 	\\
Fingerprints & string(string,integer)\\
Input Parameters& Host name , port number \\
Return Value&{\tt done}\\
\hline
\end{tabular}
\bigskip

\noindent
\begin{tabular}{l|p{5in}}
\hline
Command name &{\tt get3DLocation }\\ 
Description &
 Returns the 3D position of the camera and focal point: X, Y, Z, FX, FY, FZ, Theta, Phi, Rho (floats) 
 	\\
Fingerprints & strings()\\
Input Parameters&\\
Return Value& 3D position of the camera\\
\hline
\end{tabular}
\bigskip

\noindent
\begin{tabular}{l|p{5in}}
\hline
Command name &{\tt set3DLocation }\\ 
Description &
 Sets the 3D position of the camera 
 	\\
Fingerprints & string()\\
Input Parameters& Y, Z \\
Return Value&{\tt done}\\
\hline
\end{tabular}
\bigskip


\section{Obsolete Commands}
\noindent
\begin{tabular}{l|p{5in}}
\hline
Command name &{\tt open }\\ 
Description &
 Opens a specified file with the given flag 
 	\\
Fingerprints & long(string,string)\\
Input Parameters& File name , open mode \\
Return Value& File pointer :FILE*\\
\hline
\end{tabular}
\bigskip

\noindent
\begin{tabular}{l|p{5in}}
\hline
Command name &{\tt writeLine }\\ 
Description &
 Writes a line in to the given file 
 	\\
Fingerprints & string(long,string)\\
Input Parameters& File pointer FILE* , line \\
Return Value&{\tt done}\\
\hline
\end{tabular}
\bigskip

\noindent
\begin{tabular}{l|p{5in}}
\hline
Command name &{\tt readLine }\\ 
Description &
 Reads a line from the specified file 
 	\\
Fingerprints & string(long)\\
Input Parameters& File pointer FILE* \\
Return Value& Line read from file \\
\hline
\end{tabular}
\bigskip

\noindent
\begin{tabular}{l|p{5in}}
\hline
Command name &{\tt close }\\ 
Description &
 Closes the specified file 
 	\\
Fingerprints & string(long)\\
Input Parameters& File pointer :FILE*\\
Return Value&{\tt done}\\
\hline
\end{tabular}
\bigskip


\section{Control Commands}
\noindent
\begin{tabular}{l|p{5in}}
\hline
Command name &{\tt ChangeStatus }\\ 
Description &
 The server changes to busy or idle state 
 	\\
Fingerprints & void(integer)\\
Input Parameters& 0/1 \\
Return Value&void\\
\hline
\end{tabular}
\bigskip

\noindent
\begin{tabular}{l|p{5in}}
\hline
Command name &{\tt SyncDone }\\ 
Description &
 An indication that the server has finished processing all commands preceding the last {\tt sync} command 
 	\\
Fingerprints & void()\\
Input Parameters&\\
Return Value&void\\
\hline
\end{tabular}
\bigskip

\noindent
\begin{tabular}{l|p{5in}}
\hline
Command name &{\tt AbortProgram }\\ 
Description &
 The server is aborting due to an internal error 
 	\\
Fingerprints & void(string)\\
Input Parameters& Reason \\
Return Value&void\\
\hline
\end{tabular}
\bigskip

\noindent
\begin{tabular}{l|p{5in}}
\hline
Command name &{\tt ProcessViewSelected }\\ 
Description &
 A view has been selected as ``current view'' 
 	\\
Fingerprints & void(string)\\
Input Parameters& View name \\
Return Value&void\\
\hline
\end{tabular}
\bigskip

\noindent
\begin{tabular}{l|p{5in}}
\hline
Command name &{\tt ProcessViewFilterChange }\\ 
Description &
 The visual filter of a view has changed 
 	\\
Fingerprints & void(string,boolean,string,boolean)\\
Input Parameters& View name , flushed , xlow , marked \\
Return Value&void\\
\hline
\end{tabular}
\bigskip


\section{User Commands}
\noindent
\begin{tabular}{l|p{5in}}
\hline
Command name &{\tt setVisualFilter }\\ 
Description &
Pose visual query, including rubber band, zoom in/out etc.
 	\\
Fingerprints & string(string,float,float,float,float)\\
Input Parameters&view name, xLow, yLow, xHigh, yHigh\\
Return Value&{\tt done}\\
\hline
\end{tabular}
\bigskip

\noindent
\begin{tabular}{l|p{5in}}
\hline
Command name &{\tt doPopUp }\\ 
Description &
Pose record-level query and display the result in the form of a small window
 	\\
Fingerprints & string(string,float,float,float,float,integer,integer,strings)\\
Input Parameters&view name,x,y, xHigh, yHigh,button, number of messages, messages \\
Return Value&{\tt done}\\
\hline
\end{tabular}
\bigskip

\noindent
\begin{tabular}{l|p{5in}}
\hline
Command name &{\tt moveCursor }\\ 
Description &
Move cursor to (x,y) position
 	\\
Fingerprints & string(string,float,float,integer)\\
Input Parameters&winName, x, y, mouse button\\
Return Value&{\tt done}\\
\hline
\end{tabular}
\bigskip


\section{Color Commands}
\noindent
\begin{tabular}{l|p{5in}}
\hline
Command name &{\tt GetColor }\\ 
Description &
Return rgb color string specified by a color ID
 	\\
Fingerprints & string(integer)\\
Input Parameters&color ID\\
Return Value&color string\\
\hline
\end{tabular}
\bigskip

\noindent
\begin{tabular}{l|p{5in}}
\hline
Command name &{\tt GetColorID }\\ 
Description &
Return the color ID from a rgb string
 	\\
Fingerprints & integer(string)\\
Input Parameters&color string\\
Return Value&color ID\\
\hline
\end{tabular}
\bigskip

\noindent
\begin{tabular}{l|p{5in}}
\hline
Command name &{\tt GetForeground }\\ 
Description &
Return the foreground rgb string
 	\\
Fingerprints & string(string)\\
Input Parameters& view name\\
Return Value&color string\\
\hline
\end{tabular}
\bigskip

\noindent
\begin{tabular}{l|p{5in}}
\hline
Command name &{\tt GetBackground }\\ 
Description &
Return the background rgb string
 	\\
Fingerprints & string(string)\\
Input Parameters& view name\\
Return Value&color string\\
\hline
\end{tabular}
\bigskip

\noindent
\begin{tabular}{l|p{5in}}
\hline
Command name &{\tt SetForeground }\\ 
Description &
Set the foreground color for a specific view
 	\\
Fingerprints & string(string,string)\\
Input Parameters& view name, color string\\
Return Value&{\tt done}\\
\hline
\end{tabular}
\bigskip

\noindent
\begin{tabular}{l|p{5in}}
\hline
Command name &{\tt SetBackground }\\ 
Description &
Set the background color for a specific view
 	\\
Fingerprints & string(string,string)\\
Input Parameters& view name, color string\\
Return Value&{\tt done}\\
\hline
\end{tabular}
\bigskip

\noindent
\begin{tabular}{l|p{5in}}
\hline
Command name &{\tt GetAllColors }\\ 
Description &
Return all current rgb strings
 	\\
Fingerprints & strings()\\
Input Parameters&\\
Return Value&rgb string list\\
\hline
\end{tabular}
\bigskip

\noindent
\begin{tabular}{l|p{5in}}
\hline
Command name &{\tt AutoColor }\\ 
Description &
Given background color, generate foreground color automatically
 	\\
Fingerprints & integer(integer)\\
Input Parameters&color ID\\
Return Value&color ID\\
\hline
\end{tabular}
\bigskip

\noindent
\begin{tabular}{l|p{5in}}
\hline
Command name &{\tt NewPalette }\\ 
Description &
Generate a new color palette and return palette id
 	\\
Fingerprints & integer()\\
Input Parameters&\\
Return Value&palette ID or Error message\\
\hline
\end{tabular}
\bigskip

\noindent
\begin{tabular}{l|p{5in}}
\hline
Command name &{\tt GetCurrentPalette }\\ 
Description &
Return current color palette's ID
 	\\
Fingerprints & integer()\\
Input Parameters&\\
Return Value&palette ID\\
\hline
\end{tabular}
\bigskip

\noindent
\begin{tabular}{l|p{5in}}
\hline
Command name &{\tt SetCurrentPalette }\\ 
Description &
This command always "succeeds", 
 
failure to change the palette
 
is indicated by returning the current pid after the operation.
 	\\
Fingerprints & integer(integer)\\
Input Parameters&palette ID\\
Return Value&palette ID\\
\hline
\end{tabular}
\bigskip

\noindent
\begin{tabular}{l|p{5in}}
\hline
Command name &{\tt GetPaletteColors }\\ 
Description &
Return all rgb color strings of a palette
 	\\
Fingerprints & strings(integer)\\
Input Parameters&palette ID\\
Return Value&color strings\\
\hline
\end{tabular}
\bigskip

\noindent
\begin{tabular}{l|p{5in}}
\hline
Command name &{\tt CheckColoring }\\ 
Description &
Return the RMS distance for a view
 	\\
Fingerprints & float(string)\\
Input Parameters&view name\\
Return Value&distance\\
\hline
\end{tabular}
\bigskip



\newpage
\section*{Copyright}

\verbatiminput{Copyright}

\section*{Disclaimer}

\verbatiminput{Disclaimer}

\newpage
\section*{Agreement}

\verbatiminput{Agreement}

\end{document}
