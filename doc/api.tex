%  ========================================================================
%  DEVise Data Visualization Software
%  (c) Copyright 1992-1996
%  By the DEVise Development Group
%  Madison, Wisconsin
%  All Rights Reserved.
%  ========================================================================
%
%  Under no circumstances is this software to be copied, distributed,
%  or altered in any way without prior permission from the DEVise
%  Development Group.

%%%%%%%%%%%%%%%%%%%%%%%%%%%%%%%%%%%%%%%%%%%%%%%%%%%%%%%%%%%%

%  $Id$

%  $Log$
%  Revision 1.6  1996/05/31 15:47:58  jussi
%  Added setLinkMaster, resetLinkMaster, and getLinkMaster commands.
%
%  Revision 1.5  1996/05/15 16:35:16  jussi
%  Added sync and SyncDone commands. Removed ExecuteScript command.
%
%  Revision 1.4  1996/05/13 22:23:50  jussi
%  Added description of control commands. Removed description of
%  some commands that no longer exist.
%
%  Revision 1.3  1996/04/15 15:10:53  jussi
%  Added second version of insertMapping, one that takes a mapping
%  label as a parameter. Added getMappingLegend.
%
%  Revision 1.2  1996/04/14 00:13:29  jussi
%  Removed interpMapClassInfo and clearInterp. Replaced createInterp
%  with createMappingClass, and moved createInterp to the obsolete
%  section.
%
%  Revision 1.1  1996/04/12 22:00:07  jussi
%  Initial revision.
%

%%%%%%%%%%%%%%%%%%%%%%%%%%%%%%%%%%%%%%%%%%%%%%%%%%%%%%%%%%%%

\documentstyle[fullpage,verbatim]{article}

\renewcommand{\topfraction}{1.0}
\renewcommand{\bottomfraction}{1.0}
\renewcommand{\textfraction}{0.0}
\advance\intextsep by 5pt

\def\filename#1{{\tt #1}}
\def\code#1{{\tt #1}}
\def\menu#1{{\tt #1}}
\def\term#1{#1}
\def\variable#1{{\tt #1}}

\def\scaleepspic[#1]#2#3{
\begin{figure}[htb]
\centering\leavevmode\epsfxsize=#1\epsfbox{#2}
\caption{#3}
\end{figure}
}

\def\fullepspic#1#2{
\begin{figure}[htb]
\centering\leavevmode\epsfxsize=\textwidth\epsfbox{#1}
\caption{#2}
\end{figure}
}

\begin{document}
\title{DEVise Application Programming Interface}
\author{The DEVise Development Group \\
\code{devise@cs.wisc.edu}
}
\date{\today}

\maketitle

\section{Introduction}

The DEVise application programming interface (API) allows a user
interface, application program or other entity to request
visualization services from the DEVise visualization engine. The
requestor ({\em client program}) and the visualization engine ({\em
server}) communicate using the DEVise command language which
implements the API.

This document is intended for client program developers who want to
customize the appearance of DEVise visualization sessions. The API
allows the programmer to specify such visualization characteristics as
placement of visualization windows, layout of views within windows,
and enabling and disabling of statistics displays. Also, the client
program can request information about objects such as Source Data,
Graph Data, and Schemas to build up dialogs for interacting with the
user.

\section{DEVise Command Language}

The commands supported in the DEVise API are divided into eight
categories: Generic, Source Data, Schema, Graph Data, View, Window,
Link and Cursor, and Miscellaneous. Commands in the Generic category
allow the client program to create and remove objects of the other
categories, or to query the parameters used when the objects were
created. See Section \ref{objcreate} for a more detailed description
of the object creation command.

The category-specific commands control the behavior of the
objects. The Miscellaneous category includes commands for querying and
altering the state of the query processor and buffer manager, among
others. The Obsolete category lists some obsolete commands that are
being phased out and should not be used.

A DEVise command consist of a verb (command name) followed by zero or
more input parameters. The parameters may be of type string, integer,
float, or date, but their values are always represented as ASCII
strings. The result of the command is returned as another
string. Error conditions are indicated at the client-server protocol
level, and the returned string is the error message. This document
describes the return values resulting from a successful command
execution. Commands that do not have any other return values return
the value {\tt done}.

The commands are described in the following format.

\bigskip

\noindent
\begin{tabular}{l|p{5in}}
\hline
Command name & Name of the command \\
Description  & What the command does \\
Input \#     & Number of input parameters \\
Input Types  & Input parameters \\
Output       & Return value or error message \\
\hline
\end{tabular}

\section{Object Creation in DEVise\protect\label{objcreate}}

DEVise stores information about all visualization objects in a common
data structure that has a uniform interface. DEVise uses a three-level
object hierarchy. At the top are {\em categories}: tdata, mapping,
view, window, link, or cursor. Next are {\em classes}. At the bottom
of the hierarchy are {\em instances} whose names share a global name
space.

The following table shows the meaning of ``class'' for each category.
For tdata and mapping, the class names are assigned by the client
program or the user. A view can either be sorted along the X axis
(SortedX) or not (Scatter). Only one window class (TileLayout) is
currently defined (older versions of DEVise defined WinVertical and
WinHorizontal but these classes are subsumed by TileLayout). A link
must belong to the Visual\_Link class and a cursor to the Cursor
class.

\bigskip

\noindent
\begin{tabular}{l|l|p{4in}}
Category & What class means           & Parameters\\
\hline
tdata    & Schema name                & File path, tdata name \\
mapping  & Client-program assigned    & Tdata name, mapping name, empty,
                                        X attribute, Y, Z, color, size,
                                        pattern, orientation, shape,
                                        shape attribute 0, 1, 2 \\
view     & ``SortedX'' or ``Scatter'' & View name, X low, X high, Y low,
                                        Y high, background color \\
window   & ``TileLayout''             & Window name, X position, Y position,
                                        width, height (all expressed as a
                                        fraction of screen size) \\
link     & ``Visual\_Link''           & Link name, link flag \\
cursor   & ``Cursor''                 & Cursor name, cursor type \\
\hline
\end{tabular}

\bigskip

The command used to create an object of any aforementioned category
and class is simply {\tt create}. The number of parameters varies
according to object category. The types of the parameters are shown in
the table.

The link flag is a binary value with the following values OR'ed: X
(1), Y (2), size (4), pattern (8), color (16), orientation (32), shape
(64). The bits that are one specify the attributes that are used to
link views.

The cursor type is a binary value with X (1) and Y (2) OR'ed. The bits
that are one specify whether the cursor operates in the X-direction,
the Y-direction, or both.

\section{Generic Commands}

\noindent
\begin{tabular}{l|p{5in}}
\hline
Command name & {\tt create} \\
Description  & Create a new instance of a class \\
Input \#     & Variable \\
Input Types  & Category name (string), class name (string), parameter values
               (strings) \\
Output       & Name of the class created or empty (string) \\
\hline
\end{tabular}

\bigskip

\noindent
\begin{tabular}{l|p{5in}}
\hline
Command name & {\tt destroy} \\
Description  & Destroys the given instance \\
Input \#     & 1 \\
Input Types  & Instance name (string) \\
Output       & {\tt done} \\
\hline
\end{tabular}

\bigskip

\noindent
\begin{tabular}{l|p{5in}}
\hline
Command name & {\tt setDefault} \\
Description  & Sets the default parameter values for a class \\
Input \#     & Variable \\
Input Types  & List of parameter values (strings) \\
Output       & {\tt done} \\
\hline
\end{tabular}

\bigskip

\noindent
\begin{tabular}{l|p{5in}}
\hline
Command name & {\tt getCreateParam} \\
Description  & Get the parameters of an instance \\
Input \#     & 3 \\
Input Types  & Category name (string), class name (string), instance
               name (string) \\
Output       & List of parameter values or empty (strings) \\
\hline
\end{tabular}

\bigskip

\noindent
\begin{tabular}{l|p{5in}}
\hline
Command name & {\tt changeParam} \\
Description  & Change the parameters for the given instance \\
Input \#     & Variable \\
Input Types  & Instance name (string), list of parameter values (strings) \\
Output       & {\tt done} \\
\hline
\end{tabular}

\bigskip

\noindent
\begin{tabular}{l|p{5in}}
\hline
Command name & {\tt changeableParam} \\
Description  & Check whether parameters of instance are changeable \\
Input \#     & 1 \\
Input Types  & Instance name (string) \\
Output       & 0/1 (integer) indicating whether instance parameters are
               changeable \\
\hline
\end{tabular}

\bigskip

\noindent
\begin{tabular}{l|p{5in}}
\hline
Command name & {\tt getInstParam} \\
Description  & Get the current parameters of the instance \\
Input \#     & 1 \\
Input Types  & Instance name (string) \\
Output       & List of parameter values or empty (strings) \\
\hline
\end{tabular}

\bigskip

\noindent
\begin{tabular}{l|p{5in}}
\hline
Command name & {\tt get} \\
Description  & Get the classes corresponding to the given category \\
Input \#     & 1 \\
Input Types  & Category name (string) \\
Output       & List of class names or empty (strings) \\
\hline
\end{tabular}

\bigskip

\noindent
\begin{tabular}{l|p{5in}}
\hline
Command name & {\tt get} \\
Description  & Get all the instances belonging to the class \\
Input \#     & 2 \\
Input Types  & Category name (string), class name (string) (optional) \\
Output       & List of instance names or empty (strings) \\
\hline
\end{tabular}

\section{Source Data Commands}

\noindent
\begin{tabular}{l|p{5in}}
\hline
Command name & {\tt tdataFileName} \\
Description  & Return the actual file name for the given tdata name \\
Input \#     & 1 \\
Input Types  & Tdata name (string) \\
Output       & File name (string) \\
\hline
\end{tabular}

\bigskip

\noindent
\begin{tabular}{l|p{5in}}
\hline
Command name & {\tt tcheckpoint} \\
Description  & Checkpoints all tdata \\
Input \#     & 1 \\
Input Types  & Tdata name (string) \\
Output       & {\tt done} \\
\hline
\end{tabular}

\bigskip

\noindent
\begin{tabular}{l|p{5in}}
\hline
Command name & {\tt getSchema} \\
Description  & Gets the schema for the given tdata \\
Input \#     & 1 \\
Input Types  & Tdata name (string) \\
Output       & \{\{attrname type sorted highval lowval\} \ldots \}
               where type is float, double, string, int, or date \\
\hline
\end{tabular}

\section{Schema Commands}

\noindent
\begin{tabular}{l|p{5in}}
\hline
Command name & {\tt importFileType} \\
Description  & Import the schema from the specified file \\
Input \#     & 1 \\
Input Types  & File name (string) \\
Output       & Schema name or empty (string) \\
\hline
\end{tabular}

\bigskip

\noindent
\begin{tabular}{l|p{5in}}
\hline
Command name & {\tt catFiles} \\
Description  & Get all the schemas imported \\
Input \#     & 0 \\
Input Types  & None \\
Output       & List of schema names currently loaded (strings) \\
\hline
\end{tabular}

\bigskip

\noindent
\begin{tabular}{l|p{5in}}
\hline
Command name & {\tt getTopGroups} \\
Description  & Returns the top level groups for the given schema \\
Input \#     & 1 \\
Input Types  & Schema name (string) \\
Output       & List of top group names or empty (strings) \\
\hline
\end{tabular}

\bigskip

\noindent
\begin{tabular}{l|p{5in}}
\hline
Command name & {\tt getItems} \\
Description  & Return items in a given subgroup \\
Input \#     & 3 \\
Input Types  & Schema name (string), top group name (string), group name
              (string) \\
Output       & List of items or empty (strings) \\
\hline
\end{tabular}

\section{Graph Data Commands}

\noindent
\begin{tabular}{l|p{5in}}
\hline
Command name & {\tt createMappingClass} \\
Description  & Create interpreted mapping class \\
Input \#     & 1 \\
Input Types  & Mapping class name (string) \\
Output       & Mapping class name (string) \\
\hline
\end{tabular}

\bigskip

\noindent
\begin{tabular}{l|p{5in}}
\hline
Command name & {\tt getMappingTData} \\
Description  & Get name of tdata used in mapping \\
Input \#     & 1 \\
Input Types  & Mapping name (string) \\
Output       & Tdata name (string) \\
\hline
\end{tabular}

\bigskip

\noindent
\begin{tabular}{l|p{5in}}
\hline
Command name & {\tt isInterpretedGData} \\
Description  & Check if the given GData is interpreted or not \\
Input \#     & 1 \\
Input Types  & Mapping name (string) \\
Output       & 0 if false, 1 if true (integer) \\
\hline
\end{tabular}

\bigskip

\noindent
\begin{tabular}{l|p{5in}}
\hline
Command name & {\tt insertMapping} \\
Description  & Inserts the mapping to a view \\
Input \#     & 2/3 \\
Input Types  & View name (string), mapping name (string), legend (string)
               (optional) \\
Output       & {\tt done} \\
\hline
\end{tabular}

\bigskip

\noindent
\begin{tabular}{l|p{5in}}
\hline
Command name & {\tt getMappingLegend} \\
Description  & Returns the legend used for the mapping \\
Input \#     & 2 \\
Input Types  & View name (string), mapping name (string) \\
Output       & Legend (string) \\
\hline
\end{tabular}

\bigskip

\noindent
\begin{tabular}{l|p{5in}}
\hline
Command name & {\tt setMappingLegend} \\
Description  & Sets the legend used for the mapping \\
Input \#     & 3 \\
Input Types  & View name (string), mapping name (string), legend (string) \\
Output       & {\tt done} \\
\hline
\end{tabular}

\bigskip

\noindent
\begin{tabular}{l|p{5in}}
\hline
Command name & {\tt removeMapping} \\
Description  & Removes a mapping from a view \\
Input \#     & 2 \\
Input Types  & View name (string), mapping name (string) \\
Output       & {\tt done} \\
\hline
\end{tabular}

\bigskip

\noindent
\begin{tabular}{l|p{5in}}
\hline
Command name & {\tt getPixelWidth} \\
Description  & Get the current pixel width for the given mapping \\
Input \#     & 1 \\
Input Types  & Mapping name (string) \\
Output       & Pixel width (integer) \\
\hline
\end{tabular}

\bigskip

\noindent
\begin{tabular}{l|p{5in}}
\hline
Command name & {\tt setPixelWidth} \\
Description  & Sets the pixel width for the given view \\
Input \#     & 2 \\
Input Types  & Mapping name (string), width (integer) \\
Output       & {\tt done} \\
\hline
\end{tabular}

\section{View Commands}

\noindent
\begin{tabular}{l|p{5in}}
\hline
Command name & {\tt setLabel} \\
Description  & Sets the label parameters for the given view \\
Input \#     & 4 \\
Input Types  & View name (string), occupyTop (integer 1/0),
               extent (integer), label name (string) \\
Output       & {\tt done} \\
\hline
\end{tabular}

\bigskip

\noindent
\begin{tabular}{l|p{5in}}
\hline
Command name & {\tt setFilter} \\
Description  & Sets the current filter for the view \\
Input \#     & 5 \\
Input Types  & View name (string), xLow (integer), yLow (integer),
               xHigh (integer), yHigh (integer) \\
Output       & {\tt done} \\
\hline
\end{tabular}

\bigskip

\noindent
\begin{tabular}{l|p{5in}}
\hline
Command name & {\tt getViewOverrideColor} \\
Description  & Returns the override color used by a view \\
Input \#     & 1 \\
Input Types  & View name (string) \\
Output       & "x y" where x = 0 if override not active and 1 if active,
               y = override color (if active) \\
\hline
\end{tabular}

\bigskip

\noindent
\begin{tabular}{l|p{5in}}
\hline
Command name & {\tt setViewOverrideColor} \\
Description  & Sets the override color used by a view \\
Input \#     & 3 \\
Input Types  & View name (string), active (0 if false, 1 if true),
               override color (string) \\
Output       & {\tt done} \\
\hline
\end{tabular}

\bigskip

\noindent
\begin{tabular}{l|p{5in}}
\hline
Command name & {\tt insertWindow} \\
Description  & Insert the view in to the window \\
Input \#     & 2 \\
Input Types  & View name (string), window name (string) \\
Output       & {\tt done} \\
\hline
\end{tabular}

\bigskip

\noindent
\begin{tabular}{l|p{5in}}
\hline
Command name & {\tt swapView} \\
Description  & Swaps the position of the two views within the window \\
Input \#     & 3 \\
Input Types  & Window name(string), view1 name (string), view2 name (string) \\
Output       & {\tt done} \\
\hline
\end{tabular}

\bigskip

\noindent
\begin{tabular}{l|p{5in}}
\hline
Command name & {\tt setAxisDisplay} \\
Description  & Set the axis display on/off for the given view \\
Input \#     & 3 \\
Input Types  & View name (string), "X"/"Y" (string) indicating axis,
               0/1 (integer) indicating off/on \\
Output       & {\tt done} \\
\hline
\end{tabular}

\bigskip

\noindent
\begin{tabular}{l|p{5in}}
\hline
Command name & {\tt insertviewHistory} \\
Description  & Insert history in to the view without changing the filter \\
Input \#     & 6 \\
Input Types  & View name (string), xLow (float), yLow (float), xHigh (float),
               yHigh (float), marked (integer) \\
Output       & {\tt done} \\
\hline
\end{tabular}

\bigskip

\noindent
\begin{tabular}{l|p{5in}}
\hline
Command name & {\tt markViewFilter} \\
Description  & Mark the n'th view as marked or unmarked \\
Input \#     & 3 \\
Input Types  & View name (string), index (integer),
               0/1 (integer) indicating unmark/mark \\
Output       & {\tt done} \\
\hline
\end{tabular}

\bigskip

\noindent
\begin{tabular}{l|p{5in}}
\hline
Command name & {\tt highlightView} \\
Description  & Highlight/unhighlight the given view \\
Input \#     & 2 \\
Input Types  & View name (string), 0/1 (integer) indicating highlight
               or unhighlight \\
Output       & {\tt done} \\
\hline
\end{tabular}

\bigskip

\noindent
\begin{tabular}{l|p{5in}}
\hline
Command name & {\tt setAxis} \\
Description  & Sets the axis label for the specified axis and view \\
Input \#     & 3 \\
Input Types  & View name (string), axis label name (string),
               axis type X/Y (string) \\
Output       & {\tt done} \\
\hline
\end{tabular}

\bigskip

\noindent
\begin{tabular}{l|p{5in}}
\hline
Command name & {\tt getAxis} \\
Description  & Gets the axis label for the given axis and view \\
Input \#     & 2 \\
Input Types  & View name (string), axis X/Y (string) \\
Output       & Label name (string) \\
\hline
\end{tabular}

\bigskip

\noindent
\begin{tabular}{l|p{5in}}
\hline
Command name & {\tt setAction} \\
Description  & Sets the action for a view \\
Input \#     & 2 \\
Input Types  & View name (string), action name (string) \\
Output       & {\tt done} \\
\hline
\end{tabular}

\bigskip

\noindent
\begin{tabular}{l|p{5in}}
\hline
Command name & {\tt getAxisDisplay} \\
Description  & Check if the given axis is turned on or off \\
Input \#     & 2 \\
Input Types  & View name (string), axis  X/Y (string) \\
Output       & 1/0 (string) indicating axis is turned on/off \\
\hline
\end{tabular}

\bigskip

\noindent
\begin{tabular}{l|p{5in}}
\hline
Command name & {\tt setViewStatistics} \\
Description  & Set the statistical display status \\
Input \#     & 2 \\
Input Types  & View name (string), statistic status string (string
               of 7 chars), each array element being '1' indicates
               that the mean, max, min, count, confidence values 80\%,
               85\% and 90\%, respectively \\
Output       & {\tt done} \\
\hline
\end{tabular}

\bigskip

\noindent
\begin{tabular}{l|p{5in}}
\hline
Command name & {\tt getViewStatistics} \\
Description  & Get the status of the statistic display for the given view \\
Input \#     & 1 \\
Input Types  & View name (string) \\
Output       & A string of 7 chars where 0/1 represents whether the
               statistic is to be displayed or not. The positions stand for
               mean, max, min, count, confidence values 85\%, 90\%, and 95\%,
               respectively \\
\hline
\end{tabular}

\bigskip

\noindent
\begin{tabular}{l|p{5in}}
\hline
Command name & {\tt getViewDimensions} \\
Description  & Returns the number of dimensions of the given view \\
Input \#     & 1 \\
Input Types  & View name (string) \\
Output       & Dimensions (integer) \\
\hline
\end{tabular}

\bigskip

\noindent
\begin{tabular}{l|p{5in}}
\hline
Command name & {\tt setViewDimensions} \\
Description  & Sets the dimensions for the view \\
Input \#     & 2 \\
Input Types  & View name (string), dimensions (integer) \\
Output       & {\tt done} \\
\hline
\end{tabular}

\bigskip

\noindent
\begin{tabular}{l|p{5in}}
\hline
Command name & {\tt savePixmap} \\
Description  & Saves the view's pixmap to the given file \\
Input \#     & 2 \\
Input Types  & View name (string), file pointer (FILE *) (long) \\
Output       & {\tt done} \\
\hline
\end{tabular}

\bigskip

\noindent
\begin{tabular}{l|p{5in}}
\hline
Command name & {\tt loadPixmap} \\
Description  & Load the view's pixmap \\
Input \#     & 2 \\
Input Types  & View name (string), pixelmap file pointer (FILE *) (long) \\
Output       & {\tt done} \\
\hline
\end{tabular}

\bigskip

\noindent
\begin{tabular}{l|p{5in}}
\hline
Command name & {\tt replaceView} \\
Description  & Replaces the first view with the second \\
Input \#     & 2 \\
Input Types  & View name (string) (original), view name (string) (new one) \\
Output       & {\tt done} \\
\hline
\end{tabular}

\bigskip

\noindent
\begin{tabular}{l|p{5in}}
\hline
Command name & {\tt getCurVisualFilter} \\
Description  & Get the current visual filter for the given view \\
Input \#     & 1 \\
Input Types  & View name (string) \\
Output       & 4 floats indicating the filter's xLow, xHigh, yLow and yHigh
               with only 2 digits after the decimal point \\
\hline
\end{tabular}

\bigskip

\noindent
\begin{tabular}{l|p{5in}}
\hline
Command name & {\tt getVisualFilters} \\
Description  & Get all the visual filters for a view in string format \\
Input \#     & 1 \\
Input Types  & View name (string) \\
Output       & \{\{xLow yLow xHigh yHigh marked\} \ldots \};
               marked is set to one if the user has marked the filter as
               an interesting one. Note: date values are printed in the
               Datestring format. \\
\hline
\end{tabular}

\bigskip

\noindent
\begin{tabular}{l|p{5in}}
\hline
Command name & {\tt removeView} \\
Description  & Remove a view from it's window \\
Input \#     & 1 \\
Input Types  & View name (string) \\
Output       & {\tt done} \\
\hline
\end{tabular}

\bigskip

\noindent
\begin{tabular}{l|p{5in}}
\hline
Command name & {\tt getViewMappings} \\
Description  & Get all the mappings connected to the given view \\
Input \#     & 1 \\
Input Types  & View name (string) \\
Output       & List of mapping names (strings) \\
\hline
\end{tabular}

\bigskip

\noindent
\begin{tabular}{l|p{5in}}
\hline
Command name & {\tt refreshView} \\
Description  & Refreshes the given View \\
Input \#     & 1 \\
Input Types  & View name (string) \\
Output       & {\tt done} \\
\hline
\end{tabular}

\bigskip

\noindent
\begin{tabular}{l|p{5in}}
\hline
Command name & {\tt getAction} \\
Description  & Get the Action for the view \\
Input \#     & 1 \\
Input Types  & View name (string) \\
Output       & Action name (string), or "" if no action specified \\
\hline
\end{tabular}

\bigskip

\noindent
\begin{tabular}{l|p{5in}}
\hline
Command name & {\tt showkgraph} \\
Description  & Create a new Kiviat graph or use an existing one to display
               statistics of specified views \\
Input \#     & variable (minimum 3) \\
Input Types  & KGraph name (string), statistic (min/max/avg/count) (string),
               name of KGraph window; remaining parameters list the names of
               views to be included in the KGraph (strings) \\
Output       & {\tt done} \\
\hline
\end{tabular}

\bigskip

\noindent
\begin{tabular}{l|p{5in}}
\hline
Command name & {\tt invalidatePixmap} \\
Description  & Invalidate the cached pixelmap for the view \\
Input \#     & 1 \\
Input Types  & View name (string) \\
Output       & {\tt done} \\
\hline
\end{tabular}

\bigskip

\noindent
\begin{tabular}{l|p{5in}}
\hline
Command name & {\tt isMapped} \\
Description  & Returns 1 if the view is mapped, 0 otherwise \\
Input \#     & 1 \\
Input Types  & View name (string) \\
Output       & 1/0 (integer) indicating whether view is mapped or not \\
\hline
\end{tabular}

\bigskip

\noindent
\begin{tabular}{l|p{5in}}
\hline
Command name & {\tt getLabel} \\
Description  & Get the label parameters for the given view \\
Input \#     & 1 \\
Input Types  & View name (string) \\
Output       & "x y z" where x = occupyTop (1/0, integer),
               y = extent (integer), z = title (string) \\
\hline
\end{tabular}

\bigskip

\noindent
\begin{tabular}{l|p{5in}}
\hline
Command name & {\tt getViewWin} \\
Description  & Get the window name for this view \\
Input \#     & 1 \\
Input Types  & View name (string) \\
Output       & Window name (string) if existing, else "" \\
\hline
\end{tabular}

\bigskip

\noindent
\begin{tabular}{l|p{5in}}
\hline
Command name & {\tt clearViewHistory} \\
Description  & Clear the visual filter history of the specified view \\
Input \#     & 1 \\
Input Types  & View name (string) \\
Output       & {\tt done} \\
\hline
\end{tabular}

\bigskip

\noindent
\begin{tabular}{l|p{5in}}
\hline
Command name & {\tt raiseView} \\
Description  & Raises the view to top of window stacking order \\
Input \#     & 1 \\
Input Types  & View name (string) \\
Output       & {\tt done} \\
\hline
\end{tabular}

\bigskip

\noindent
\begin{tabular}{l|p{5in}}
\hline
Command name & {\tt lowerView} \\
Description  & Lowers the view to bottom of window stacking order \\
Input \#     & 1 \\
Input Types  & View name (string) \\
Output       & {\tt done} \\
\hline
\end{tabular}

\section{Window Commands}

\noindent
\begin{tabular}{l|p{5in}}
\hline
Command name & {\tt getWinViews} \\
Description  & Get all the views in the given window \\
Input \#     & 1 \\
Input Types  & Window name (string) \\
Output       & List of view names (strings) \\
\hline
\end{tabular}

\bigskip

\noindent
\begin{tabular}{l|p{5in}}
\hline
Command name & {\tt saveWindowImage} \\
Description  & Saves the window image to a file \\
Input \#     & 3 \\
Input Types  & Window name (string), format (eps, gif, postscript) (string),
               file name (string) \\
Output       & {\tt done} \\
\hline
\end{tabular}

\bigskip

\noindent
\begin{tabular}{l|p{5in}}
\hline
Command name & {\tt getWindowLayout} \\
Description  & Returns the window's height and width \\
Input \#     & 1 \\
Input Types  & Window name (string) \\
Output       & "x y z" where x = height (integer), y = width (integer),
               z = 1/0 (integer) indicating stacked mode on/off \\
\hline
\end{tabular}

\bigskip

\noindent
\begin{tabular}{l|p{5in}}
\hline
Command name & {\tt setWindowLayout} \\
Description  & Sets the preferred window layout (height, width); optionally
               set window to stacked mode or reset normal mode \\
Input \#     & 3/4 \\
Input Types  & Window name (string), height (integer), width (integer),
               stacked (1/0) (integer, optional) \\
Output       & {\tt done} \\
\hline
\end{tabular}

\section{Link and Cursor Commands}

\noindent
\begin{tabular}{l|p{5in}}
\hline
Command name & {\tt insertLink} \\
Description  & Connects a view to the link \\
Input \#     & 2 \\
Input Types  & Link name (string), view name (string) \\
Output       & {\tt done} \\
\hline
\end{tabular}

\bigskip

\noindent
\begin{tabular}{l|p{5in}}
\hline
Command name & {\tt setLinkFlag} \\
Description  & Sets the type of the link (x/y/color/etc) \\
Input \#     & 2 \\
Input Types  & Link name (string), flag value (integer) \\
Output       & {\tt done} \\
\hline
\end{tabular}

\bigskip

\noindent
\begin{tabular}{l|p{5in}}
\hline
Command name & {\tt getLinkFlag} \\
Description  & Return the type of the link \\
Input \#     & 1 \\
Input Types  & Link name (string) \\
Output       & Type of the link (integer) \\
\hline
\end{tabular}

\bigskip

\noindent
\begin{tabular}{l|p{5in}}
\hline
Command name & {\tt setLinkMaster} \\
Description  & Sets a view to be the master of a link \\
Input \#     & 2 \\
Input Types  & Link name (string), view name (string) \\
Output       & {\tt done} \\
\hline
\end{tabular}

\bigskip

\noindent
\begin{tabular}{l|p{5in}}
\hline
Command name & {\tt resetLinkMaster} \\
Description  & Sets a link to be without a master \\
Input \#     & 1 \\
Input Types  & Link name (string) \\
Output       & {\tt done} \\
\hline
\end{tabular}

\bigskip

\noindent
\begin{tabular}{l|p{5in}}
\hline
Command name & {\tt getLinkMaster} \\
Description  & Return name of link master \\
Input \#     & 1 \\
Input Types  & Link name (string) \\
Output       & Name of master view (string) or empty string \\
\hline
\end{tabular}

\bigskip

\noindent
\begin{tabular}{l|p{5in}}
\hline
Command name & {\tt getLinkViews} \\
Description  & Get all the views in the given link \\
Input \#     & 1 \\
Input Types  & Link name (string) \\
Output       & List of view names (strings) \\
\hline
\end{tabular}

\bigskip

\noindent
\begin{tabular}{l|p{5in}}
\hline
Command name & {\tt viewInLink} \\
Description  & Check if the view is in the given link \\
Input \#     & 2 \\
Input Types  & Link name (string), view name (string) \\
Output       & 0/1 (integer) indicating not present/present \\
\hline
\end{tabular}

\bigskip

\noindent
\begin{tabular}{l|p{5in}}
\hline
Command name & {\tt unlinkView} \\
Description  & Unlinks the view from the given link \\
Input \#     & 2 \\
Input Types  & Link name (string), view name (string) \\
Output       & {\tt done} \\
\hline
\end{tabular}

\bigskip

\noindent
\begin{tabular}{l|p{5in}}
\hline
Command name & {\tt setCursorSrc} \\
Description  & Sets the source view for the given cursor \\
Input \#     & 2 \\
Input Types  & Cursor name (string), view name (string) \\
Output       & {\tt done} \\
\hline
\end{tabular}

\bigskip

\noindent
\begin{tabular}{l|p{5in}}
\hline
Command name & {\tt setCursorDst} \\
Description  & Sets the destination view for the given cursor \\
Input \#     & 2 \\
Input Types  & Cursor name (string), view name (string) \\
Output       & {\tt done} \\
\hline
\end{tabular}

\bigskip

\noindent
\begin{tabular}{l|p{5in}}
\hline
Command name & {\tt getCursorViews} \\
Description  & Gets the source and destination views of a cursor \\
Input \#     & 1 \\
Input Types  & Cursor name (string) \\
Output       & \{source dest\} (strings) \\
\hline
\end{tabular}

\section{Miscellaneous Commands}

\noindent
\begin{tabular}{l|p{5in}}
\hline
Command name & {\tt date} \\
Description  & Returns the current date \\
Input \#     & 0 \\
Input Types  & None \\
Output       & mm/dd/yy (string) \\
\hline
\end{tabular}

\bigskip

\noindent
\begin{tabular}{l|p{5in}}
\hline
Command name & {\tt clearAll} \\
Description  & Clears all session-related data structures from memory \\
Input \#     & 0 \\
Input Types  & None \\
Output       & {\tt done} \\
\hline
\end{tabular}

\bigskip

\noindent
\begin{tabular}{l|p{5in}}
\hline
Command name & {\tt setBatchMode} \\
Description  & Sets server to batch or normal mode \\
Input \#     & 1 \\
Input Types  & 0/1 indicating normal or batch mode (integer) \\
Output       & {\tt done} \\
\hline
\end{tabular}

\bigskip

\noindent
\begin{tabular}{l|p{5in}}
\hline
Command name & {\tt sync} \\
Description  & Requests a SyncNotify message from the server when it has
               processed all commands up to this point \\
Input \#     & 0 \\
Input Types  & None \\
Output       & {\tt done} \\
\hline
\end{tabular}

\bigskip

\noindent
\begin{tabular}{l|p{5in}}
\hline
Command name & {\tt parseDateFloat} \\
Description  & Converts a date in string format to Unix time format \\
Input \#     & 1 \\
Input Types  & Date (string) \\
Output       & Unix time (double, really an integer) \\
\hline
\end{tabular}

\bigskip

\noindent
\begin{tabular}{l|p{5in}}
\hline
Command name & {\tt changeMode} \\
Description  & Change modes to Layout or Display \\
Input \#     & 1 \\
Input Types  & 0/1 (integer) indicating display mode (0) or layout mode (1) \\
Output       & {\tt done} \\
\hline
\end{tabular}

\bigskip

\noindent
\begin{tabular}{l|p{5in}}
\hline
Command name & {\tt exists} \\
Description  & Check to see if the given instance exists \\
Input \#     & 1 \\
Input Types  & Instance name (string) \\
Output       & 0/1 (integer) indicating whether instance exists or not \\
\hline
\end{tabular}

\bigskip

\noindent
\begin{tabular}{l|p{5in}}
\hline
Command name & {\tt addReplicaServer} \\
Description  & Adds a replica server where client commands are echoed \\
Input \#     & 2 \\
Input Types  & Host name (string), port number (integer) \\
Output       & {\tt done} \\
\hline
\end{tabular}

\bigskip

\noindent
\begin{tabular}{l|p{5in}}
\hline
Command name & {\tt removeReplicaServer} \\
Description  & Removes a replica server \\
Input \#     & 2 \\
Input Types  & Host name (string), port number (integer) \\
Output       & {\tt done} \\
\hline
\end{tabular}

\bigskip

\noindent
\begin{tabular}{l|p{5in}}
\hline
Command name & {\tt get3DLocation} \\
Description  & Returns the position of the 3D camera \\
Input \#     & 0 \\
Input Types  & None \\
Output       & X, Y, Z (floats) \\
\hline
\end{tabular}

\section{Obsolete Commands}

Commands listed in this section are obsolete and not supported. Some
of the functions they provide should be implemented in the client
program internally and should not be visible to the visualization
server.

\bigskip

\noindent
\begin{tabular}{l|p{5in}}
\hline
Command name & {\tt open} \\
Description  & Opens a specified file with the given flag \\
Input \#     & 2 \\
Input Types  & File name (string), open mode (string) \\
Output       & File pointer (FILE *) (long) \\
Errors       & Cannot open file (console) \\
\hline
\end{tabular}

\bigskip

\noindent
\begin{tabular}{l|p{5in}}
\hline
Command name & {\tt writeLine} \\
Description  & Writes a line in to the given file \\
Input \#     & 2 \\
Input Types  & File pointer (FILE *) (long), line (string) \\
Output       & {\tt done} \\
\hline
\end{tabular}

\bigskip

\noindent
\begin{tabular}{l|p{5in}}
\hline
Command name & {\tt readLine} \\
Description  & Reads a line from the specified file \\
Input \#     & 1 \\
Input Types  & File pointer (FILE *) (long) \\
Output       & Line read from file (newline removed) \\
\hline
\end{tabular}

\bigskip

\noindent
\begin{tabular}{l|p{5in}}
\hline
Command name & {\tt close} \\
Description  & Closes the specified file \\
Input \#     & 1 \\
Input Types  & File pointer (FILE *) (long) \\
Output       & {\tt done} \\
\hline
\end{tabular}

\section{Control Commands}

Some server events are echoed back to the client as {\em control
commands}. These commands let the server inform the client about a
status change in the server.  For examply, when a user changes the
size of a data window and the server starts redisplaying the data, the
client receives a control command that instructs it of this event. The
following list describes the control commands issued by the
server. Note that the client does not necessarily have to take any
action in response to these commands. That is why there is also no
return value or error code sent in response to these commands.

\bigskip

\noindent
\begin{tabular}{l|p{5in}}
\hline
Command name & {\tt ChangeStatus} \\
Description  & The server changes to busy or idle state \\
Input \#     & 1 \\
Input Types  & 0/1 (integer) indicating idle or busy status \\
\hline
\end{tabular}

\bigskip

\noindent
\begin{tabular}{l|p{5in}}
\hline
Command name & {\tt SyncDone} \\
Description  & An indication that the server has finished processing all
               commands preceding the last {\tt sync} command \\
Input \#     & 0 \\
Input Types  & None \\
\hline
\end{tabular}

\bigskip

\noindent
\begin{tabular}{l|p{5in}}
\hline
Command name & {\tt AbortProgram} \\
Description  & The server is aborting due to an internal error \\
Input \#     & 1 \\
Input Types  & Reason (string) \\
\hline
\end{tabular}

\bigskip

\noindent
\begin{tabular}{l|p{5in}}
\hline
Command name & {\tt ProcessViewSelected} \\
Description  & A view has been selected as ``current view'' \\
Input \#     & 1 \\
Input Types  & View name (string) \\
\hline
\end{tabular}

\bigskip

\noindent
\begin{tabular}{l|p{5in}}
\hline
Command name & {\tt ProcessViewFilterChange} \\
Description  & The visual filter of a view has changed \\
Input \#     & 7 \\
Input Types  & View name (string), flushed (0/1 integer), xlow (string),
               ylow (string), xhigh (string), yhigh (string),
               marked (0/1 integer) \\
\hline
\end{tabular}

\newpage 
\section*{Copyright}

\verbatiminput{Copyright}

\section*{Disclaimer}

\verbatiminput{Disclaimer}

\newpage
\section*{Agreement}

\verbatiminput{Agreement}

\end{document}
