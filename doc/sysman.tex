% ========================================================================
% DEVise Data Visualization Software
% (c) Copyright 1992-1997
% By the DEVise Development Group
% Madison, Wisconsin
% All Rights Reserved.
% ========================================================================
%
% Under no circumstances is this software to be copied, distributed,
% or altered in any way without prior permission from the DEVise
% Development Group.

%###########################################################

% $Id$

% $Log$

%###########################################################

\documentstyle[epsf,fullpage,verbatim]{article}

\def\Devise{{\tt DEVise} }
\def\filename#1{{\tt #1}}
\def\code#1{{\tt #1}}
\def\term#1{#1}
\def\variable#1{{\tt #1}}

\begin{document}
\title{\Devise Developer's Manual}
\author{The \Devise Development Group \\
\code{devise@cs.wisc.edu}
}
\date{\today}

\maketitle

\section{Introduction}

\section{Schema Definition and Browsing}

\subsection{Organization of Schemas - Physical and Logical}

The schema directory consists of two subdirectories ``physical'' and
``logical''. The ``physical'' directory stores the physical schemas
defined in the systems.  The physical schema consists of the actual
list of all the attributes and their types. The high and low values of
the attributes can also be optionally specified.  The syntax can be
defined as:

\code{[sorted] attr <name> <type> [hi <val>] [lo <val>]}

where \code{[]} denotes optional and \code{<>} denotes necessary parts
of the definition.

The ``logical'' schema directory contains all the logical schemas
defined on the current set of physical schemas in the system.  Each
logical schema is associated with a physical schema and there may be
more than one logical schema associated with the same physical
schema. The logical schema contains the grouping information of the
attributes.  The attributes of the schema can be arranged in a
hierarchical structure of groups i.e. there are a set of groups
containing either attributes or subgroups of attributes. The logical
schema represents one ``view'' or a way of thinking about the schema
which is independent of the actual order of attributes in the data
file.

The association between the physical and logical schemas is captured
in the special file called \filename{catalog.tcl} in the schema
directory.  Every physical schema is associated with the list of
logical schemas defined on it.

\subsection{User Interface for Schema Browsing}

The file \filename{\$LIBDIR/schema.tk} contains the procedures to
handle schema browsing in both user and superuser mode (see user
manual for details).  The superuser is presented with the list of all
physical schemas and can modify them or define new ones. When a new
physical schema is defined, a default logical schema is automatically
generated. This default logical schema will have no group structures
defined.  Further, an entry is added to catalog.tcl to maintain the
correspondence.

\subsection{Schema Parser and Internals of Schema Storage}

The schema parser is in
\filename{graphics.new/ParseCat.[ch]}. Currently, it can handle
schemas defined both in the old and the new formats. In the old
format, there is only one schema file and there is no distinction
between physical and logical schema. The function
\code{ParseCatOriginal} handles this case. In the new format, the
logical and physical schemas are separate and are parsed by
\code{ParseCatLogical} and \code{ParseCatPhysical}.

The physical schema is parsed and an attribute list is built up. The
list consists of \code{AttrInfo} structs for each attribute which
stores the type, high and low values and a few other things (see
\filename{graphics/AttrList.h}).

The logical schema is parsed and a group directory is built up. This
consists of a list of all the groups (top level and sub groups) that
have been defined (see \filename{GroupDir.[ch]},
\filename{Group.[ch]}, \filename{ItemList.[ch]}). Each group class has
links to all the members of the groups (base attrs or subgroups).

\section{Mapping of Data Sources of Different Schemas}

\subsection{Introduction}

For a given schema, if there are a large number of data sources on a
single media (like thousands of companies on the \term{CompuStat}
tape), we build an index at the beginning to speed up the access. A
complete scan is done over the data and the index is built.  This
index file consists of the offsets for each of the data sources. In
addition, we also extract a few, chosen attributes that act as
identifiers for this data source.  All this information forms the
index file. (see \filename{CompuStat.idx}, \filename{crsp.idx}, etc.).

Given two schemas \code{s1} and \code{s2} and their index files
\filename{s1.idx} and \filename{s2.idx}, we need to build a mapping
between the two i.e.\ we need a third file \filename{s1\_s2.table}
that maps the data sources in \code{s1} to those of \code{s2}.  In
general, this can be a many-to-many mapping.

We use \term{CORAL} to perform the mappings which are essentially
joins of large relations.

\subsection{Relation Loader Programs}

The index file contains the records comprising a single relation.
There are ``loader'' programs that convert the index file entries into
a coral program that loads the relation. Note that this program
depends on the structure of the particular index file and so, we need
separate loader programs for different schemas. However, these are
extremely simple Tcl scripts.  See \filename{compcreate.tcl} and
\filename{crspcreate.tcl} for the loaders for \term{CompuStat} and
\term{CRSP}.

\subsection{The Mapping Program}

The \term{CORAL} mapping program can be actually written by the user
and executed separately. However, we also provide a easy interface for
defining the mappings in a fashion that does not assume any knowledge
of \term{CORAL}.  The file \filename{mapping.tk} contains the user
interface procedures.  This displays the list of attributes in the
indexes of the two chosen schemas and lets the user specify the pairs
that need to joined together. Once the user has specified all these
pairs, the \term{CORAL} program is automatically generated and
executed.  It creates the mapping table. However, the rules defined by
the user may not succeed in mapping all the entries in both the
schemas. If there are any unresolved entries, these are dumped into
files whose names are given by the user. Thus at the end of the
execution of the coral program, we have three files: the file
containing the mapping table and the two files containing the
unresolved entries. The user can choose to, at any time, specify
manual mappings as explained in the next section.

\subsection{Manual Definition of Mappings}

The user can manually specify the data sources to be linked together
in the form of a mapping. The file \filename{resolve.tk} contains the
user interface procedure. It presents the user with a list of the
unresolved entries and new mappings can be defined. If these are
saved, they beccome a part of the mapping table.

\subsection{Usage of the Mapping Table}

The mapping table is used in following data sources from one schema to
the other. The list of all the schemas which have been mapped is
maintained in \filename{mapdef.tcl}. This maintains the name of the
file containing the mapping table for each pair of schemas.

In the window displaying the list of selected data sources, there is a
menu button that can be used to {\em follow} the data source to a
different schema. This action looks through the appropriate mapping
table and finds the corresponding data source in the other
schema. Then this data source is loaded in the usual fashion.

\section{Statistics and Kiviat Graphs}

\subsection{Introduction}

\Devise has the capability of computing and displaying
statistics. These statistics are computed only on the data displayed
in the current views in the window. For example, mean is the average
\variable{Y} value of the data being displayed.

Currently, max, min and mean are computed.  However, the current
framework allows easy addition of new statistics, as needed.
Confidence intervals are also computed. These statistics are displayed
on demand by the user.  Click on View-Statistics to see the various
options.

\subsection{Statistics - Internals}

See \filename{ViewStats.[Ch]} and \filename{BasicStats.[Ch]}.  There
is a \code{qBasicStats} object associated with every view. Whenever a
visual query is executed on the view, all the \variable{<X,Y>} values
are passed through the statistics class too.  Any computation can be
done at this time of sampling. Finally when all the \variable{<X,Y>}
values are read, the statistics are actually computed (based on the
counters maintained like sum and sum of squares). These are displayed
only on demand.

The statistics are displayed using the bit-XOR option of
X-Windows. This allows for the toggling of statistics display without
re-reading the underlying data . Mean, max and min are displayed as
lines and can be erased by simply drawing them again. Confidence
intervals are displayed as ``transparent'' bands and can also be
similarly erased.

Note that the confidence intervals are being computed only for 85\%,
90\% and 95\% levels. The batch size is assumed to be 1 and the
corresponding \variable{Z} values have been hard-coded in.

The set of statistics to be displayed is conveyed to the display
routine through a binary string of length = number of stats. If a
particular position is 1, it means that the statistic is to be
displayed.  See \filename{ViewGraph.[Ch]} to see the manipulation of
this binary string to achieve the right response to removing and
adding stats from the display.

\subsection{Kiviat Graphs - Internals}

Kiviat graphs display the collective information about a set of views.
Currently, we can define kiviat graphs on all the views in a single
window.  The Kiviat graph can be made to display any of the statistics
(mean, max, min or count).  Further, the same graph can be resued or
new ones formed. See \filename{KGraph.[Ch]}.

The Kiviat graph keeps track of the changes in the views that comprise
it.  This is achieved by registering a callback with each of the
\code{BasicStat} object for the views. Whenever there is a change in
any of the \code{BasicStat} object (linked to a particular view), it
calls a routine in each of the \filename{KGraphs} built on it.  This
routine will refresh the graphs with the new values.

\newpage
\section*{Copyright}

\verbatiminput{Copyright}

\section*{Disclaimer}

\verbatiminput{Disclaimer}

\newpage
\section*{Agreement}

\verbatiminput{Agreement}

\end{document}
