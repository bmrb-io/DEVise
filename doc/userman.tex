\documentstyle[epsf,fullpage,verbatim]{article}

\def\Devise{{\tt DEVise} }
\def\filename#1{{\tt #1} }
\def\code#1{{\tt #1} }
\def\term#1{#1 }
\def\variable#1{{\tt #1} }

\begin{document}
\title{\Devise User Manual}
\author{The \Devise Development Group \\
\code{devise@cs.wisc.edu}
}
\date{\today}

\maketitle

\section{Introduction}

\section{Schema Definition and Browsing}

\subsection{User and Superuser Modes}

\Devise can be brought up in one of two modes - the user mode or the
superuser mode.  This is decided by the value of the variable
\variable{UserMode} defined in the \filename{.rc} file read by
\Devise.  A value of 1 indicates that it is run in the user mode and 0
indicates the superuser mode.

The mode determines the capabilities and presentation of the schema
browser.  In the user mode, the browser can be user to inspect the
attributes in any of the already defined schemas. Further, the views
defined on these schemas (through groupings) can be seen and
modified. New views can also be defined.

In the superuser mode, the browser provides complete control over all
the schemas. Existing schemas can be opened to display the types of
the attributes and their high and low values (if defined). The schemas
can also be modified, new attributes inserted, range values modified,
etc. Entirely new schemas can also be defined.  This is an operation
that would be performed by the superuser when a new data source (with
a new data layout) is to be visualized through \Devise.

\subsection{Schema Browsing in User Mode}

When the schema browser is invoked in the user mode, a list of all the
schemas defined in the system is presented. The user can see the list
of attributes in any of these. Further, the set of views defined on
these can be seen as a list.  The user can either choose an existing
view and open it up or specify the name of a brand new view to be
created. In either case, a group browser comes up and the user can
modify/define the groups and its contents in terms of base attributes
or subgroups. Finally, if a new view is saved, it becomes a part of
the list of views on that particular schema.

\subsection{Schema Browsing in Superuser Mode}

When the schema browser is invoked in the superuser mode, the list of
all the schemas currently defined in the system is displayed as a
list. Any of them can be chosen to be opened or a brand new schema can
be specified to be created. In any case, the opened schema is
displayed in terms of the list of all its attributes. The types of
these attributes can be one of those shown. Also, the high and low
values that these attrs can take can be specified.  This will help in
the automatic selection of scales during display of data. When a new
schema is saved, it becomes a part of the list of schemas and a
default view is also defined.  This view will consist of all the
attributes of the schema and will have no group structures. This can
be modified or new views defined at any time.

\section{Mapping: Association of Data Sources in Different Schemas}

\subsection{Defining Mappings}

The user can define mappings between data sources of different
schemas.  For example, associations can be established between
companies in \term{CompuStat} and \term{CRSP} databases. The user
interface to define the mappings is extremely simple and
straightforward. First, the two schemas to be mapped are specified and
also the names of files used for storing the following:

\begin{itemize}
\item Mapping table - this will be the list of associations
\item Unresolved data sources of the first schema 
\item Unresolved data sources of the second schema
\end{itemize}

It is necessary for the user to specify the file names since they have
to be used later to manually resolve the conflicts and unresolved
sources. The list of key attributes of the two schemas are displayed
and the user can pick pairs that are to be matched against each other.
The system will then automatically generate all the mappings based on
the rules specified by the user.  Any sources that are not resolved
are dumped into separate files which the user can then manually
browse.

\subsection{Resolving Conflicts and Unresolved Sources}

After the automatic generation of the mapping table, in addition to
the file containing the mappings, there are two files containing the
unresolved sources of the two schemas. By specifying these three file
names, the user can bring up the list of all the unresolved sources
and manually define the mappings based on his/her knowledge of the
sources.

\subsection{Using Mappings}

The mappings generated either automatically or manually are used to
switch between equivalent data sources in different schemas
easily. While browsing through a list of data sources in a particular
schema, say \variable{X}, the equivalent data source in a different
schema, say \variable{Y} can be loaded with a mouse click if the
mapping between \variable{X} and \variable{Y} has been defined.

\section{Statistics and Kiviat Graphs}

\subsection{Statistics}

The statistics of the displayed data can be visualized. Specifically,
the mean (average), minimum and maximum can be seen superimposed on
the data.  There are options to view the statistics for either the
current view or all the views.  Further, confidence intervals can also
be displayed.  The 85\%, 90\% and 95\% confidence intervals can be
visualized as transparent color bands around the mean line.

\subsection{Kiviat Graphs}

A Kiviat graph is a powerful method for displaying the collective
information of the data displayed in several views. For example, a
Kiviat graph used in the context of performance data can be used to
see how well the system is balanced. The Kiviat graph for the mean,
maximum, minimum or the count of the data displayed can be created. If
the underlying data changes (due to zooming or axes change), the
Kiviat graph keeps track and displays information about the current
data (We can say that a link exists between the data views and the
Kiviat graph built on them). The actual values along the various axes
of the Kiviat graph can be brought up by clicking the middle mouse
button on the graph. There are options to reuse the same Kiviat graph
to display different information or create new graphs for each.

\newpage
\section{Copyright}

\verbatiminput{Copyright}

\section{Disclaimer}

\verbatiminput{Disclaimer}

\newpage
\section{Agreement}

\verbatiminput{Agreement}

\end{document}
