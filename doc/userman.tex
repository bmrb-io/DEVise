\documentstyle[epsf,fullpage,verbatim]{article}

\def\Devise{{\tt DEVise} }
\def\filename#1{{\tt #1}}
\def\code#1{{\tt #1}}
\def\menu#1{{\tt #1}}
\def\term#1{#1}
\def\variable#1{{\tt #1}}

\def\scaleepspic[#1]#2#3{
\begin{figure}[htb]
\centering\leavevmode\epsfxsize=#1\epsfbox{#2}
\caption{#3}
\end{figure}
}

\def\fullepspic#1#2{
\begin{figure}[htb]
\centering\leavevmode\epsfxsize=\textwidth\epsfbox{#1}
\caption{#2}
\end{figure}
}

\begin{document}
\title{\Devise User Manual}
\author{The \Devise Development Group \\
\code{devise@cs.wisc.edu}
}
\date{\today}

\maketitle

\section{Introduction}

\section{Installing \Devise}

\subsection{Extracting Files from the \code{tar} Archive}

\subsection{Customization}

\section{Running \Devise}

\subsection{Path Names and Environment Variables}

\subsection{Running with the \code{devise} Script}

\subsection{Customization with \filename{devise.rc}}

\section{Tutorial Session with \Devise}

{\em The tutorial session is intended to give you a feel for
\Devise. If you already know how to use it fairly well, you can skip
to the next section.}

\scaleepspic[0.6\textwidth]{screen1.eps}{Control Panel\label{screen1}}
\scaleepspic[0.6\textwidth]{screen2.ps}{File select Window\label{screen2}}

\subsection{Open a session}
	
Now that you have successfully installed the system, the next thing
to do is to get a feel for \Devise and what it can do for you.

Go to the \filename{run} directory and type \filename{devise} to start
up \Devise. If all goes well and your installation was right, you will
get a window named {\em DEVise}. This is the {\em Control Panel}, the
nerve center of \Devise. Now click on Session (you can also type Alt-S
if you like the keyboard better). This would bring up the session
window. For the moment, a session in \Devise is nothing but a
collection of windows and mappings. Now click on Open. This brings up
a window named {\em Select file} with a lot of \filename{tk} files in
it. Now open the \filename{demo.tk} file. You can do this by double
clicking on the \filename{demo.tk} name or clicking once and pressing
OK. {\em Almost everywhere in case of selection, you can double click
on the option or filename etc., or click once on them and then click
on the OK button.}

\subsection{Visualization}

\scaleepspic[\textwidth]{screen3.eps}{Data Display Windows\label{screen3}}

The screen would now be filled with 3 windows named {\em Global view},
{\em Trade Price} and {\em Trade Volume}. The various graphs represent
the stock trade price and volume data for different companies. The
various lines are in fact the actual field values. Each graph is
called a {\em view}. Any view can be made {\em current} by clicking
within its boundary. The current view is highlighted by either a
thick, vertical line to the left of the view, or by a thin rectangle
surrounding the view. Whatever action you do is done on the current
view (though as you will see later, there can be side effects on other
views).

You can play around with these windows as any other normal X window
i.e. you can resize the windows, move them around, close them etc.

What you see in the views is a partial snapshot of the trade prices
and trade volumes of various companies. Now lets get down to exploring
the views. Click on the IBM price view in the Trade Price window. Make
sure the number lock is on and start using the numeric keypad now.

You can use the numeric keypad to scroll, zoom in and zoom out. The
numbers 4 and 6 scroll the view left and right, respectively. Now as
you do the scrolling you can find two interesting things
happening. One is that the other views (except the global view) also
scroll. And two, a colored block moves in the global view.

\subsection{Links and Cursors}

All the views (except the global view) are linked together on their
X-axis. What this means is that the X axis ranges (high and low) are
the same for all the views linked. Thus you can see the snapshot of
all the companies for the same X value (here it is date) at the same
time. In this way you can also link the views (either all of them or
some of them) along the Y axis or on a color, shape, etc. Thus this
linking essentially helps to you to view the required graphs in the
same range on any attribute.

Next let us go to exploring the colored band in the global view. This
is called the {\em cursor}. The width of the colored band corresponds
to the X range visible in all other views. The global view happens to
display all records in the IBM file (hence the name global view) and
the cursor marks the part that you are seeing in the other views. The
IBM price view and the global view are linked by this cursor. When you
zoom in on the X axis (using 7) in the IBM price view, the cursor
becomes smaller since the view is now a smaller portion of the overall
data. You can also move the cursor from the global view window, by
clicking under the X axis (on the global view window) . As you can see
the cursor is repositioned to that point and all the views also
change. Thus interesting patterns on the global view can be analyzed
by moving the cursor to that point and zooming in on the
view. Similarly, as you zoom out on the trade price or trade volume
views the cursor is enlarged.

\subsection{Y Axis Actions}

Keys 8 and 2 are for scrolling up and down on the Y axis. As you may
have guessed keys 1 and 3 are for zooming in and zooming out on the Y
axis, but just a moment. Zooming out on the Y axis would change the Y
range to be from some lower, negative value value to some higher,
positive value. The bar chart that we are displaying contains no
information on the negative side. So to zoom in on the Y axis only on
the positive side, use the combination of scrolling down/up and
zooming up/down. For instance, to zoom in Y use 1 {\em followed by} 8
and to zoom out use 2 {\em followed by} 3.

Now that you know how to play around with the views, let us go to some
other interesting stuff, like hiding a unwanted view, reviving and
also analyzing it.

\subsection{Controlling Views}

Now shift your attention to the control panel. We will concentrate on
the \menu{View} menu. The rest will be dealt in detail in the next
sections. Now click on \menu{View} (move the global view window, if it
obscures it). Now you can see several options displayed. We will
concentrate on some of them.

\subsubsection{Moving Around Views}

First let us look at the swap position. This helps to swap the
position of the last two views clicked. The two views to be swapped
must be in the same window. Click on the views you need to swap in
sequence. Now click on the \menu{Swap Position} command in the
\menu{View} menu. The two views exchange places.

You can throw away any view that you do not like from a window, by
using the \menu{Remove from window} command. As an example click on
the IBM price view in the trade Price window. Now use the \menu{Remove
from window} command to junk this view. The view disappears from the
window but not from \Devise's internal world.

To bring back any view, use the \menu{Bring back to window}
command. This one gives you a list of all views, including those that
are displayed and those that have been removed. Click on the Close Vs
Date4 entry and then click on OK (or double click on Close vs Date4).
Now another box pops up to ask you where to put the view i.e. in which
window. Select Trade Price to restore the IBM price view back to its
place.

Now you can also push the views in to various windows using the
\menu{Move to window} command. Before using it, select the view that
you want to move (by clicking on it) and then select the command. It
displays a list of window names. Choose the name of the window to move
the view into.

Off to the toggle menus. Click on the \menu{Toggle X Axis} menu
command to switch on or switch off the X axis values. Currently the X
axis values have been disabled, except in the global view. The
sub-menu \menu{All Views/Current View} lets you toggle all the views
or the view selected. Similarly the Y axis values can be toggled. If
you carefully observe, when the X axis is toggled off in the global
view, there is no way of moving the cursor directly in that window,
since to move the cursor we need to click under the X axis.

\subsubsection{Colors and Titles}

The \menu{Title} command allows you to change the title of the current
view. You can also delete the title of a view. If you do so then for
that view the current view indicator (the rectangle that was
surrounding a view) becomes a bar on the left.

The background option allows you to change the background color to
your favorite one. This feature is not yet supported. Now lets move on
to the statistics part.

\subsubsection{Statistics}

All along you might have wondered about the presence of a line across
each view. This is the statistical mean line for the view and for that
particular X range. This can be set or reset using the Statistics
menu. This menu yields a {\em tear-off} sub-menu. By clicking on the
dashed lines at the top of the sub-menu you can make it an independent
window and use the options directly from that window.

So now click on the dashed line and move the resulting window to a
desired position on the screen. You will find that the Mean option has
a red block before it meaning that it has been currently chosen. Click
on Mean, Max or Min to toggle them. Now press Apply to apply your
changes. To remove the statistical lines, simply check Mean, Max or
Min off and press apply. You can also set the confidence intervals to
be 85, 90, 95 percentages or none. The \menu{All Views/Current View}
option makes the selections affect all the views or only the current
view.

\subsubsection{Kiviat Graph}

\scaleepspic[0.4\textwidth]{screen5.eps}{Kiviat Graphs\label{screen5}}
	
You can also use the Kiviat graph tool to analyze the views. The
KGraph sub-menu is also a tear-off one. As in the case of the
statistical sub-menu you can select the Mean, Max, Min or Count (but
only one of them for each Kiviat graph). You can ask \Devise to give
you a new Kiviat graph or reuse the last one. This graph is also
computed only for the particular X range. So scrolling the views would
also change the graph displayed. Also you can use the middle button of
the mouse to know the value of the record.

\subsection{Undoing}

\scaleepspic[0.7\textwidth]{screen4.eps}{The history screen\label{screen4}}

If you inadvertently zoom in or out or scroll and want to go back,
there is a history system maintained by \Devise. The \menu{back one}
command in the control panel sets the X and Y values to the previous
values. Then press \menu{use} to apply the values. Also clicking on
\menu{history} gives you a list of recently used X and Y axis
values. You can click on any one of them and then press
\menu{use}. You can also directly edit the axis values. The
\menu{undo-edit} then comes in handy to undo any wrong values entered.

\subsection{Mouse control}

If you like to use the mouse only, for the scrolling and zooming etc.,
you can do that too! \Devise's control panel has the arrow symbols in
its left corner. The keys function like the numeric keys.

\subsection{One Last Trick}

You might be curious about the button labeled display mode in the
control panel. This is useful when you are resizing windows. If a
large data set is being visualized, then while resizing the windows,
you may want \Devise not to redisplay data every time but instead wait
until you have finalized the window sizes, positions, etc. So clicking
on this toggles it to Layout mode in which none of the views are
computed nor displayed. Then go ahead resize the windows and toggle
the button back to Display mode.

\subsection{Concluding Remarks}

To finish the session use the \menu{Close} command under the
\menu{Session} menu. To exit from the program use the \menu{Quit}
command.
	
Now that you have a feel for \Devise you can go ahead and continue
with the rest of the chapters that deal in-depth with many other
features and commands. And so ends our tutorial section. Hope it was
fun.

\section{\Devise Concepts}

This section gives an in-depth view of the concepts behind this
tool. In particular, the various Data Streams, schema, Graphical data
and view properties.

\subsection{Data Stream}

A {\em data stream} is a sequence of records that are to be visualized
using a given schema. For example, the daily stock prices for a
company can be a data stream. Data streams may come from different
sources. For example, it could be a UNIX file or the output stream of
a database query. Some data streams come from magnetic tapes.

Each data stream is associated with a data source such as the ones
mentioned above. Also associated with each stream is a schema that
describes the layout and the data fields of the stream, and a key to
uniquely identify the stream from the other streams. In the case of a
database stream, a query can be given as the input and the results of
the query are computed and cached as a data stream. The definition of
data streams is edited in the Data Stream window.

Following is a description of the various types of data streams.

\subsubsection{Unix File}

This data stream is just a normal UNIX file. The data fields are
separated by a standard delimiter specified in the schema. A sample
set of records (or tuples) is shown below.

An ASCII file containing $sin(t)$ and $cos(t)$ might look like this:

\begin{tabbing}
XXXXXXXXXx \= XXXXXXXXXX \= XXXXXXXXXX \kill
\#  time \> sin \> cos \\
0.000000.2 \> 1.000000.2 \> 0.000000.2 \\
0.017453.2 \> 0.999848.2 \> 0.017452.2 \\
0.034906.2 \> 0.999391.2 \> 0.034898.2 \\
0.052358.2 \> 0.998630.2 \> 0.052334.2 \\
... 
\end{tabbing}

The output of any program that generates similar sequences of data can
visualized in \Devise as a UNIXFILE stream. The schema defines the
delimiter (here a blank space), comment character (here \#) as
mentioned above and also assigns the meaning (name) to the various
columns. The file is selected by pressing the select button next to
the key field.

\subsubsection{WWW Data}

This data stream may come from any WWW source. The URL (Uniform
Resource Locator) could either be one using the \code{ftp} or
\code{http} protocol. The data is fetched from the Web and cached
locally. This method offers the advantage of directly using the WWW
data without additional effort. As in the case of the UNIXFILE type,
the stream delimiter and comment characters must be specified in the
schema. The key is assigned by the user and must be a unique name
within this stream type.

\subsubsection{SEQ Query}

A sequence query can be specified and executed in a SEQ database
server.  The result of the query is cached on to the disk using the
key value given. The key name needs to be distinct for the SEQ data
source type.

\subsubsection{CRSP Data}

CRSP (Center for Research in Security Prices) provides comprehensive
stock price data via two primary files, the NYSE/AMEX file and the
Nasdaq file. The CRSP data is available on a tape. The information
stored consists of company name, distribution, share, delisting and
Nasdaq information. The NYSE/AMEX Daily file contains data beginning
July 2 1962 and tracks roughly 7,250 securities. The Nasdaq file
contains information for over 12,500 common stocks stored on the
Nasdaq stock market, since Dec 14, 1972.

\subsubsection{COMPUSTAT Data}

COMPUSTAT is a database of financial, statistical, and market
information. It provides more than 300 annual and 100 quarterly Income
Statements, Balance Sheets, Statement of Cash Flows, and supplemental
data items on more than 7,500 publicly help companies. For most
companies, annual and quarterly data is available for a maximum of 20
years and 48 quarters. The data records consist of about 350 fields.

\subsubsection{ISSM Data}

The ISSM (Institute for the Study of Security Markets) is another
institute that provides transaction data for NYSE and AMEX listed
securities and NASD securities. The data for NYSE and AMEX are
available from 1983 to the present, whereas for NASD it is available
from 1990 to the present. The data record consists of trades and
quotes timed to the tick. Ancillary information such as CUSIP numbers,
firm names, shares outstanding and dividend information are also
provided. One of the most heavily traded firm (IBM) for example had
over 800,000 trades and quotes in 1992. This is also available in tape
form.

\subsubsection{SQL Query}

This is similar to the SEQ query except that the SQL query is passed
to a SQL database system from which the results are cached. The key
again must be a unique name within the SQL domain.

\subsubsection{Command Output}

\Devise executes the (Unix) command specified in the Info/Command
entry, and caches the result of the program execution. A key must be
assigned to identify this data stream from other Command Output data
streams.

\subsubsection{Network Data}

\subsection{Physical Schema}

A schema file describes the layout of the input data stream. It is
used to convey the name, type, and range of attributes, characters
that separate the attributes in the file, and characters that should
be ignored while reading the file. The attribute range information is
optional. For our example file, the schema file looks like: \\

 type Sensor ascii \\
 comment \# \\
 whitespace ' ' \\
 attr time double hi 1000 lo 0 \\
 attr sin double hi 1 lo -1 \\
 attr cos double hi 1 lo -1 \\

The first line names the file type: Sensor. All sensor files have data
stored in the same format. More than one file type can be imported
into \Devise, each having its own schema file. For example, we can
also create a Stock file type to read information about stock prices,
with data stored in a different format. The second line tells \Devise
to ignore lines that start with '\#'. The third line tells \Devise to
treat spaces as whitespace in the input stream and as a separator
between field values. The remaining lines describe attribute names,
types, and ranges.

The physical schema describes the actual layout of the input stream
with all its attributes. The physical schema may be browsed using the
\menu{Schema} command in the control panel. Every data stream is
associated with a physical schema, and it is using this schema, that
\Devise parses the file to extract the various fields.

\subsubsection{Attribute Name}

The attribute name identifies a particular field of a record. For
example in the schema described above, sin is the name of the first
attribute. This name is used when defining the attributes to be used
for the X and Y axes in defining views.

The attribute name can be changed by using the schema editor.

\subsubsection{Attribute Type}

The attribute may be one of the following five types: double, int,
string, or float, having their natural meanings. The type can also be
date which is the Unix time data type. In our example, sin is of type
double.

\subsubsection{Attribute Value Range}

The attributes may be defined to have a maximum and a minimum
range. This is useful for \Devise to choose the range values when
displaying the field values. For example, if the range value of an
attribute were to go from 0 to 200, then when displaying the attribute
on one of the axes, \Devise would automatically choose a range of 0 to
200. If no range is defined, by default, \Devise chooses a range of 0
to 100 for that attribute.

\subsubsection{Sorted vs.\ Unsorted}

The records in the stream may be sorted on a field, say YEAR. This can
also be specified in the physical schema. \Devise uses this
information to optimize data access. Data is also displayed in an
interleaved fashion instead of displaying the data in a continuous
fashion left to right (which may take quite some time for dense
data). This helps in seeing an outline of the data quicker.

\subsection{Logical Schema}

The logical schema defines a logical view on the physical schema
defined. It is useful to provide a restricted range of viewing of the
attributes in the schema. For example in the COMPUSTAT database there
are 350 attributes. For a person interested only in the Current
assets, income tax, price close etc. he/she can create a logical group
involving these attributes and can then use that to define the
views. In this way the user need not search around for the usual
attributes that are needed for visualization. Each logical schema may
consist of a number of groups. For example under a Sales logical
schema, there may be many groups one for say the sales taxes, the
other for the sales revenue etc. The logical schema may also be
browsed using the Logical option under Schema.

The group may be nested. The top-level groups are those that are on
the top of the hierarchy of the group attributes. The groups may be
nested to any arbitrary level and may consist of any number of
attributes.

For any physical schema, by default a logical schema with the same
name as that of the physical schema is created with all the attributes
in the physical schema. It is basically one group consisting of all
the attributes.

%%% This is how far I got

\subsubsection{Name of Physical Schema}

Each logical schema is a view definition of one physical schema (only
one physical schema) The physical schema name in the logical schema
associates the logical schema with the particular physical schema. As
said already a default logical schema is created for every physical
schema that has the same name as the physical schema with all the
attributes. Any creation or modification to a logical schema needs to
proceed first by the selection of the appropriate physical schema, so
the name of the physical schema is implicitly built in to the logical
schema definition.

\subsubsection{Group Name}

Each logical schema may consist of a number of groups. The group name
identifies the particular group within a logical schema. When
visualizing the data stream, the user is automatically prompted to
choose the proper group, for defining the mappings. The group name
muse be unique within the logical schema and to a certain extent must
be comprehensible to the user. This group is called the top level
group. We can define any number of groups within this group in a
nested fashion.

\subsubsection{Attribute Name}

The attribute name within a group definition includes that field in
that group. The attribute names may appear within any number of
sub-groups within a logical group. The attribute name must be the same
as defined in the physical schema, as it is the name that associates
the attribute in the logical group to the actual physical one.

\subsection{Graphical Data}

The graphical data or GData as it is called, is the second
intermediate stage in the processing of the data records before they
are displayed. As described in the earlier section, the raw data is
first parsed using the schema and TData is created. Now on this TData,
the user can select the required attributes to be displayed, their
color, shape etc. This mapping is used to form another intermediate
representation called GData or Graphical Data. This GData is the one
used to actually show the records on the screen.

Thus the GData is the graphical representation of TData. It consists
of attributes: x, y, color, pattern, size, orientation, shape and
shape specific attributes.

The mapping can be specified using the visualization window. This is
automatically generated when a data is to be imported. The window
basically picturizes the various attributes in the logical schema of
the relation. The user can choose the attributes to be displayed as
views and also the color, layout mode, window name and title. The plot
can be done using a bar chart, image plot or scatter plot. A bar
chart draws the bar lines from the Y = 0 line to the filed value. A
scatter plot simply plots the point on the graph.

The view thus defined can then be modified using the Edit mapping
choice under the View main option.

The mapping can be edited to change any one or all of the following.
X axis field, Y axis field, color, orientation, size, pattern, shape
and shape attributes.

For all of these the values may either be defined as a constant or as
a variable taking its value from one of the attributes itself, or as
the value of the expression of attributes.

\subsubsection{Location: X, Y, and Z}

The Location parameters namely X,Y and Z options help to change the
attribute begin currently displayed on the respective axis of the
current view. A constant value on any one of these axis displays the
values of the other attributes for this particular value of this
attribute. For example if the Y axis is YEAR and we want to see the
value of the X axis for records from year 1995 only, then we can
specify the Y axis value as a constant of 1995.The variable in this
case is the attribute to be displayed. The expression can also be
used.

%% This is still a question...

\subsubsection{Size}

The values of records can be displayed as objects. The size attribute
is used to change the size of the object displayed in the view. The
constant specifies the factor of enlargement of the object
displayed. The variable value is interesting to use in the case when
we need the object displayed for a record to vary according to the
value of some attribute of the record. For example, if we are using a
square object to represent the points and X,Y axis represent the date
and closing price respectively, say. Then we can vary the size of the
squares according to the amount of trading on the stock on that day,
which would yield an interesting graph. The expression is even more
powerful as the size can then be varied according to a combination of
attribute values.

\subsubsection{Color}

Similar to the size of the attribute, the color can also be made as a constant or variable or an expression. There are currently 43 colors defined and any expression or variable value computed is mapped to one of these colors. The use of variable color also yields interesting results. For example in a scatter plot we can change the color according to the rainfall on a day, with date and temperature on the axes. This would yield a plot with varying colors indicating the amount of rainfall on a day.

\subsubsection{Pattern}

The pattern may also be changed for the objects. The pattern
essentially decides the type of pattern to be used for filling the
objects displayed.

\subsubsection{Orientation}

The orientation decides the tilt of the objects that would be
displayed for the filed values. This is useful for objects like
rectangles for example to get a diamond object. As before this also
can be defined as a constant, variable or as an expression.

\subsubsection{Shape}

The points can be plotted as one of the following, Rectangle (Rect),
Square (RectX), Bar, Polygon, Oval, Vector or Block. The shapes again
may be defined as a constant or variable values. The default is the
bar. The square (RectX) gives a visual square shape, block represents
a 3 dimensional block, and a polygon represents a dodecagon. The
vector is a directional one, with an arrow at the end. The bar begins
from the point (x,0) to the point (x,y).

\subsubsection{Shape Attributes}

The shapes have associated with them the width, height and other
values. This can also be modified in the Edit window. The shape
attributes vary for each shape. It is summarized below.

\begin{tabbing}
Polygonx \= Directional vectorx \= ShapeAttr0x \= Shaprattr1x \= Shaperattr2x \kill
Shape \> Meaning \> ShapeAttr0 \> ShapeAttr1 \> ShapeAttr2 \\ \\
Rect  \> Rectangle \>  Width \> Height \> N.A.  \\
Bar   \> Bar from X axis\> Width \> N.A. \> N.A. \\
Oval  \> Oval \> Xwidth \> Yheight \> N.A.\\
Vector \> Directional Vector \> Xwidth \> Yheight \> N.A. \\
Block \> cuboid \> Xwidth \>\\
RectX \> square \> Xside\\
Polygon \> do-decagon \> Xwidth \> Ywidth\\  
\end{tabbing}

\subsection{View Properties}

A view is used to display a range of the attributes selected for that
view. Each association of an X attribute to a Y attribute is a
view. The views may be acted upon in a number of ways to yield the
required results.

\subsubsection{Title}

The view title is independent of the graph displayed and may be
modified by the user. As a default the view name is "$<X attrib name >
VS <Y attrib name >$" (e.g. DATE VS SALES. The title may be edited
using the Title choice under the View option.  The title may be
deleted in which case the {\em current view indicator} which is
normally a rectangular boundary surrounding the view becomes a
rectangular bar on the left edge. \Devise attaches no significance of
the title and the data displayed in that view. It is left to the
discretion of the user.

\subsubsection{Axes}

The X axes and Y axes normally have their axes displayed and the min
and max values displayed. The axes may be turned off, in which case
only the data is displayed without the min, max values and axis
lines. The X axis may also be selectively toggled on or off for a
view.

\subsection{Display of Graphical Data}

This subsection deals with the display of the Graphical data and also
visual queries that can be executed on them. The graphical data are
displayed in many views on many windows. The queries on the views are
essentially in the form of selecting a range by zooming in, zooming
out or scrolling.

\subsubsection{Visual Filter}

A visual filter defines a query over the graphical data attributes of
the gdata. Our implementation supports range query over the X and Y
GData attributes. The visual filter is used to specify portions of
GData to be viewed. i.e. the X range and Y range on a view is the
visual filter for that view, since it filters out the rest of the
GData from being displayed.

The visual filter can be changed explicitly by editing the X/ Y high
and low values in the control panel or by scrolling and zooming on the
data displayed.

As explained earlier the history buffer maintains a list of visual
filters for that view. As a default the filter value is from 0 to 100
unless the attribute range has been specified.

\subsubsection{View}

A view is used to display those gdata that fall within the range of
the visual filter. Currently there are two types of views: Scatter and
SortedX. The Scatter view is used to draw a scattered plot. The
SortedX view implements optimizations used to reduce the time used to
draw the gdata if the X attribute is sorted. The view is basically a
graph of the values of the attributes chosen. The views may be
manipulated in many ways for effective visualization.

\subsubsection{Window}

A window provides the screen real estate used by views to draw the
gdata. It it also responsible for arranging views in its
boundaries. Currently, \Devise supports tiled/automatic, vertical, and
horizontal window layouts.

A view can be removed from a window to reduce clutter, or be moved to
another window so that related views are brought together for
comparison.

Windows can be duplicated, a very handy feature.

In the arrangement of views within windows, the automatic mode selects
the appropriate layout according to the dimensions of the
window. e.g. if the window is long and thin, then a horizontal
arrangement is chosen etc. The views may also be arranged in a fixed
way viz. Horizontal or Vertical and the number of views to be stacked
either horizontally or vertically may also be defined.

When changing the window sizes, sometimes (particularly when the data
set is large) it may be desirable to stop \Devise from redrawing the
graphs as the windows are altered and instead draw after all the
changes have been done. Such a provision is provided by the Layout/
Display button in the control panel. Toggling it to the Layout mode
stops \Devise from redrawing any views, while the window shapes may be
altered. After the changes are done, toggling the switch back to
Display mode starts the activity again.

\subsection{Visual Assistants}

\subsubsection{Visual Cursor}

A cursor is a visual band that helps to map the boundary of a view
within another view.  i.e. it displays the X/Y boundaries of one view
within another. This is useful to picturize the range of one of the
view with respect to the other (a possibly) longer range view.(One
could be the view of the entire range of values, whereas the other
could be a part of that range)

A cursor contains both a source view and a destination view. The
source view is where the cursor fetches information about the current
view X/Y axes boundaries. The boundaries of the source view are drawn
as line segments in the axes of the destination view. The cursor
displayed in the destination view can be moved by clicking below/left
of the axis lines (by changing its visual filter) at the location
desired.

\subsubsection{Visual Link}

A visual link represents a connection between two views on any one or
more of the graphical attributes viz. X/Y axes values, color, size
etc. By linking two views together any change in the linked
parameter, manifests itself as a change in the corresponding linked
view also. For example when two views are linked together on their X
axis, then scrolling along the X axis on one of the views would
automatically scroll the other view along the X axis.

The visual links can be named for say reference in the case of adding
a view to the link or removing one from the link. There can be any
number of links existing at a time between all the views.

\subsubsection{History}

The history field (as described in the tutorial section) is used to
record the previous values of the visual filter. i.e. it records the
max and min of the X and Y axes for all the views. The history buffer
gives the history of the current buffer and using the history buffer
one can go back to an interesting filter. The history values can be
used by clicking on the appropriate value in the history window and
then {\em using} it.

The history mechanism also provides for {\em marking} of filters,
that forces \Devise to store them in the buffer, in the case when the
buffer becomes full and some older filter values need to be thrown
away.

\subsection{Statistics}

\Devise provides Mean, Max and Min calculations along with the Kiviat
Graph method. These are discussed below.

\subsubsection{Min, Max, and Mean}

The mean, max and min are displayed as lines in the views for which
they are applied. The mean, max and min are calculated only on the
range of the values displayed in the particular view and not on the
entire range of the data.

\subsubsection{Confidence Intervals}

The confidence intervals are visually shown as transparent bands
around the mean line. The confidence interval can either be 85, 90 or
95 \%.The default is no confidence interval.

\subsubsection{Kiviat Graphs}

The Kiviat graphs provide a visual method of analyzing the mean, max
and min values over the set of views in a particular window. The graph
can be used to analyze the balance in the values of the various
views. The Kiviat graph can be drawn for either the min,max,mean or
count of the data values. the Kiviat graph changes with changes in the
visual filters of the corresponding views. Also by clicking on the
edges of the graph, the values on the vertices (i.e. the mean
etc. value for 2 of the views) can be found out.

\subsection{Session and Template}

A session is simply the collection of windows, views and the mappings
defined for each of those views. It is the collection of the data
stream, the schema, GData mapping, the various links, cursors, and
graphical and statistical attributes for the various views. In short
it is the state of each visual query.

The template is a session without the associated data stream. i.e. it
consists of all the mappings etc. which can be applied to any data
stream confirming to the schema. This is useful to view the data of
different streams under the same mappings and state.

The current state of the \Devise can be stored as either a session or
as a template. The required data stream is to supplied by the user
when opening a template.

\section{Data Stream Management}

A data stream as already explained is just a sequence of records. It
can come from any source like the World Wide Web or COMPUSTAT tape
etc. This section talks about the various parameters of a data stream
and their functions.

\subsection{Data Stream Catalog}

The data stream catalog contains the name of the data streams defines,
their associated physical schema and the caching information. The data
stream catalog can be opened by using the Import option under the
Visualize menu. The catalog can be browsed and new data streams can be
added and existing ones modified. The data is read from the source and
cached on to the disk. This is indicated by the word Cached in the
catalog. A data stream cannot be visualized unless it is cached on to
the hard disk.

The catalog provides the following options.
\begin{description}
\item[Define] This option allows the adding of new data streams. The
new data stream may be manually configured or automatically found and
added.
\item[Stream] This menu provides options to edit, copy and delete data
streams. The edit option brings up the Data stream definition window
to modify the stream parameters.
\item[Display] This is used to limit the number of streams listed in
the catalog and also provides options for sorting the streams.
\item[Follow to] This is used to {\em follow} from a data stream for
a company in a particular schema to the same company's data in another
schema. e.g. to follow from a company's data stream in COMPUSTAT to
that in CRSP.
\item[Help] This provides some on-line help for using the catalog.
\end{description}

\subsection{Data Stream Definition}

The edit data stream definition window is used to create or modify a
data stream. The user needs to select the type of source
(e.g. COMPUSTAT), the logical schema file associated with that
stream, a key to uniquely identify the stream from the rest, a command
(if needed) and cache parameters. These are discussed below.

\subsubsection{Display Name}

The display name is the name of the stream to be defined. This is the
name that will be listed in the catalog as the data stream name. It
could be the name of the company whose data stream is being defined or
could be any other distinguishable name.  e.g. ARCHER DANILES MIDLAND
CO. This name must be unique within the catalog.

\subsubsection{Source}

The source option indicates the source from which data is to be
retrieved. There are 9 possible sources namely, COMPUSTAT, CRSP,
COMMAND, NETWORK, ISSM, WWW, SQL, SEQ and UNIXFILE. The source
determines where \Devise has to search for, in order to get the data
for caching. The COMPUSTAT, CRSP and ISSM, by default are tape
oriented. The WWW needs a URL to extract the data. The SEQ and SQL
need queries to evaluate and the COMMAND option needs a valid UNIX
command.

\subsubsection{Key}

The key is a name that is used to uniquely identify the stream from
the rest when storing. Usually the stream definitions are stored as
the source name followed by the key name. Thus keys selected should be
unique.

\begin{enumerate}
\item
The keys for the UNIX source must be a valid pathname in UNIX.
e.g. /p/devise/dat/UNIX\_data1

\item
The keys for the WWW, NETWORK, COMMAND, SQL and SEQ must be a unique
and valid name chosen by the user. The actual filename is obtained by
concatenating the source with the user chosen name. Thus the key
chosen needs to be unique only for that domain. e.g. there can be a
key ATLAS with source SQL and another ATLAS with source COMPUSTAT and
these are considered different.

\item
For the ISSM database, the key chosen must be the 3 alphabet code for
the company. \Devise provides an automatic listing of the 3 digit Alpha
codes for all the companies through the Select option.
\end{enumerate}

\subsubsection{Schema}

The schema option is used to associate the correct schema for the
stream. Remember that \Devise parses and understands the stream with
the help of the schema. The schema is not actually used to retrieve
the data, but is used when caching it on to the TData form. The schema
must be previously defined and stored in a file, whose name is
mentioned here. It is important to note that it is the logical schema
that is stored here. The schema name needs to be a full file
pathname. It can also be browsed and selected through the Select
option provided.

\subsubsection{Evaluation Factor}

The evaluation factor specifies the relative time at which the data
stream is to be extracted from the specified source and cached on to
the local disk. A evaluation factor of 100 would force \Devise to get
the data from the source, as soon as the stream is defined and cache
it. A low value of the evaluation factor delays this procedure and a
value of 0 would postpone this to the time of actual visualization of
the stream.

\subsubsection{Cache Priority}

The cache priority indicates the priority to be given to this stream
in the disk cache. When a large number of items are cached, the disk
would run out of space. At this point \Devise uses the cache priority
specified and removes the least priority item from the disk cache. As
a default it is specified as 50\%.

\subsubsection{Info/Command}

The command window is used to take in a command for the SQL, SEQ and
other data sources. For the case of the WWW source the URI (Uniform
Resource Identifier) needs to be specified. (e.g. \code{ftp://...}) In
the case of SEQ query the query is sent to the SEQ program which
evaluates it and the tuples returned are cached. The command could be
any UNIX command that can supply a stream of tuples
(e.g. \code{awk...}).

\subsection{Manipulating the Catalog}

This section deals with manipulating the catalog, part of which has
already been discussed above.

\subsubsection{Create}

The data stream creation can be done by using the
Visualize/Import/Define/New option.  The data stream to be created is
associated with the following.

\begin{description}
\item[Display name] This defines the name of the data stream
\item[Data source] The data source whether COMPUSTAT, CRSP etc. needs
to be defined.
\item[Key] As already explained, the key name is to be given to
uniquely identify the stream. In case of UNIXFILE type it is the full
pathname, in case of Business Databases, it is the unique numeric
identifier and in the case of the queries, it is a user chosen
name. The names must be unique within their domain.
\item[Schema file] The logical schema file needs to be specified.
\item[Evaluation factor] This specifies the time at which the stream
must be cached. A high value results in caching immediately after the
data stream is defined and a lower one postpones it to the
visualization time.
\item[Cache priority] This specifies the priority of storage of the
stream on the disk cache.
\item[Info/Command] This is useful for the query type of sources. The
query or command (say \code{http://...}) etc. is needed to be
specified in this.
\end{description}

%% Major problems here ... The edit did not check the actual schema even after 
%% uncaching it.. Should we uncache it?

\subsubsection{Edit}

The editing of a stream can be done by using the Stream/Edit option in
the Data stream window. The edit is similar to the creation of a new
stream and hence is left undiscussed.

\subsubsection{Copy}

This is used to copy an existing data stream definition on to a
different data stream name. The same edit window is displayed, but
with the parameters of the data stream to be copied. A unique name is
all that is to be given to complete the copy phase.(Any other
parameter can also be changed) This is useful for defining a lot of
streams having same characteristic, like schema etc. This option is
available under Stream/Copy in the data stream window.

\subsubsection{Delete}

This is used to delete an existing data stream. The stream is selected
first and then deleted. Available under Stream/Delete in the stream
window.

\section{Defining and Browsing Schemas}

\subsection{User and Superuser Modes}

\Devise can be brought up in one of two modes --- the user mode or the
superuser mode.  This is decided by the value of the variable
\variable{UserMode} defined in the \filename{.rc} file read by
\Devise.  A value of 1 indicates that it is run in the user mode and 0
indicates the superuser mode.

The mode determines the capabilities and presentation of the schema
browser.  In the user mode, the browser can be user to inspect the
attributes in any of the already defined schemas. Further, the views
defined on these schemas (through groupings) can be seen and
modified. New views can also be defined.

In the superuser mode, the browser provides complete control over all
the schemas. Existing schemas can be opened to display the types of
the attributes and their high and low values (if defined). The schemas
can also be modified, new attributes inserted, range values modified,
etc. Entirely new schemas can also be defined.  This is an operation
that would be performed by the superuser when a new data source (with
a new data layout) is to be visualized through \Devise.

\subsection{Schema Browsing in User Mode}

When the schema browser is invoked in the user mode, a list of all the
schemas defined in the system is presented. The user can see the list
of attributes in any of these. Further, the set of views defined on
these can be seen as a list.  The user can either choose an existing
view and open it up or specify the name of a brand new view to be
created. In either case, a group browser comes up and the user can
modify/define the groups and its contents in terms of base attributes
or subgroups. Finally, if a new view is saved, it becomes a part of
the list of views on that particular schema.

\subsection{Schema Browsing in Superuser Mode}

When the schema browser is invoked in the superuser mode, the list of
all the schemas currently defined in the system is displayed as a
list. Any of them can be chosen to be opened or a brand new schema can
be specified to be created. In any case, the opened schema is
displayed in terms of the list of all its attributes. The types of
these attributes can be one of those shown. Also, the high and low
values that these attributes can take can be specified.  This will help in
the automatic selection of scales during display of data. When a new
schema is saved, it becomes a part of the list of schemas and a
default view is also defined.  This view will consist of all the
attributes of the schema and will have no group structures. This can
be modified or new views defined at any time.

\section{Mapping: Association of Data Sources in Different Schemas}

\subsection{Defining Mappings}

The user can define mappings between data sources of different
schemas.  For example, associations can be established between
companies in \term{COMPUSTAT} and \term{CRSP} databases. The user
interface to define the mappings is extremely simple and
straightforward. First, the two schemas to be mapped are specified and
also the names of files used for storing the following:

\begin{itemize}
\item Mapping table --- this will be the list of associations
\item Unresolved data sources of the first schema 
\item Unresolved data sources of the second schema
\end{itemize}

It is necessary for the user to specify the file names since they have
to be used later to manually resolve the conflicts and unresolved
sources. The list of key attributes of the two schemas are displayed
and the user can pick pairs that are to be matched against each other.
The system will then automatically generate all the mappings based on
the rules specified by the user.  Any sources that are not resolved
are dumped into separate files which the user can then manually
browse.

\subsection{Resolving Conflicts and Unresolved Sources}

After the automatic generation of the mapping table, in addition to
the file containing the mappings, there are two files containing the
unresolved sources of the two schemas. By specifying these three file
names, the user can bring up the list of all the unresolved sources
and manually define the mappings based on his/her knowledge of the
sources.

\subsection{Using Mappings}

The mappings generated either automatically or manually are used to
switch between equivalent data sources in different schemas
easily. While browsing through a list of data sources in a particular
schema, say \variable{X}, the equivalent data source in a different
schema, say \variable{Y} can be loaded with a mouse click if the
mapping between \variable{X} and \variable{Y} has been defined.

\section{Statistics and Kiviat Graphs}

\subsection{Statistics}

The statistics of the displayed data can be visualized. Specifically,
the mean (average), minimum and maximum can be seen superimposed on
the data.  There are options to view the statistics for either the
current view or all the views.  Further, confidence intervals can also
be displayed.  The 85\%, 90\% and 95\% confidence intervals can be
visualized as transparent color bands around the mean line.

\subsection{Kiviat Graphs}

A Kiviat graph is a powerful method for displaying the collective
information of the data displayed in several views. For example, a
Kiviat graph used in the context of performance data can be used to
see how well the system is balanced. The Kiviat graph for the mean,
maximum, minimum or the count of the data displayed can be created. If
the underlying data changes (due to zooming or axes change), the
Kiviat graph keeps track and displays information about the current
data (We can say that a link exists between the data views and the
Kiviat graph built on them). The actual values along the various axes
of the Kiviat graph can be brought up by clicking the middle mouse
button on the graph. There are options to reuse the same Kiviat graph
to display different information or create new graphs for each.

\section{Creating a Visualization}

This section deals with a detailed view on creating a
visualization. For those who have read the tutorial session before, a
substantial amount of the material covered in this section might
appear to be repetitive, but this gives a step by step process of how
to create a session rather than just how to use it.

The first step before visualization is to know what are the data
streams available and what are the attributes that would be useful for
this visualization. This section would basically illustrate the
operations performed with a demo stream.

Before creating a visualization, open the Visualize/Import menu and
choose a stream to be visualized. e.g. the IBM Stock(MIT Database) A
Define Visualization window appears. Now use this stream for creating
the visualization.

The Define Visualization window contains the parameters explained in
the previous section.

\subsection{Window Name}

The name is the title of the window that would carry all the views to
be defined for this stream. The window name is by default the stream
name itself. The window name defined here is fixed for the life of the
window and cannot be modified.

\subsection{Window Layout}

The window layout specifies the arrangement of the views within the
window. A vertical layout would arrange the views on top of each
other. A horizontal one on the other hand would arrange it side by
side and a tiled one would arrange the views in a neat square. The
tiled version would be overridden by a vertical layout if the number
of views in the window is not even. The shape of the window changes
with the layout. For a vertical layout the window created is long,
whereas that for a horizontal layout is wide.

Remember that the layout can be changed any time by using the
Window/Layout option.

\subsection{X Attribute}

The X attribute specifies the filed in the schema that must be used as
the X axis attribute for the views to be defined. The option
automatically provides all the attributes available for
selection. When importing a file with many logical groups the groups
needs to be selected prior to using the Define visualization window
(This is automatically asked for by \Devise). Hence only the attributes
present in the top level groups selected are displayed.

\subsection{Y Attributes}

These are displayed as a number of buttons that can be toggled on and
off. Pressing the button with an attribute name would create a view
with that attribute in the Y axis. In case of groups the button is
represented in a different color and clicking on them yields a sub-menu
and so on. The views generated will have the same X attribute and one
of the Y attributes chosen.

\subsection{View Background Color}

The background color option allows one to choose the specific
background color needed for the views.  A color chart is presented for
selection. This background color pretty much stays along with the
view, even when it is moved to a new window.

\subsection{View Title}

The view title option is used to indicate whether the title is to be
displayed for the views or not. The default title is {\em X attribute
Vs Y attribute} . Again remember the title can be changed any time
from the title option under the View menu. If the title is not
displayed, the current view indicator becomes a side bar instead of a
surrounding rectangle.

\subsection{Link X}

This option is used to indicate if the views are to be linked along
their X axis. Recall that the linking of X axes of all the views would
essentially lock them along the X axis, .i.e. a scroll in one of the
views along the X directions would scroll the other views too! The
name of the X axis link is {\em X attrib Link} . This name is useful
to select, when removing a view from a link or adding a new one to a
link.

\subsection{X and Y Axes}

The X and Y axes toggle buttons determine whether the axes must be
drawn or not. i.e. whether the axes line and the high and low values
need to be drawn or not. In case of cursor creation it is necessary to
have the X axis, to be able to move the cursor, as was explained in
the tutorial section. The X and Y axes can always be toggled using the
options under the View menu.

\section{Visual Filter}

A visual filter is the range of X and Y values. The filter filters the
data from the GData values and displays it on the screen. It can be
changed in two ways as mentioned below.

\subsection{Change Visual Filter}

The visual filter is shown visually as a separate box in the control
panel. The X and Y highs and lows in that represent the current filter
that is active for the current view. To change the value the numbers
can be directly edited in the boxes. After editing them press use to
commit them or use undo-edit to erase all the editing done.

The back one option allows to go back one filter in the history. Again
press use to commit it or use undo-edit to undo the change.

\subsection{Filter History}

\Devise maintains a filter history, which is nothing but a list of X
and Y extremes stored in the reverse order in which they were last
used. Users can go back to those filter values by clicking on the
entry in the history window, and then clicking the `use' button on the
control panel.

Users can also mark or unmark places of interest by clicking in the
`mark' column of the history window. The marked entries are preceded
by an `*'.  Because the history window has a finite size, old history
entries are replaced by new ones after the size of the history window
reaches it maximum. However, marked entries are never replaced, unless
there is no other way.

\section{Graphical Data}

This section on Graphical data presents the various options and
choices for manipulating the mapping from the TData to GData.

The mapping from the TData and GData is partially specified when the
use chooses the options at the time of importing the data through the
Define Visualization window. The Edit mapping sub menu under the View
menu, presents a much more wide spectrum of choices to change the
mapping. These are discussed below.

\subsection{Editing the Mapping}

As said above after importing the data or opening a session, click on
the Edit mapping option under the View menu to open a range of choices
to change the mapping. This brings up a separate window thorough which
the changes can be done. As explained in the before, the changes that
can be done are

\begin{description}
\item[X] The attribute mapped to the X axis can be changed using this
option. A constant value gives the value of the Y axis for a
particular X.
\item[Y] This helps in changing the attribute mapped on to the Y axis.
\item[Color] The foreground color of the views can be changed using
this option. A constant value paints the graph with the constant value
selected, while a variable value maps the attribute value to one of
the base colors (using modulus {\em No of colors}) An expression
works in a similar way.
\item[Orientation] The orientation option rotates the objects about
its center. Thus giving a proper orientation for a square can make it
to appear as a diamond.
\item[Size] The size changes the size of the object. The default
values specified using the ShapeAttr is size 1.
\item[Pattern] The pattern is used to change the filling style for
filled objects.
\item[Shape] The shapes may be any one of Rect, RectX, Oval, Polygon,
Bar, Block or Vector with the qualities as already explained. The
variable and expression values if chosen map the value given to one of
these (again using modulus on the {\em Number of shapes}
\item[ShapeAttrs] These are used to change the dimensions of the
objects displayed. This have been already discussed.
\end{description}

\section{View Properties}

The view properties section deals with manipulating the views, windows
etc.

\subsection{Swap View Positions}

To swap two views, choose the two views by clicking on them one after
the other and use the option Swap views under the view menu. The two
views must be in the same window for this to take effect.

\subsection{Move View to Another Window}

To move a view to another window, click on the view to make it current
and then use the option View/Move to Window. This brings up a table of
window names. Choose the required window name to move the view to that
window. You can also move the view to a new window by using the New
option in the table. This brings up yet another window prompting the
user to enter the window parameters like size etc. and then moves the
view to that window.

\subsection{Remove a View}

To remove a view, simply click on it and then use the Remove option
under the View menu. This removes the view visually from the window,
but not from \Devise's memory.

\subsection{Bring a View Back}

To being back a view to the window, use the View/Bring back to Window
option. This presents a list of view names to choose from and after
the selection prompts for the window to put the view in. This option
may also be used to copy a view to another window, since all the views
(not just the ones removed) are displayed when in the view selection
box.

\subsection{Destroy View}

%% Did not understand this...

\subsection{Switch TData}

When visualizing a particular data stream, it may be often the case
that the user wishes to see the data stream in the same format as that
of the current session to compare. The switch TData helps in
simplifying the task, by switching only the TData component (i.e. the
data streams) without changing the filters, view and window
layouts. The switched data stream must conform to the same logical
schema as that of the current view.

Before using this option the current view, whose TData is to be
changed must be selected.  Now choosing the option, presents a table
showing the data streams currently open. This is to facilitate the
switching of TData to the data streams already open. Choosing other in
the table brings up the entire data stream list, from which the
desired data stream can be selected.

%% Don't know if all the views with the TData will be changed or
%% only that TData will be changed.

\subsection{Title}

The title option is used to set, clear or edit the title of a
view. The option brings up a title box containing the present title
for the current view. This can be cleared or edited. Pressing OK
commits the changes. Clearing the title and pressing OK would remove
the title for the view. This has the side effect of changing the
current view indicator from rectangle surrounding the view to a bar on
the left of the view.

\subsection{Axes}

The Axes option allows one to choose to have the axis line and the
high and low values displayed or not. It is important to note that
switching off the axis line in the destination axis of a the cursor in
a view, would disable one from moving the cursor under the destination
view.

\section{Window Properties}

This section deals with the manipulation of windows.

\subsection{Change View Layout}

The view layout may be changed by using the Window/layout option. The
user is presented with three options viz. Automatic/ Vertical/
Horizontal. The Horizontal mode forces \Devise to display the views
side by side, whereas the Vertical mode makes it display the views one
on top of each other. The automatic mode adjusts the layout depending
on the dimensions of the window. If the window is long and
lean,(longer on the horizontal direction than the vertical direction)
then \Devise goes in for horizontal layout mode, whereas if the window
is short and wide (i.e. longer on the vertical direction) it goes in
for vertical layout.

The width and height parameters affect the number of views to be
displayed horizontally or vertically. The Width option specifies the
number of views to be displayed side by side in the vertical layout
mode. For example a value of 2 in it would display the views in pairs
of two vertically. The height determines the number of views to be
displayed on top of one another in the horizontal layout mode.

\subsection{Duplicate Window}

\Devise provides a method of duplicating a window. This is available
under Window/Duplicate option. This duplicates the while window
including the layout, visual filters, colors etc. This is useful in
defining cursors and switching TData options. The new window is given
a name as the old windows name with the appended (Auto \#) where \#
represents the number of same windows existing.

%% Can we change the window name ??

\subsection{Remove Window}

To remove a window first all the views in it must be removed by
clicking on the views and using the View/Remove from Window
option. Then to remove the window use the Window/Remove option. You
can also remove the window by simply double clicking on its top left
corner button.

%% Does it work..?

\section{Visual Link}

The visual link helps to link two or more views through a common
characteristic say the X axis range, Y axis range, color
etc. Changing one of the filters would change the filter in the
linked view also. The following sub sections deal with the
manipulation of the view links.

\subsection{Create Link}

To create a link, we can use the Link X toggle at the time of
importing in the Define Visualization window. This creates a link
between all the views defined using that data stream. (through that
definition)

After the views are visualized too, we can create new links. This can
be done using the Link option under the View menu. Choosing this
option brings up a GetLink window that lists all the current links
available. Clicking on Info brings up the views linked together by the
chosen link.

Use create to create a new link. The link may be on any of the
combination of the following attributes viz. x, y, color, size,
pattern, orientation, shape. The link name must be supplied for this
to be added to the list of links existing. Then the views may be added
to this link as explained in the next subsection.(After a new link is
added the Link option automatically prompts for linking the current
view to anyone of the existing links. Press Cancel if you do not wish
to add the current view to any of the links)

\subsection{Link}

The link option in the View menu, also helps in linking the current
view with a particular link name. This would link it to the other
views already defined under that link name, with all the attributes
defined. i.e. if the link name specifies linking with the X and Y axis
say then linking a view to that link name would link it on the X and Y
axis to all the other views under that link name.

\subsection{Unlink}

To remove a link, use the unlink option available under the View
menu. This brings up a list of link names. Click on the link name from
which the view is to be removed.  .

\subsection{Delete Link}

%% Dont know how ? 

\section{Visual Cursor}

The visual cursor as already explained is a useful tool to superimpose
the visual filter of the source view on that of the destination. The
following subsections explain the

\subsection{Create Cursor}

To create a cursor, use the Cursor/Create option. The cursor name needs
to be supplied and also the type of cursor link i.e. whether on the X
axis or Y axis or both is to be supplied. This creates a new cursor
with the required characteristics.

\subsection{Set Source}

The cursor needs to have a source and a destination. Click on the view
to be made as the source. Then use the Cursor/Set source option. This
brings up a list of cursors to choose from. Choose the required
cursor, or use the new option to define a new cursor. This sets up the
cursor source.

\subsection{Set Destination}

To select the destination proceed as above and use the Cursor/Set
destination option instead. A new cursor may also be defined and the
destination set.

\subsection{Delete Cursor}

To delete a cursor, use the Cursor/Delete command. This deletes the
cursor band displayed on the corresponding views.

\section{Statistics}

As already explained, \Devise provides for two type of statistics
measurement. One is the use of visual lines to represent the mean, max
or min in the view. Another is the use of Kiviat graphs. The usage of
these is described below.

\subsection{Enable Statistics}

To enable statistics the View/Toggle Statistics menu is chosen. The
Statistics sub-menu is a tear down window. i.e. by clicking on the
dotted lines on the top of the sub-menu, a new window is created with
the options, and it can be used from then on to control the statistics
displayed, without the need of going back to the View option. (Or the
sub-menu may be directly used from the View/Toggle Statistics option,
without creating this additional window) The enable statistics sub-menu
has 3 main options. One contains the choice of statistics to be
displayed, i.e. whether Mean and or Max and or Min. The second is the
choice of Confidence intervals, which may either be 85, 90 or 95
percentage, or none at all. The third is the choice of applying the
statistics option to all of the views or only the current one.

The choices may be chosen by clicking on them. This shows up as a
black colored diamond/ square before the choices. After the choices
are selected, click on Apply statistics to apply the changes to the
view.

This then represents the statistics chosen as lines on the view. Again
remember that the statistics displayed is for the range of X values in
the view only and not for the whole set of data.

The window may be closed at any time, as any other normal window, by
double clicking on the top left corner.

\subsection{Disable Statistics}

To disable the statistics option, simply open the statistics window
(If you have the window already existing, use that instead) Click off
the Mean/Max/Min options and Apply the changes. The statistics option
is now turned off. The stats may be turned off for the current view
alone or for all the views.

\subsection{Enable Kiviat Graph}

The Kiviat graph enabling is quite similar to that of the Statistics
one. Use the View/Show KGraph option, This too is a tear down
window. It has again 3 main option. The first presents the choice of
Mean/Max/Min/Count as the statistical value to be used in the
Graph. The second controls the range of the views to be used,
i.e. whether the views in the current window or a specific set of
views.  The third option gives the flexibility of choosing a new
KGraph window to display the choices selected or to reuse the last
KGraph window.

The KGraph displayed changes with the change in the visual filter of
the underlying views.

The KGraph may be disabled by closing the KGraph windows. (double
clicking on the top left corner button)

\section{Using Sessions and Templates}

This section deals with the use of sessions and templates. A session
as described earlier is the set of all the data streams, their GData
mappings, window preferences, etc. In short it contains enough
information to recreate the same screen as viewed before. When the
user initially starts with importing the data, then creating all the
view mappings, window styles etc., he is basically running a session
with \Devise. The session may be saved and retrieved later in 2 ways as
described in the following sub sections.

\subsection{Save Session}

In order to save the session as such the save option can be used. Like
any other normal save option, this would overwrite any already
existing session file under the current name. When saving this a save
box appears which prompts the user for the filename to save under, if
no name has been associated for the session. The path and filename can
be selected and the current session saved. Be aware that the older
version of the session file would be substituted with the new version.

\subsection{Save Session as Template}

In the ordinary save option, the data stream associated with the
session is also saved. The template as mentioned earlier, is just the
mappings and other preferences without any associated data stream. Any
data stream may be joined with a template, provided it has the same
schema, to get a new visualization. The save as template choice gives
the option of saving the current session as a template. This could be
opened any time later and associated with one or more data streams to
get a new visualization, with the same environment (i.e. same window
locations, sizes, mappings etc.)

\subsection{Restore Session or Template}

To restore the session stored, use the open option in the session menu
and choose the desired session. The session may be restored as a
template, i.e. the session may be invoked as a template and by
choosing a new data stream for the template, one can get a new
visualization. When opening the session as a template, \Devise prompts
for the data streams to be associated with the template. This is
equivalent to storing the original session as a template and opening
the template as explained below.

\subsection{Restore Session as Template}

The restoration of templates is similar and \Devise would prompt for
the data streams to be chosen. The user needs to choose as many data
streams as they were in the session that was stored as a template. The
streams chosen must match the schema of the original streams. After
this a new session is in effect with the new data streams.

\subsection{Pixel Image of View}

%% Dont know what this means.....

\section{Exporting Data and Window Images}

The Data and Window images may be either printed or exported to
different print file formats.

\subsection{Print Window Image}

The window image may be printed using the Print option under the
Session menu. Check the {\em To Printer} button to print. All the
views or only the selected view may be printed.

\subsection{Save Window Image to File}

The window image may be saved as a GIF file, postscript file or as an
eps file. This can be done through the print menu, by selecting the
{\em To file} option instead of the printer. A sub-menu offers choice
on the 3 file formats. Also it is possible to save all the views or
only the selected view.

\section{On-line Help}

An overall on-line help is available. This can be reached by the Help
menu in the Control panel. There is also help available for the Data
definition windows and some others. There is however no context
sensitive help yet!

\section{Exiting \Devise}

To exit \Devise, use the Quit option under the Session menu. Remember
\Devise does not confirm whether the current session is to be saved. It
only confirms of you do want to quit, so make sure that you have saved
whatever relevant needs to be saved before exiting. Upon exit, \Devise
removes all the windows and graphs that it created during the
sessions.

\section{Internet Information Resources}

The \Devise WWW page at
\code{http://www.cs.wisc.edu/~jussi/devise/devise.html} contains a
browseable version of this document.

\newpage 
\section*{Copyright}

\verbatiminput{Copyright}

\section*{Disclaimer}

\verbatiminput{Disclaimer}

\newpage
\section*{Agreement}

\verbatiminput{Agreement}

\end{document}
