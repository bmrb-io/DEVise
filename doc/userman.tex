%*******************************************************************************
% DEVise Data Visualization Software
% (c) Copyright 1992-1997
% By the DEVise Development Group
% Madison, Wisconsin
% All Rights Reserved.
%
% Under no circumstances is this software to be copied, distributed,
% or altered in any way without prior permission from the DEVise
% Development Group.
%*******************************************************************************
% File: userman.tex
% Last modified: Thu Feb  6 14:41:30 1997 by Christopher Weaver
%*******************************************************************************
% NOTES:
% How to use indirection on Fig refs in text?
% The Data Streams section (sec. 4) was not revised...I don't get it.
%*******************************************************************************

\documentstyle[epsf,fullpage,verbatim]{article}

\renewcommand{\topfraction}{1.0}
\renewcommand{\bottomfraction}{1.0}
\renewcommand{\textfraction}{0.0}
\advance\intextsep by 5pt

\def\filename#1{{\tt #1}}
\def\code#1{{\tt #1}}
\def\menu#1{{\tt #1}}
\def\term#1{#1}
\def\variable#1{{\tt #1}}

\def\scaleepspic[#1]#2#3{
\begin{figure}[htb]
\centering\leavevmode\epsfxsize=#1\epsfbox{#2}
\caption{#3}
\end{figure}
}

\def\fullepspic#1#2{
\begin{figure}[htb]
\centering\leavevmode\epsfxsize=\textwidth\epsfbox{#1}
\caption{#2}
\end{figure}
}

\begin{document}
\title{DEVise User Manual}
\author{The DEVise Development Group \\
\code{devise@cs.wisc.edu}
}
\date{\today}

\maketitle

%*******************************************************************************
% Introduction
%*******************************************************************************

\section{Introduction}

COMPLETE ME

%*******************************************************************************
% Installation, Operation, and Customization
%*******************************************************************************

\section{Installation, Operation, and Customization}

COMPLETE ME

\subsection{Extracting Files from the \code{tar} Archive}

COMPLETE ME

\subsection{Path Names and Environment Variables}

COMPLETE ME

\subsection{Running with the \code{devise} Script}

COMPLETE ME

\subsection{Customization with \filename{devise.rc}}

COMPLETE ME

%*******************************************************************************
% Concepts
%*******************************************************************************

\section{Concepts}

This section explains the concepts behind DEVise in depth. DEVise is a ``data
extraction and visualization'' tool, and has the primary purpose in assisting a
human operator to discover patterns in data sets by free exploration of the data
from varied graphical perspectives. The process of promoting raw file data to a
specific graphical presentation is conceptually layered as follows:

\begin{itemize}
	\item Data Streams
	\item Physical Schema
	\item Logical Schema
	\item Graphical Data
	\item Views and Windows
\end{itemize}

Additional concepts behind DEVise include:

\begin{itemize}
	\item View Statistics
	\item Sessions and Templates
\end{itemize}

%*******************************************************************************

\subsection{Data Streams}

A {\em data stream} is a sequence of records to be visualized using a particular
schema. For example, the daily stock prices for a company is a data stream. Data
streams may come from different sources, such as a UNIX file, magnetic tape, or
a database query output stream.

Each data stream is associated with a data source and a schema. The schema
describes the layout and data fields of records in the stream, as well as a key
to uniquely identify the stream. For a database stream, a query can
be given as input; the query results are computed and cached as a
data stream.

The data streams definition and edit operations in DEVise are accessed from the
\menu{Tables} menu, by selecting the \menu{New Table} and \menu{Edit} options,
respectively. There are numerous data streams which accompany DEVise,
descriptions of which follow.

To use a data file, select the \menu{New Table} menu option. A dialog box
entitled ``Define Data Stream'' will appear. Selecting the topmost ``Select''
button will create a file selection dialog for picking a key.

\subsubsection{Unix File}

The data fields of a normal UNIX file data stream are separated by a standard
delimiter as specified in the schema. An ASCII file containing $sin(t)$ and
$cos(t)$ might contain the following records (or tuples):

\begin{verbatim}
# time   sin      cos
0.000000 1.000000 0.000000
0.017453 0.999848 0.017452
0.034906 0.999391 0.034898
0.052358 0.998630 0.052334
... 
\end{verbatim}

The output of any program that generates similar sequences of data can be
visualized in DEVise as a UNIXFILE stream. The schema defines the delimiter
(here a blank space) and comment character (here \#), and also assigns the
meaning (name) to the various record fields.

\subsubsection{WWW Data}

Any World Wide Web source can feed a WWW data stream. The URL (Uniform Resource
Locator) can use the \code{ftp} or \code{http} protocol. Data at the URL is
fetched from the WWW and cached locally. This method offers the advantage of
directly using the WWW data without additional effort. As for the UNIXFILE type,
the stream delimiter and comment characters must be specified in the schema. The
key is assigned by the user and must be a unique name within this stream type.

\subsubsection{SEQ Query}

A sequence query can be specified and executed in a SEQ database server. The
result of the query is cached on disk using the key value given. The key name
needs to be distinct for the SEQ data source type.

\subsubsection{CRSP Data}

CRSP (Center for Research in Security Prices) provides comprehensive stock price
data via two primary files, the NYSE/AMEX file and the Nasdaq file. The CRSP
data is available on tape. The information stored consists of company name,
distribution, share, listing and Nasdaq information. The NYSE/AMEX Daily file
contains data beginning July 2 1962 and tracks roughly 7,250 securities. The
Nasdaq file contains information for over 12,500 common stocks stored on the
Nasdaq stock market, since Dec 14, 1972.

\subsubsection{COMPUSTAT Data}

COMPUSTAT is a database of financial, statistical, and market information. It
provides more than 300 annual and 100 quarterly Income Statements, Balance
Sheets, Statement of Cash Flows, and supplemental data items on more than 7,500
publicly held companies. For most companies, annual and quarterly data is
available for a maximum of 20 years and 48 quarters. The data records consist of
about 350 fields.

\subsubsection{ISSM Data}

The ISSM (Institute for the Study of Security Markets) provides transaction data
for NYSE and AMEX listed securities and NASD securities. The data for NYSE and
AMEX are available from 1983 to the present, whereas for NASD it is available
from 1990 to the present. The data record consists of trades and quotes timed to
the tick. Ancillary information such as CUSIP numbers, firm names, shares
outstanding and dividend information are also provided. One of the most heavily
traded firms, IBM, had over 900,000 trades and quotes in 1992. (Also available
in tape form.)

\subsubsection{SQL Query}

This is similar to the SEQ query except that the SQL query is passed to a SQL
database system from which the results are cached. Again, the key must be a
unique name within the SQL domain.

\subsubsection{Command Output}

DEVise executes the (Unix) command specified in the Info/Command entry and
caches the output of the program. A key must be assigned to identify this data
stream from other Command Output data streams.

\subsubsection{Network Data}

DEVise caches data delivered over the network. A key must be assigned to
identify this data stream from other Command Output data streams.

%*******************************************************************************

\subsection{Physical Schema}

A schema file describes the format/layout of an input data stream. It conveys
the name, type, and range of attributes, characters that separate the attributes
in the file, characters that should be ignored while reading the file, and
optional attribute range information. More than one file type can be imported
into DEVise, each having its own schema file.

The physical schema describes the actual layout of the input stream with all its
attributes. Every data stream is associated with a physical schema, and the
schema is used by DEVise to extract data values from a data stream. To browse
the physical schema, use the \menu{Physical...} menu option in the \menu{Schema}
submenu of the \menu{Tables} menu.

For our example file, the schema file looks like:

\begin{verbatim}
type Sensor ascii
comment #
whitespace ' '
attr time double hi 1000 lo 0
attr sin double hi 1 lo -1
attr cos double hi 1 lo -1
\end{verbatim}

The first line names the file type: Sensor. All Sensor files have data stored in
the same format. The second line tells DEVise to ignore lines that start with
'\#'. The third line tells DEVise to treat spaces as whitespace in the input
stream and as a separator between field values. The remaining lines describe
attribute names, types, and ranges.

\subsubsection{Attribute Name}

An attribute name identifies a particular field of a record. In the schema
defined above, sin is the name of the first attribute. This name is used when
defining attributes of X and Y axes in views.

\subsubsection{Attribute Type}

An attribute may be one of the following five types: double, int, string, float,
or date. Each type has its ubiquitous meaning and representation, except date
which is the Unix time/data type. In the example above, sin is of type double.

\subsubsection{Attribute Value Range}

The attributes may be defined to have a range bounded by mimimum and maximum
values. When DEVise displays field values, the range values displayed are
calculated using the attribute ranges. For example, for an displayed attribute
with value range between 0 and 200, DEVise chooses to display exactly that
range. If the attribute's range in undefined, DEVise uses the range 0 to 100 by
default.

\subsubsection{Sorted vs Unsorted}

Records in a stream may be sorted on a field. The specification of a sort key
may be specified in the physical schema. DEVise uses this information to
optimize data access and display. Moreover, data display in interleaved by value
to achieve rapid partial display of data. This allows perusal an outline of the
plotted data as the views update, partcularly useful for ``dense'' data.

%*******************************************************************************

\subsection{Logical Schema}

A logical schema defines a logical view on a physical schema, insodoing
restricting the viewed range of an attribute in the schema.

In the COMPUSTAT database there are 350 attributes. A person interested in only
the current assets, income tax, and stock price can construct a logical group of
these attributes and use that group to define views. In this manner the user
need not repeatedly wade through display of the data for the entire physical
schema when a retricted subset of data is desired for visualization.

Logical schemata may consist of multiple groups. For example, under a Sales
logical schema, there may be groups for sales tax, sales revenue, etc. A group
may be nested: the top-level groups are those atop the hierarchy of group
attributes. Groups may be nested to any arbitrary level and may consist of any
number of attributes. For any physical schema, a default logical schema with the
same name is created having all attributes of the physical schema---essentially
one group consisting of all attributes.

Like physical schema, logical schema may be browsed under the \menu{Logical...}
menu option in the \menu{Schema} submenu of the \menu{Tables} menu.  A logical
schema may also be browsed using the Logical option under Schema.

\subsubsection{Name of Physical Schema}

Each logical schema is a view definition based on one physical schema. The
physical schema name in the logical schema associates the logical schema with
the particular physical schema. A default logical schema is created for every
physical schema that has the same name and attributes as the physical schema.

Creation or modification to a logical schema proceeds first by selection of an
appropriate physical schema, hence the name of the physical schema is implicitly
built in to the logical schema definition.

\subsubsection{Group Name}

Each logical schema may consist of many groups. A group name identifies a
particular group within a logical schema. When visualizing a data stream, the
user is automatically prompted to choose the proper group for mapping
definition. This group is the {\em top level group}.

The group name must be unique within the logical schema (and, hopefully,
comprehensible to the user). Any number of groups may be defined within this
group in a nested fashion.

\subsubsection{Attribute Name}

The attribute name within a group definition includes that field in that
group. The attribute names may appear within any number of sub-groups within a
logical group. The attribute name must be the same as defined in the physical
schema---it is the name that associates a the logical group attribute to the
actual physical one.

%*******************************************************************************

\subsection{Graphical Data}

Graphical data is the second intermediate stage in processing data records
before display. Raw data is parsed using the schema to create {\em TData}. On
the TData, the user selects the required display attributes such as shape,
color, etc.

The result is a {\em mapping} used to form another intermediate representation
called {\em GData} (Graphical Data). GData is the set of values used for actual
display of records on the screen, a graphical representation of TData. It
consists of the following attributes: X, Y, Z, size, color, pattern,
orientation, shape, and shape-specific attributes.

A mapping can be specified and edited using the visualization window
(Figure~\ref{screen7}). (Default mappings are automatically created when a new
data stream is imported.) The window lists the various attributes in the logical
schema of the relation. The user can choose attributes to be displayed as views,
as well as the color, layout mode, window name and title. The plot can be done
using a bar chart, image plot or scatter plot.

\scaleepspic[0.7\textwidth]{screen6.eps}{History Screen\label{screen7}}

A view so defined can be modified using the \menu{Edit Data} menu option in the
\menu{GraphData} menu. The mapping can be edited to change one or all of the
GData attributes. Mapping elements may be defined as constant or variable
attribute values, or a mathmetical expression using the attribute values
described next.

\subsubsection{Location: X, Y, and Z}

The X, Y and Z parameters specify the location of a record's visual
representation with respect to the coorinate axes.

\subsubsection{Size}

The size of the displayed point/object/glyph in a view is determined by the size
attribute. If a constant, the attribute is the enlargement factor of the
displayed object. If a variable, the size of the object displayed is dependent
of the value of some attribute in the TData.

For example, if a square represents data points and the X and Y axes are date
and stock closing prices, respectively, the square size can be proportional to
the stock's volume on each date.

Using an expression for the size attribute is even more powerful, as the
displayed object size is a function of several TData attributes.

\subsubsection{Color}

Like the size attribute, color can be a constant, variable, or expression. The
use of variable color also yields interesting results. For example, in a scatter
plot the color can vary with daily rainfall, using date and temperature on the
axes. Currently, a fixed palette is defined such that any expression or variable
value computed is mapped to one of 43 colors.

\subsubsection{Pattern}

The fill pattern may also be changed for displayed objects. [This feature is
not supported yet.]

\subsubsection{Orientation}

Orientation specifies the tilt or rotation of displayed objects, which is useful
to display some objects such as diamonds. Like other attributes, orientation can
be defined as a constant, variable or as an expression. [This feature is not
supported yet.]

\subsubsection{Shape}

Data points can be plotted as one of the following shapes:

\begin{itemize}
	\item Rectangle (Rect)
	\item Square (RectX)
	\item Bar
	\item Polygon
	\item Oval
	\item Vector
	\item Block
\end{itemize}

The shapes may be defined as constant or variable values (the default is the
bar). The square (RectX) gives a visual square shape, block represents a 3
dimensional block, and a polygon represents a dodecagon. The vector is a
directional one, with an arrow at the end. The bar is drawn from (X,0) to (X,Y).

\subsubsection{Shape Attributes}

The shapes have associated with them width, height and other values. Shape
attributes can be modified in the Edit Mapping window. The shape attributes
are summarized in Table~\ref{shapeattrs}.

\begin{table}[htbp]
\begin{center}
\caption{Shape Attributes\label{shapeattrs}}
\bigskip
\begin{tabular}{l|l|l|l|l}
Shape   & Meaning      & ShapeAttr0 & ShapeAttr1 & ShapeAttr2 \\
\hline
Rect    & Rectangle    & Width      & Height     & n/a \\
RectX   & Square       & Width      & n/a        & n/a \\
Bar     & Vertical Bar & Width      & n/a        & n/a \\
Polygon & Dodecagon    & Width      & Height     & n/a \\  
Oval    & Oval         & Width      & Height     & n/a \\
Vector  & Vector       & X delta    & Y delta    & n/a \\
Block   & Cuboid       & Width      & Height     & Depth
\end{tabular}
\end{center}
\end{table}

%*******************************************************************************

\subsection{Views and Windows}

After the creation of a mapping from TData to GData, association of an X
attribute to a Y attribute defines a {\em view}. A view is used to display GData
that fall within a specified range of attributes. Views can be manipulated in a
variety of ways to achieve visualization of important patterns in the data
stream.

Windows provide the screen real estate used by views to draw the GData. A window
is responsible for arranging views in its boundaries (the current possible
arrangements include tiled/automatic, vertical, and horizontal). Views can be
removed from a window to reduce clutter, or be moved to other windows for
comparison. Windows can be duplicated.

\subsubsection{Manipulating Views in Windows}

To arrange views in a window, the automatic mode selects an appropriate layout
using the dimensions of the window. For example, a horizontal arrangement is
chosen for a long, thin window. Views may also be arranged in a fixed,
user-specified way (horizontal, vertical, etc.). The number of views to be
stacked either horizontally or vertically may also be defined.

When changing a window, it may be desirable to force DEVise to postpone redraw
of complex views until all changes are complete, particularly when working with
a large data set. In the Control Panel, toggle the Display button to Layout
to block window refreshing. After the changes are complete, toggle the button
back to Display to restart continuous window refreshing.

\subsubsection{View Title}

The view title is independent of the graph displayed and can be modified. The
default view title is "$<X attrib name> VS <Y attrib name>$" (e.g. DATE VS
SALES. To edit the title of the current view, select the \menu{Title} menu
option in the \menu{View} menu. The title may also be deleted, in which case the
{\em current view indicator}, normally a rectangular boundary surrounding the
view, becomes a rectangular bar on the left edge. DEVise attaches no
significance between the title and data displayed in the view.

\subsubsection{View Axes}

In default, the X and Y axes of a view's graph are displayed with the minimum
and maximum range values shown for each dimension. The visibility of the X axis
may be toggled on/off by selecting the \menu{Toggle X axis} menu option in the
\menu{View} menu. This toggling may be performed on all views or the current
view only. The same is true of the Y axis.

\subsection{Visual Filtering and Display of GData}

Graphical data are displayed in many views in many windows. Queries on views
take the form of range/region selection using zooming and scrolling
mechanisms. For explicit control over view ranges, the Query Tool may be used
by selecting the Query button.

A view displays GData that fall within a {\em visual filter}. Currently there
are two types of views: Scatter and SortedX. The Scatter view is used to draw a
scattered plot. The SortedX view implements optimizations used to reduce the
time used to draw the GData if the X attribute is sorted. The view is basically
a graph of the values of the attributes chosen. Views may be manipulated in many
ways for effective visualization.

A visual filter defines a query over the graphical data attributes of the
GData. Our implementation supports range query over the X and Y GData
attributes. The visual filter is used to specify portions of GData to be viewed;
the (X,Y) region in a view is the visual filter for that view, since it obscures
the rest of the GData.

\subsubsection{History}

DEVise has a history buffer which maintains a list of previous visual filters
for a view. Each history entry records the minimum and maximum X and Y axis
values for all views. Using the history buffer, old filters can visited,
encouraging exploration with the knowledge that old visualizer states can be
restored.

To restore an old visualizer state, select the \menu{History} menu option in the
\menu{Help} menu. A history entry can be invoked by clicking on the desired
value in the history window. The history mechanism also provides for {\em
marking} of filters, which forces DEVise to retain them in the buffer. If the
buffer becomes full, unmarked history entries are discarded.

\subsubsection{Visual Assistants}

Two features allow establishment of a persistent connection between views: the
{\em visual cursor} and the {\em visual link}.

Cursors are visual bands that maps and displays the (X,Y) boundary of one view
within another, like a zoom window. Cursors provide visualization of a single set of data over multiple, different ranges and scales.

A cursor connects a source view and a destination view. The source view is
where the cursor fetches information about the current view (X,Y) axis
boundaries, used to display the cursor in the destination view. If the region displayed in the source view changes by scrolling or zooming, the cursor in the destination view moves. Likewise, the cursor can be moved by clicking in the destination view; the cursor will be centered on the click point, and the region displayed in the source view will be updated.

Visual links represent a connection established between two views using one or
more of the graphical attributes: (X,Y) axis values, color, size, etc. By
linking two views together, changing a linked parameter in one view causes the
corresponding changed in the linked view. For example, if two views are linked
together by their X axis value, scrolling along the X axis in one view
automatically scrolls the other view along the X axis.

Visual links can be named for reference. There can be any number of links at a
time between views.

%*******************************************************************************

\subsection{Statistics}

DEVise provides mean, minimum, maximum, and confidence interval calculation and
display as well as Kiviat graphs.

The mean, minimum, and maximum are displayed as lines in the views, calculated
over the range of values displayed in the particular view, not over the range of
the entire data set.

Confidence intervals are visualized as transparent bands around the mean line. A
confidence interval can be any of 85\%, 90\% or 95\%, or none (the default).

Kiviat graphs provide a visual method of analyzing statistical measures over a
set of views in a window. A graph can be used to analyze the balance in the
values of the various views. Kiviat graphs can display the mean, minimum,
maximum, or count (number) of data values. A Kiviat graph changes with changes
in the visual filters of the corresponding views.

%*******************************************************************************

\subsection{Sessions and Templates}

A session is a collection of data streams, schemata, mappings, views, and
windows. Furthermore, it maintains information including links, cursors, and
graphical and statistical preferences for each view. Think of a session as the
state of a visual query.

A template is a session without associated data stream(s), consisting of all
mappings which can be applied to any data stream conforming to the session's
schema. Templates provide a way to view data of different streams under the same
mappings and state.

While working in DEVise the current state can be stored as a session or a
template. When opening a template, the user must supply a data stream conforming
to the schema of that template.

%*******************************************************************************
% Data Streams
%*******************************************************************************

\section{Data Streams}

Data streams are simply sequences of records. A data stream can come from any
source such as the World Wide Web, COMPUSTAT, tape, etc. Numerous parameters and
functions characterize data streams.

%*******************************************************************************

\subsection{Data Stream Catalog}

I DON'T UNDERSTAND THIS STUFF.

A data stream catalog contains the name of the data streams defines, their
associated physical schema and the caching information.

A data stream catalog can be opened using the Import option under the Visualize
menu. The catalog can be browsed and new data streams can be added and existing
ones modified. The data is read from the source and cached on to the disk. This
is indicated by the word Cached in the catalog. A data stream cannot be
visualized unless it is cached on to the hard disk.

The catalog provides the following options.
\begin{description}
\item[Define] This option allows the adding of new data streams. The
new data stream may be manually configured or automatically found and
added.
\item[Stream] This menu provides options to edit, copy and delete data
streams. The edit option brings up the Data stream definition window
to modify the stream parameters.
\item[Display] This is used to limit the number of streams listed in
the catalog and also provides options for sorting the streams.
\item[Follow to] This is used to {\em follow} from a data stream for
a company in a particular schema to the same company's data in another
schema. e.g. to follow from a company's data stream in COMPUSTAT to
that in CRSP.
\item[Help] This provides some on-line help for using the catalog.
\end{description}

\subsection{Data Stream Definition}

The Define Data Stream dialog box is used to create or modify a data stream. To
create/edit a data stream, select the \menu{New Table} menu option in the
\menu{Tables} menu. Within the dialog box, select the following: type of source
(e.g. COMPUSTAT), the logical schema file associated with that stream, a key to
uniquely identify a stream from others, a command (if needed) and cache
parameters. These are discussed below.

\subsubsection{Display Name}

The display name is the name of the stream to be defined. This name that will be
listed in the catalog as the data stream name. This name must be unique within
the catalog.

\subsubsection{Source}

The source option indicates the source from which data is to be retrieved. There
are 9 possible sources namely, COMPUSTAT, CRSP, COMMAND, NETWORK, ISSM, WWW,
SQL, SEQ and UNIXFILE. The source determines where DEVise has to search for, in
order to get the data for caching. The COMPUSTAT, CRSP and ISSM, by default are
tape oriented. The WWW needs a URL to extract the data. The SEQ and SQL need
queries to evaluate and the COMMAND option needs a valid UNIX command.

\subsubsection{Key}

The key is a name that is used to uniquely identify the stream from the rest
when storing. Usually the stream definitions are stored as the source name
followed by the key name. Thus keys selected should be unique.

\begin{enumerate}
\item
The keys for the UNIX source must be a valid pathname in UNIX.
e.g. /p/devise/dat/UNIX\_data1

\item
The keys for the WWW, NETWORK, COMMAND, SQL and SEQ must be a unique and valid
name chosen by the user. The actual filename is obtained by concatenating the
source with the user chosen name. Thus the key chosen needs to be unique only
for that domain. e.g. there can be a key ATLAS with source SQL and another ATLAS
with source COMPUSTAT and these are considered different.

\item
For the ISSM database, the key chosen must be the 3 alphabet code for the
company. DEVise provides an automatic listing of the 3 digit Alpha codes for all
the companies through the Select option.
\end{enumerate}

\subsubsection{Schema}

The schema option is used to associate the correct schema for the stream.
Remember that DEVise parses and understands the stream with the help of the
schema. The schema is not actually used to retrieve the data, but is used when
caching it on to the TData form. The schema must be previously defined and
stored in a file, whose name is mentioned here. It is important to note that it
is the logical schema that is stored here. The schema name needs to be a full
file pathname. It can also be browsed and selected through the Select option
provided.

\subsubsection{Evaluation Factor}

The evaluation factor specifies the relative time at which the data stream is to
be extracted from the specified source and cached on to the local disk. A
evaluation factor of 100 would force DEVise to get the data from the source, as
soon as the stream is defined and cache it. A low value of the evaluation factor
delays this procedure and a value of 0 would postpone this to the time of actual
visualization of the stream.

\subsubsection{Cache Priority}

The cache priority indicates the priority to be given to this stream in the disk
cache. When a large number of items are cached, the disk would run out of
space. At this point DEVise uses the cache priority specified and removes the
least priority item from the disk cache. As a default it is specified as 50\%.

\subsubsection{Info/Command}

The command window is used to take in a command for the SQL, SEQ and other data
sources. For the case of the WWW source the URI (Uniform Resource Identifier)
needs to be specified. (e.g. \code{ftp://...}) In the case of SEQ query the
query is sent to the SEQ program which evaluates it and the tuples returned are
cached. The command could be any UNIX command that can supply a stream of tuples
(e.g. \code{awk...}).

\subsection{Manipulating the Catalog}

This section deals with manipulating the catalog, part of which has already been
discussed above.

\subsubsection{Create}

The data stream creation can be done by using the Visualize/Import/Define/New
option.  The data stream to be created is associated with the following.

\begin{description}
\item[Display name] This defines the name of the data stream
\item[Data source] The data source whether COMPUSTAT, CRSP etc. needs
to be defined.
\item[Key] As already explained, the key name is to be given to
uniquely identify the stream. In case of UNIXFILE type it is the full
pathname, in case of Business Databases, it is the unique numeric
identifier and in the case of the queries, it is a user chosen
name. The names must be unique within their domain.
\item[Schema file] The logical schema file needs to be specified.
\item[Evaluation factor] This specifies the time at which the stream
must be cached. A high value results in caching immediately after the
data stream is defined and a lower one postpones it to the
visualization time.
\item[Cache priority] This specifies the priority of storage of the
stream on the disk cache.
\item[Info/Command] This is useful for the query type of sources. The
query or command (say \code{http://...}) etc. is needed to be
specified in this.
\end{description}

%% Major problems here ... The edit did not check the actual schema even after 
%% uncaching it.. Should we uncache it?

\subsubsection{Edit}

The editing of a stream can be done by using the Stream/Edit option in the Data
stream window. The edit is similar to the creation of a new stream and hence is
left undiscussed.

\subsubsection{Copy}

This is used to copy an existing data stream definition on to a different data
stream name. The same edit window is displayed, but with the parameters of the
data stream to be copied. A unique name is all that is to be given to complete
the copy phase.(Any other parameter can also be changed) This is useful for
defining a lot of streams having same characteristic, like schema etc. This
option is available under Stream/Copy in the data stream window.

\subsubsection{Delete}

This is used to delete an existing data stream. The stream is selected first and
then deleted. Available under Stream/Delete in the stream window.

%*******************************************************************************
% Schemata
%*******************************************************************************

\section{Schemata}

%*******************************************************************************

\subsection{User and Superuser Modes}

DEVise can be brought up in one of two modes---{\em user} or {\em superuser}.
The mode used is set by the value of the \variable{UserMode} variable defined in
the \filename{.rc} file read by DEVise. A value of 1 indicates user mode, 0
indicates superuser mode. The mode determines the capabilities and presentation
of the schema browser.

To browse schemata, select the \menu{Schema} menu in the \menu{Table} menu, then
select the \menu{Browse...} menu option in the former menu.

\subsubsection{User Mode}

In user mode, the browser allows attribute selection in any predefined schema.
Furthermore, views defined on these schemas (through groupings) can be seen and
modified. New views can also be defined.

When the schema browser is invoked in the user mode, a list of all schemata
defined in the system is presented. By selecting the Attributes button in the
browser window, the attributes associated with a schema can be viewed.
Furthermore, the set of views defined on these attributes can be seen as a list
by selecting the Views button.

New and existing views can be modified within the group browser.  The user can
either open an existing view for modification or specify a name to create a new
view. In either case, a group browser appears. The user can modify/define groups
and their contents in terms of base attributes or subgroups. Finally, if a new
view is saved, it becomes a part of the list of views on that particular schema.

% This stuff is NOT clear at all, and the UI is incomprehensible.

%*******************************************************************************

\subsection{Schema Browsing in Superuser Mode}

In superuser mode, the browser provides complete control over all schemata.
Existing schemata can be opened to display the types and value range of defined
attributes, and can be edited by inserting new attributes, modifying range
values, etc. Entirely new schemata can also be defined, an operation
necessitated by a new data source (and new data layout) to visualized in DEVise.

When the schema browser is invoked in superuser mode, a list of all current
system schemata is displayed. One of these schemata can be chosen, or a new one
created. In any case, the opened schema is displayed as a list of all possible
attributes.

The type and range limits of all attributes are displayed and can be modified.
(Range limits are used for automatic display range selection during
visualization.)

When a new schema is saved, it becomes an item in the schemata list and a
default view is defined for it. The view consists of all schema attributes with
no group structure. The default view may be modified and new views created at
any time.

%*******************************************************************************
% Mappings
%*******************************************************************************

\section{Mappings}

A {\em mapping} is an association of data sources in different schemata. For
example, associations can be established between companies in \term{COMPUSTAT}
and \term{CRSP} databases. The user interface to define the mappings is as
follows. First, the two schemas to be mapped and the names of files used for
storing the following are specified:

\begin{itemize}
	\item Mapping table---the list of associations
	\item Unresolved data sources, first schema 
	\item Unresolved data sources, second schema
\end{itemize}

File names must be specified for future manual resolution of unresolved and
conflicting sources. The list of key attributes of the two schemas are
displayed; the user picks pairs to match.  DEVise automatically generates all
mappings based on these user-specified rules. All unresolved sources are dumped
into separate files which can be browsed manually.

%*******************************************************************************

\subsection{Resolving Conflicts and Unresolved Sources}

After automatic generation of the mapping table, there are two files containing
unresolved sources of the two schemas, plus the mapping file. By specifying
these three file names, the user can bring up the list of all unresolved sources
and manually define mappings based on knowledge of the sources.

%*******************************************************************************

\subsection{Using Mappings}

Mappings generated automatically or manually are used to switch
between equivalent data sources in different schemata easily. While browsing
through a list of data sources in a particular schema, say \variable{X}, the
equivalent data source in a different schema, say \variable{Y} can be loaded
with a mouse click if the mapping between \variable{X} and \variable{Y} has been
defined.

%*******************************************************************************
% Visualization
%*******************************************************************************

\section{Visualization}

This section gives a detailed account of visualization creation. A tutorial
giving step-by-step instructions for using a visualization is below. Before
visualization, determine which data streams are available and which attributes
are needed to visualize. For pedagogical purposes, operation on a demo data
stream will be described.

%*******************************************************************************

\subsection{Visualization Characteristics}

Before creating a visualization, open a data stream to be visualized. In the
\menu{Tables} menu, select the \menu{Edit} menu option. From the list in the
resulting dialog box, select the ``IBM Stock (MIT Database)'' data stream. The
``Define Visualization'' window will appear. The visualization parameters can be
edited from this window.

\subsubsection{Window Name}

By default, the name of the window in which all views to be defined on this
stream will be contained is the name of the stream itself. Once specified, the
window name cannot be modified.

\subsubsection{Window Layout}

The window layout determines the arrangement of views within a window. A
vertical layout arranges views on top of each other, a horizontal layout
arranges views side by side, and a tiled layout arranges views in a neat
square. Tiled layouts are promoted to a vertical layout if the number
of views is odd.

The shape of the window changes with the layout. For a vertical layout the
window is long, for a horizontal layout wide. A layout can be changed at any
time by selecting a view in the window to alter, then choosing the \menu{Layout}
menu option in the \menu{Window} menu.

\subsubsection{X Attribute}

The X attribute specifies the schema field to be used as the X axis attribute in
views of the visualization. The popup menu lists all attributes available in the
schema. When importing a file with many logical groups, the groups need to be
selected prior to using the Define visualization window (DEVise automatically
prompts for this). Only the attributes present in the top level groups selected
are displayed. All views in a visualization have the same X attribute.

\subsubsection{Y Attributes}

The matrix of buttons at the bottom of the ``Define Visualization'' window can
be toggled on and off to select/deselect Y axis attributes. For each attribute
so selected, a view will be created with that attribute as its Y dimension. (For
group attributes, buttons are popup menus.)

\subsubsection{View Background Color}

The background color popup menu allows choice of the background color desired
for the views of a visualization. A color chart is presented for selection. The
background color of a view is retained, even when migrated between windows.

\subsubsection{View Title}

The view title option indicates whether the title is to be displayed
for views or not. The default title is {\em X attribute Vs Y attribute}.
The title can be changed at any time using the \menu{Title} menu option in
the \menu{View} menu.

\subsubsection{Link X}

To link all views along their X axis, select the Link X button. When views are
linked in a dimension, scrolling in that dimension changes the visual filter
(viewing area) of the views together. (The name of the X axis link is {\em X
attrib Link}; select this name when removing a view from a link or adding a new
one to a link.)

\subsubsection{X and Y Axes}

The X and Y axis toggle buttons determine whether axis lines and labels are
shown or hidden. The X and Y axis display state can be toggled via the View menu
after creation of the visualization..

%*******************************************************************************

\subsection{Visual Filters}

Visual filters are regions of X and Y values displayed in a view. A filter
extracts visible data from the GData values and displays it on the screen.

\subsubsection{Change Visual Filter}

Visual filters may be edited in the Query Tool window, accessed using the Query
button in the Control Panel. The X and Y ranges of the current view are
displayed in the window. Changing the high and low range values changes the
region of visible data in the view. To apply a change, select the ``Execute''
button. To undo the most recent change, select the ``Undo Edit'' button.

\subsubsection{Filter History}

DEVise maintains a filter history, a list of X and Y ranges in the reverse order
in which they were visited. To view a filter's history, select the ``History'' button in the Query Tool window.

Places of interest can be distinguished by clicking in the ``Mark'' column of
the history window. Entries are marked with an asterisk.  Because a history has
finite size, old history entries are replaced whenever the size of the list
exceeds its maximum. (Marked entries are preferentially retained.)

The ``Back One'' option allows to go back one filter in the history.

%*******************************************************************************

\subsection{Graphical Data}

There are several ways to manipulate mappings from TData to GData. A mapping
from TData and GData is partially specified by the options chosen at the time
data is imported.

\subsubsection{Editing a Mapping}

After importing data or opening a session, the mapping can be edited by
selecting the ``Mapping'' button in the Control Panel. The ``Edit Mapping''
window will appear, in which the following attributes may be edited:

\begin{description}
	\item[X] The attribute mapped to the X axis. A constant value gives the
		value of the Y axis for a particular X.
	\item[Y] The attribute mapped on to the Y axis.
	\item[Color] View foreground color. A constant value paints the graph with
		the constant value selected, a variable value maps the attribute value
		to one of the base colors (using modulus {\em (\#colors)} An expression
		works similarly.
	\item[Orientation] Rotates the objects about their centers. (Giving a
		proper orientation for a square can make it appear as a diamond.)
	\item[Size] Size of the object. The default value specified using the
		ShapeAttr is 1.
	\item[Pattern] The filling style for objects.
	\item[Shape] A shape may be any of Rect, RectX, Oval, Polygon, Bar, Block
		or Vector. The variable and expression options map the value
		given to one of these (using modulus {\em (\#shapes)}
	\item[ShapeAttrs] Dimensions of the displayed objects, particular to the
		shape.
\end{description}

%*******************************************************************************

\subsection{View Properties}

Views may be manipulated in a variety of ways.

\subsubsection{Swap Views}

To swap two views, click them one after the other then select the \menu {Swap
Views} menu option in the \menu{Window} menu. The two views must be in the same
window.

\subsubsection{Move View}

To move a view to a between windows, click on the view to make it current, then
select the \menu{Move View} menu option in the \menu{Window} menu. In the
resulting list of window names, select the desired window. The view may also be
moved to a new window by selecting the ``New'' button beneath the list. In the
latter case, another window in created, prompting the user to enter window
parameters suah as size.

\subsubsection{Remove a View}

To remove a view, select it then use the \menu{Remove View} menu option in the
\menu{Window} menu. The view is removed visually from the window, but not from
DEVise's memory.

\subsubsection{Bring a View Back}

To being back a view to a window, use the \menu{Bring VIew Back} menu option in
the \menu{Window} menu. Select a view from the resulting list of names and a
window from the resulting list of window names. Because all views are listed
(not just previously removed ones), this option allows copying of views as well.

\subsubsection{Destroy View}

To permanently destroy a view (remove it from the list of views for the
session), select the \menu{Destroy View} option in the \menu{View} menu. A view
so destroyed cannot be brought back.

\subsubsection{Switch TData}

When visualizing a particular data stream, it may be useful to see the data
stream in the same format as that of the current session. The \menu{Switch
TData} menu option in the \menu{GraphData} menu allows retention of filters and
view layouts while changing the data displayed. The switched data stream must
conform to the same logical schema as that of the current view.

A view must be selected before switching its TData. To switch, select an open
data stream from the resulting table (other data streams may be selected by
picking the ``Other'' button).

\subsubsection{Title}

The \menu{Title} menu option in the \menu{View} menu allows the view title to be
edited or cleared. Clearing the title removes the title from the view completely
and changes the current view indicator from a rectangle surrounding the view to
a bar to the left of the view.

\subsubsection{Axes}

The \menu{Toggle X Axis} and \menu{Toggle Y Axis} menu options in the
\menu{View} menu show or hide a views axes and high and low axis labels.

%*******************************************************************************

\subsection{Window Properties}

Like views, windows can be manipulated in a variety of ways.

\subsubsection{Change View Layout}

The layout of views in a window may be changed using the \menu{Layout} menu
option in the \menu{View} menu. Three options are presented: automatic,
vertical, and horizontal.

Horizontal mode forces DEVise to display the views side by side, vertical mode
makes displays the views one on top of each other. The automatic mode adjusts
the layout depending on the dimensions of the window. If the window is longer in
the horizontal direction than the vertical direction, DEVise uses
horizontal layout mode, otherwise vertical layout mode.

The Width and Height parameters are the number of views displayed horizontally
or vertically, respectively. For example, two views are oriented vertically when
the height is 2. The Height determines the number of views to be displayed atop
one another in the horizontal layout mode.

\subsubsection{Duplicate Window}

Windows can be duplicated, views and all, using the \menu{Duplicate} menu option
in the \menu{Window} menu. DUplication is useful for creating cursors and
switching TData options. The new window name is the name of the old window with
``Auto \#'' prepended.

\subsubsection{Remove Window}

To remove a window first remove all views in it, then select the \menu{Remove}
menu option in the \menu{Window} menu. (A window can also be removed by
double-clicking in its top left corner button, although DEVise will complain.)

%*******************************************************************************

\subsection{Visual Link}

Visual links associate two or more views through a common characteristic such as
the X axis range, Y axis range, color, etc. Changing the visual filter in one
view change the filter in the linked views.

\subsubsection{Create Link}

To create a link whle defining a visualization, select the ``Link X'' button in
the Data Visualization window. Doing so creates a link between all views defined
on that data stream.

New links can be created between existing views. Using the \menu{Link} menu
option in the \menu{Links} submenu of the \menu{View} menu, open the Select Link
window listing all currently available links. (Selecting the ``Info'' button in
the latter window lists all views connection by the chosen link.

To create a new link, select the ``New'' button then establish the link in the
resulting window by picking one of the possible link attributes: X, Y, color,
size, pattern, orientation, and shape. A link name must be provided in the field
provided to add to the list of existing links.

\subsubsection{Unlink}

To remove a link, use the \menu{Unlink} menu option in the \menu{Links} submenu
of the \menu{View} menu. Select a link name from the reulting list of link names
to remove it.

%*******************************************************************************

\subsection{Visual Cursor}

% This section is presently a little cursory %-)

Visual cursors are useful tools for superimposing the visual filter of a source
view over a destination view.

To create a cursor, select the \menu{Create} menu option in the \menu{Cursor}
submenu in the \menu{View} menu. In the resulting window, enter a name for the
cursor and specify whether or not the cursor should be constrained in the X
and/or Y dimensions.

To delete a cursor, select the \menu{Delete} menu option in the \menu{Cursor}
submenu in the \menu{View} menu. 

The source and destination views of a cursor may be changed after creation of
the cursor using the \menu{Set Source} and \menu{Set Destination} menu options
in the \menu{Cursor} submenu in the \menu{View} menu.

%*******************************************************************************
% Statistics
%*******************************************************************************

\section{Statistics}

DEVise provides two statistics facilities for data analysis. First, graphical
representation of statistical measures can be added to views, e.g. a line drawn
at the across the view at the mean. Second, Kiviat graphs may be generated from
data collected across several views.

%*******************************************************************************

\subsection{Simple Measures}

The statistics of displayed data can be visualized by superimposing the mean
(average), minimum and maximum on a view of that data. Similarly, confidence
intervals can be visualized by coloring a view with transparent bands around the
mean line at 85\%, 90\%, and 95\% confidence. All statistical values displayed
in a view are calculated for the range of visible X values only, not for the
entire data set.

Statistics may be generated and displayed for the current view or all views as
desired. To access the statistics options, select the \menu{Statistics} menu
option in the \menu{View} menu in the Control Panel. The dialog box which
appears contains toggle buttons for showing/hiding the mean, minimum, maximum,
and ``line'' (simple linear fit) in the view(s). A confidence interval around
the mean may be displayed by selecting one of the ``CI'' radio buttons.

The various statistics options may be disabled in the same manner as they are
enabled. Each view can have different displayed statistical information by
making alterations in the dialog box for successive current views.

%*******************************************************************************

\subsection{Kiviat Graphs}

A Kiviat graph is a powerful method for displaying the collective information of
data displayed in several views. For example, a Kiviat graph used in the context
of performance data can be used to see how well a system is balanced. The Kiviat
graph for the mean, maximum, minimum or count of displayed data can be created.

As the region of displayed data changes (by scrolling or zooming), the Kiviat
graph tracks and adjusts. (Informally, a link exists between data views and a
Kiviat graph based on them). The actual values along the various axes of the
Kiviat graph can be seen by clicking the middle mouse button while in the
graph. For successive displays a single Kiviat graph window may be used or new
ones generated as needed.

The Kiviat graph options exist as the (tear-off) \menu{Kiviat Graph} submenu of
the \menu{View} menu. Either a new graph can be generated or the current one
reused (if one exists). The graph is created/updated by selecting one of the
four stastical measure then selecting the \menu{Apply...} menu option.

Disable Kiviat graphs by closing the window in which they are displayed (top
left corner button).

%*******************************************************************************
% General Operation
%*******************************************************************************

\section{General Operation}

DEVise operations are performed on documents called {\em sessions}. A session is
a set of data streams with GData mappings into views. After creation of a
session, the user imports data, then explores and edits mappings by manipulation
of views, window styles, etc.

When saved, a session file can be used to recreate an entire visualization at a
later time, including the data streams visualized. To save a visualization
without its associated data streams, a {\em template} file is created. Templates
allow visualization of multiple sets of data streams while eliminating tedious,
redundant construction of views and mappings by the user.

The operations on sessions and templates are described next followed by several
general operations.

%*******************************************************************************

\subsection{Session Operations (Open, Save, Save As, Close)}

Sessions are the fundamental document type in DEVise, and as such may be opened,
saved, and closed. Session operations may be selected in the \menu{Session} menu
in the Control Panel. All operations perform as expected: Save overwrites the
old version, filenames are prompted as needed, etc.

%*******************************************************************************

\subsection{Template Operations (Open As Template, Save As Template)}

DEVise includes all data streams associated with a session when performing
session file operations. A session may be saved sans data streams by creating a
template. When opened, a template may be joined with data streams of the same
schema to create a new visualization. Although the data presented is dependent
on the associated streams, the environment---mappings, views, windows. etc.---is
that contained in the template.

Templates may be saved and opened using the \menu{Session} menu in the Control
Panel. A session file may be opened as a template, in which case DEVise will
prompt for data streams to associate with the template. The number and schemata
of data streams chosen must match those of the original session.

%*******************************************************************************

\subsection{Printing and Exporting}

Data and window images may be printed or exported to different print and file
formats. Print and export operations are accessible from the \menu{Session} menu
in the Control Panel. Upon selecting the print option, a dialog box will present
printing options with a choice of whether to print all windows or the current
selected window only.

A window image may be saved as a GIF, postscript, or eps file. In the print
dialog box, select the ``To File'' button and one of the three file formats from
the popup menu. 

%*******************************************************************************

\subsection{Quitting DEVise}

To exit DEVise, use the \menu{Quit} menu option in the \menu{Session} menu.
DEVise does not confirm whether to save any open, modified sessions. It confirms
only that quitting is desired---be sure all sessions desired for retention have
been saved before quitting.

%*******************************************************************************
% Tutorial
%*******************************************************************************

\section{Tutorial}

The tutorial is intended to give you a feel for using the basic functionality of
DEVise. The tutorial consists of the following steps:

\begin{enumerate}
	\item Starting DEVise: The Control Panel
	\item Opening a Session
	\item Visualization: Session Windows and Views
	\item Links and Cursors
	\item View Manipulation
	\item View Statistics
	\item General Session Operations
\end{enumerate}

%*******************************************************************************

\subsection{Starting DEVise: The Control Panel}

Start DEVise from the command line by typing \filename{devise} from the
\filename{run} directory. A window entitled {\em DEVise} will appear. This is
the {\em Control Panel}, the nerve center of DEVise (Figure~\ref{screen1}). The
Control Panel contains a six-menu menubar and four buttons, from which most
features of DEVise may be accessed.

\scaleepspic[0.6\textwidth]{screen1.eps}{Control Panel\label{screen1}}

%*******************************************************************************

\subsection{Opening a session}

The ``documents'' in DEVise are called {\em sessions}. For now, think of a
session as a collection of windows or views which display data in a variety of
formats from one or more perspectives.

To work on an example session, select the Session button in the Control Panel
with the mouse (or Alt-S from the keyboard). From the resulting Session menu,
select the Open option. A dialog window entitled {\em File select box} will
appear, displaying session files present in the current working directory
(Figure~\ref{screen2}). (Session files are distinguished by the \filename{tk}
extension.)

\scaleepspic[0.6\textwidth]{screen2.eps}{File Select Window\label{screen2}}

For this tutorial, double-click the \filename{demo/} directory. The session
files in this subdirectory will be displayed in the {\em File select box}. From
the list of files, select \filename{demo.tk} then the OK button. (Note: Under
most selection opportunities in DEVise, double-clicking the option performs the
same function as clicking the option then selecting the OK button.)

Once opened, the \filename{demo.tk} session is presented in three windows.
These windows contain the previously constructed {\em visualization} of the
session.

%*******************************************************************************

\subsection{Visualization: Session Windows}

The three session windows entitled {\em Global view}, {\em Trade Price} and {\em
Trade Volume} appear on the screen when the session is opened
(Figure~\ref{screen3}). Like other X-windows windows, each may be moved,
resized, closed, etc. Alteration of a window will cause the corresponding
appropriate adjustment(s) to the window's content.

\scaleepspic[\textwidth]{screen3.eps}{Data Display Windows\label{screen3}}

The data for the tutorial session are the trade price and volume of several
company stocks over a 30-month period. Inside the three session windows, the
nine graphs are the {\em views} for the session. Each view displays a subset of
the session data.

All view-oriented actions initiated from the Control Panel (or elsewhere)
operate on the {\em current} view. The current view is distinguished from other
views by the presence of a vertical highlighting bar to the left of the view or
a rectangle surrounding the view. A view can be made {\em current} by selecting
it (clicking the mouse button while the cursor is within the view boundary).

Zooming and scrolling are two operations which may be performed on a selected
view. First, make the IBM price view current by selecting the IBM Price view in
the Trade Price window. Second, press (or press and hold down) the key
corresponding to the desired operation as listed below:

\begin{itemize}
	\item 4 and 6: scroll left and right
	\item 8 and 2: scroll up and down
	\item 7 and 9: zoom in and out (X-axis)
	\item 1 and 3: zoom in and out (Y-axis)
\end{itemize}

For example, pressing the 6 key followed by the 1 key will adjust the IBM Price
view to display more recent prices valued between 30 and 70. (Note: The number
lock must be on for the numeric keypad to operate correctly.)

Because the X and Y ranges of data are bounded, a combination of zoom and scroll
is often required to focus on a particular region. This is particularly true for
some charts and graphs which display only positive-valued data. For example, the
Price and Volume bar charts displayed in the current session contain no
negative-valued information. To zoom in on positive values of Y only, combine
scrolling up/down and zooming in/out. For instance, to zoom in in the Y
dimension, press the 1 key followed by the 8 key.  To zoom out in the Y
dimension, press the 2 key followed by the 3 key.

While scrolling and zooming, the other eight views also change---the other three
Price and four Volume views scroll, and the colored block in the Global view
window moves. Such interdependicies between different views of the same data are
called {\em Links} and {\em Cursors}.

%*******************************************************************************

\subsection{Links and Cursors}

In the \filename{demo.tk} session, all the views except the global view are {\em
linked} together on their X-axis. What this means is all views display data
within the same range of X values. Thus the stock prices and volumes are
displayed for all four companies during the same time period (using trading
dates as the X values).

Two or more views may be linked along the X axis, the Y axis, or on a color,
shape, etc. Linking helps to you to synchronize browsing of multiple views, over
a fixed range of an attribute.

The colored rectangle in the Global view (window) is a {\em cursor}. A cursor
visually encompasses the rectangular region of data displayed in another
view. The X and Y ranges in the eight Price and Volume views are the same as the
X and Y range of the cursor in the Global view. Think of a cursor as a zoom box:
the part of a view which has been selected and ``zoomed in on'' in a different
view(s).

The connection between views established with a cursor is maintained as the
views are manipulated. For example, when you zoom in on the X axis (using keypad
7) in the IBM Price view, the cursor becomes smaller (as it encompasses a
smaller range of X values).

The cursor may also be moved by clicking under the X axis in the Global view
window. The cursor is repositioned such that it is centered at the point at
which the mouse was clicked. Because the views are connected via the cursor, the
displayed range of data changes in the other eight views whenever the cursor is
moved. Thus interesting patterns on the global view can be analyzed by moving
the cursor to that point and zooming in on the view. Similarly, the cursor is
enlarged as the ranges in the Trade Price or Trade Volume is changed using the
keypad scroll and zoom operations.

%*******************************************************************************

\subsection{Manipulating Views}

In addition to scroll and zoom, views may be manipulated in many ways from the
Control Panel. In the Control Panel, the \menu{Window} menu contains options for
moving views and hiding/showing views. The \menu{View} menu contains options
for adjusting view appearance.

\subsubsection{Moving Views}

To swap the position of two views in a single window, select the views to
swap, in sequence, then select the \menu{Swap Views} menu option in the
\menu{Window} menu.

To move a view to a different window, select the view to move, then select the
\menu{Move View} menu option. A dialog box will appear containing a list of
possible destination windows. Select the desired destination window---or the new
button to create a new window---followed by the OK button to complete the view
migration.

\subsubsection{Hiding/Showing Views}

An unwanted view can be discarded from a window using the \menu{Remove View}
menu option. For example, select the IBM Price view in the Trade Price
window. Use the \menu{Remove View} menu option to remove the view. The view will
disappear from the window; however, DEVise maintains memory of the view and its
contents.

To bring back a view, select the \menu{Bring View Back} menu option. A window
containing a list of {\em all} views in the session will appear. Select the
``Close vs Date4'' entry followed by the OK button. Another dialog box will
appear requesting a location (window) for the returned view. Select the ``Trade
Price'' entry to restore the IBM Price view to its former location.

\subsubsection{View Appearance}

Like typical graphing applications, the features of a view's appearance are
adjustable. The X and Y axis label values may be made visible or invisible;
the view title may be edited; and the background color may be selected.

To toggle the presence of X labels in a view, select the \menu{Toggle X Axis}
menu option in the \menu{View} menu. Of the two options available, choose to
toggle the X axis presence for all views, or for the currently selected view
only. Likewise, the presence of Y axis labels requires selection of the
\menu{Toggle y Axis} menu option.

\subsubsection{View Title and Colors}

The \menu{Title} menu option in the \menu{View} menu creates a dialog box for
editing the title of the current view. To delete the title, select the delete
button. In this special case, the current view indicator is a bar to the left of
the view, in contrast to the usual rectangle surrounding the view.

The \menu{Foreground...} and \menu{Background...} menu options in the
\menu{View} menu allow customization of foreground (pen) and background (field)
colors displayed in the current view.

The other, more advanced \menu{View} menu options are described later.

%*******************************************************************************

\subsection{View Analysis Tools}

Two mechanisms to assist in view analysis may be invoked via options in
the \menu{View} menu: Statistics and Kiviat Graph.

\subsubsection{View Statistics}

You may be curious about the presence of a line across each view. This is the
{\em statistical mean line} for the view over the visible X range. To set or
reset the line, select the \menu{Statistics} menu option in the \menu{View}
menu. A dialog box will appear with several options for display of common
statistical values, including the minimum, maximum, and mean Y values of
displayed data (Figure~\ref{screen4}). The ``line'' option fits a linear curve
to the displayed data.  In addition, the region of 85\%, 90\%, or 95\%
confidence in a view may be highlighted. To create the desired statistical
indicators in the current view or in all views, select the ``Apply Current'' or
``Apply All'' buttons, respectively.

\scaleepspic[0.4\textwidth]{screen4.eps}{Statistics\label{screen4}}

\subsubsection{Kiviat Graph}
	
Use the Kiviat graph tool to analyze the views in a window. Kiviat graph options
are accessible from the \menu{Kiviat Graph} submenu of the \menu{View} menu.
(The submenu may be torn off for repeated use.) The graph can display the result
for mean, maximum, minimum, or count as selected from the menu. When the
\menu{Apply} menu option is selected, the appropriate Kiviat graph is generated
and displayed in a new window (Figure~\ref{screen5}).

The graph is computed for the displayed X range in the current window. Clicking
the middle mouse button inside the Kiviat graph displays a table with the
statistical values graphed.

\scaleepspic[0.4\textwidth]{screen5.eps}{Kiviat Graph\label{screen5}}

%*******************************************************************************

\subsection{General Session Operations}

DEVise maintains a list of view and window manipulations to facilitate undo
operations as needed, such as when a zoom or scroll is inadvertant or
exploratory. To access this list and return to a previous data display state,
select the \menu{History} menu option in the \menu{Help} menu. A dialog box will
appear containing display states viewed since the session was opened
(Figure~\ref{screen6}), from which the desired previous state may be determined.

\scaleepspic[0.7\textwidth]{screen6.eps}{History Screen\label{screen6}}

The button labeled Display in the Control Panel is useful when resizing windows.
When a large data set is being visualized, DEVise will update and redisplay
window contents while the window is being resized, degrading performance
significantly. Toggle the Display button to change from Display mode to Layout
mode, in which view contents are not redrawn. Once the resize process in
complete, toggle the button back to Display mode to refresh the view contents
and continue the session.

To conclude a session use the \menu{Close} menu option in the \menu{Session}
menu. To exit from the program use the \menu{Quit} menu option.

This completes the tutorial of DEVise. For more in depth information and details
on the operation of other features, consult the earlier sections of this
document.

%*******************************************************************************
% Resources
%*******************************************************************************

\section{Resources}

The DEVise WWW page at \code{http://www.cs.wisc.edu/~devise/} contains a
browseable version of this document.

%*******************************************************************************
% Legal
%*******************************************************************************

\newpage 
\section*{Copyright}

\verbatiminput{Copyright}

\section*{Disclaimer}

\verbatiminput{Disclaimer}

\newpage
\section*{Agreement}

\verbatiminput{Agreement}

\end{document}

%*******************************************************************************
