% ========================================================================
% DEVise Data Visualization Software
% (c) Copyright 1992-1997
% By the DEVise Development Group
% Madison, Wisconsin
% All Rights Reserved.
% ========================================================================

% Under no circumstances is this software to be copied, distributed,
% or altered in any way without prior permission from the DEVise
% Development Group.

%###########################################################

% $Id$

%###########################################################

\documentstyle[fullpage,11pt]{article}

\begin{document}

\author{Jussi Myllymaki}
\date{May 17, 1996}
\title{ClientServer Library\\{\small Version 0.9}}
\maketitle

\section{Introduction}

The ClientServer library is a collection of C++ classes that allows a
developer to quickly create a client-server application. The class
library provides a Tcl/Tk client process which communicates with a
server process using TCP/IP. Two kinds of servers are included in the
library. One type of server uses an XWindows display to respond to
commands sent to it by the Tcl/Tk client. Another type of server
operates in text mode only. Both types of servers can send
asynchronous messages to the client. The XWindows server, for example,
uses these messages to inform the client of window events such as
resizing or button presses.

Section~\ref{installation} describes how to install this library, and
Section~\ref{usage} describes how to use the existing classes and how
to customize them via class inheritance to fit a particular
application need.

\section{Installation\label{installation}}

The package is distributed as an object library ({\tt .a}) file
together with a set of header files that define the interfaces. The
files can be copied from /p/devise/ClientServer. The following
platforms are supported:

\begin{table}[h]
\centering
\caption{Supported Platforms}
\bigskip
\begin{tabular}{lll}
Platform & Library File & Sample Makefile \\
\hline
IBM RS6000/AIX 3.2    & libcs.rs\_aix32.a     & Makefile.aix \\
HP PA/HP-UX 9.0       & libcs.hp700\_ux90.a   & Makefile \\
Intel X86/Linux 1.2   & libcs.x86\_lnx12.a    & Makefile.linux \\
Intel X86/SunOS 5.5   & libcs.sunx86\_55.a    & Makefile.solaris \\
Sun Sparc/SunOS 5.5   & libcs.sun4x\_55.a     & Makefile.solaris \\
Sun Sparc/SunOS 4.1.2 & libcs.sun4m\_412.a    & Makefile \\
DEC Alpha/OSF/1 3.2C  & libcs.alpha\_osf32c.a & Makefile.osf \\
DEC R3000/Ultrix 4.3a & libcs.pmax\_ul43a.a   & Makefile \\
SGI MIPS/IRIX 5.3     & libcs.sgi\_irix53.a   & Makefile.sgi
\end{tabular}
\end{table}

The ClientServer library is installed by copying a set of files from
the /p/devise/ClientServer directory to your own directory. First,
copy the library file that corresponds to your platform and name it
libcs.a. Then create an include directory and copy header files to
that directory from /p/devise/ClientServer/include.

A sample client and server application can also be copied from
/p/devise/ClientServer. Copy client.C, client.tcl, and server.C to
your directory. Copy the Makefile indicated in the table above
corresponding to your platform and name it Makefile. The sample client
application is a Tcl/Tk program with simple pull-down menus. The
sample server displays a rectangle and some text in a window when a
client connection is established. The server responds to window resize
events and also prints the mouse tracking events in the window. The
server responds to client commands that begin with the words ``file''
or ``edit'' by printing the command strings.  Unrecognized commands
are flagged as errors which causes the TclClient to pop up an error
message dialog. The server also understands the ``exit'' command which
the client sends in response to the File/Exit menu command.

\section{How to Use the Library\label{usage}}

In this section we describe the Client and Server classes and some of
the networking routines. For documentation on the other classes
(XDisplay, XWindowRep, and others), we refer the reader to the
corresponding header files.

\subsection{Client Class Hierarchy}

The base class in the client class hierarchy is Client which has all
the functions needed to communicate with server, but it doesn't have
an interface with a user. A derived class, TclClient, provides one
user interface, namely a Tcl/Tk interface. TclClient is created with
the hostname of the server, the port number, and Tcl script filename.
Once the connection to the server is established, TclClient executes
the Tcl script.

The Tcl script issues commands to the server by prefixing them with
the Tcl command ``Server''. Command arguments following the prefix are
sent to the server as ASCII strings. The client waits for a response
from the server which is returned as the value of the Tcl command.
The server may indicate an error which then causes the TclClient to
return TCL\_ERROR which in turn causes Tcl to pop up an error message.

To customize the TclClient for a particular application, derive a
class from it and overload the ControlCmd() virtual function. This
function is used to respond to asynchronous control commands received
from the server. The default action is to execute the command in the
Tcl interpreter (assuming that the server knows to send Tcl commands).

\subsection{Server Class Hierarchy}

The Server base class provides the basic mechanisms for waiting for a
client connection, receiving commands from the client, and terminating
the connection when the client sends an ``exit'' command or just
closes the TCP/IP connection. Once the connection is terminated, the
server starts waiting for another client connection. Currently the
server supports one client connection at a time. Support for multiple
concurrent connections is a fairly simple extension, as is an
extension to have a ``main server'' fork a child process to handle an
incoming client connection.

The WinServer class derives from the Server base class, providing
XWindows support in the server code. When the server executable is
started, it connects to the default XWindows display (defined by the
DISPLAY environment variable). Note that this is not necessarily the
same as the host address of the client that is connecting. (This
behavior may change at a later date.)

To customize the Server or WinServer class to a particular
application, derive a class from one of them. You will need to provide
a ProcessCmd() function that receives the command from the client and
acts upon it. This function essentially defines the command language
understood by the server. The base server understands only one
command: exit. This command terminates the client connection.

For each command received, the ProcessCmd() must send a reply back to
the client as the client blocks until it receives a reply. See
Section~\ref{network} for a description of the network routines used
for communication with the client.

When a new client connection is established, the function
BeginConnection() is called. When the connection is about to
terminate, EndConnection() is called. A server derived from a
WinServer may want to create or destroy windows in these functions.

A server that is derived from the WinServer class can provide a set of
functions that are executed when an XWindows event occurs on one of
the windows created by the server. The functions and corresponding
events are briefly listed here. See ClientServer.h or the sample file
server.C for more details. The default action is no action.

\begin{itemize}
\item
HandleResize() handles ConfigureNotify events (move and resize window)
\item
HandleButton() handles ButtonPress, ButtonRelease, and Motion events
\item
HandleKey() handles KeyPress events
\item
HandleExpose() handles Expose events
\item
HandleWindowMappedInfo() handles Map and Unmap events
\end{itemize}

Note that in the current code release the server receives an event for
every movement of the mouse, whether a button is pressed or not. The
server may want to send a control message to the client inside these
handler functions. See Section 3.3 for details on control messages.

\subsection{Network Routines\label{network}}

The type of a message exchanged between a client and a server is
identified by one of four values: API\_CMD, API\_ACK, API\_NAK, or
API\_CTL. The client uses API\_CMD to send a command to the server which
the server when executes. The server responds with a success code of
either API\_ACK (successful command execution) or API\_NAK (failure in
command execution). Asynchronous commands sent by the server to the
client are indicated by the API\_CTL type.

Of the functions defined in Network.h, two will be most useful for the
server and one for the client. The first one is the NetworkPaste()
function which the server and client can use the to concatenate the
argument vector received from the other party.

The second useful function is NetworkSend() which the server uses to
send replies or control messages to the client. The first argument is
the socket which, for the Server class is \_clientFd. The next
argument is the message type (API\_ACK, API\_NAK, or API\_CTL). The
third argument indicates whether the network send routine should
protect spaces found in the strings in the argument vector. If the
value is non-zero, the strings will be enclosed in curly braces {like
this} to protect spaces. This is typically used by the server to
return multi-element, multi-word replies back to the Tcl client, which
can then call NetworkPaste() to produce a single string and call
Tcl\_Eval() to execute it.

The remaining two arguments to NetworkSend() hold the number of
arguments (argc) and pointers to the argument strings (argv).

\section{Support}

For technical support or suggestions, send email to jussi@cs.wisc.edu.

\end{document}
